



\begin{document}

Introduction to Topo-Philosophy





B. Skowron, \textit{Część i~całość: w~stronę topoontologii}, Oficyna Wydawnicza Politechniki Warszawskiej: Warszawa 2021.



In philosophy, it is always refreshing to introduce unconventional ideas. It requires a~certain audacity from the author; she/he may face the wall of silence or be shunned by academia, both treatments being undesirable. But still these are more rewarding than gathering laurels for beating the dead philosophical cats like Humes, Leibnitzs, Wittgensteins, Whiteheads and others, the practice that for many philosophers is their life opus. Bartłowmiej Skowron\footnote{Bartlomiej Skowron is an adjunct professor at the Faculty of Administration and Social Sciences at Warsaw University of Technology. B. Skowron research interests include formal ontology (part-whole theory, mereotopology), phenomenology, philosophy of morality, axiology, philosophical anthropology, the basis and philosophy of mathematics, applied logic and applied topology. He recently edited a~special issue of ZFN on category theory 
%\label{ref:RNDtxYl8OmCwG}(see editorial note Eckstein and Skowron, 2020)
\parencite[see editorial note][]{eckstein_is_2020} %
 and the book on contemporary Polish ontology 
%\label{ref:RNDDAoLfBPmL1}(Skowron, 2020; reviewed by Krzanowski, 2020).
\parencites[][]{skowron_contemporary_2020}[reviewed by][]{krzanowski_contemporary_2020}.%
}'s book \textit{Part and Whole: Towards Topo-Onotology} published by Oficyna Wydawnicza Politechniki Warszawskiej in 2022\footnote{The book was published in Polish with the title \textit{Część i~całość.W stronę topontologii}. The book is also available in in an open access model as a~PDF file at https://philarchive.org/archive/SKOCIC-2.} is certainly not in this last category.



The book is quite rich in content and topics. It may be seen, as the author suggests, as a~review of mereological and topological perspectives 
%\label{ref:RNDx0A8hPcE4N}(Skowron, 2021, p.xi)
\parencite[][p.xi]{skowron_czesc_2021} %
 or a~topological vision of what exists 
%\label{ref:RNDJcU0MgN4fz}(Skowron, 2021, p.xi).
\parencite[][p.xi]{skowron_czesc_2021}. %
 The book may also be regarded as an advanced introduction to topology, mereology, and mereo-topology as well as their historic roots, beginning with Plato and ending with Brentano and Ingarden.\footnote{Taking a~historical perspective in philosophy is always a~rewarding exercise, if only to teach us humility. This goes contrary to the deep-seated conviction in Anglo-Saxon philosophy that philosophy began with Hume and Locke et al., and everything that came before---with the exception of Plato, Aristotle, and a~few others---were musings about ultimate questions that were of no importance, both in terms of the questions and the musings.} The concept of an advanced introduction is clearly an oxymoron, yet it still seems to reflect the book's content. For example, the definitions and formalism in the book certainly go beyond an introductory level but the chapters are relatively short, hence introduction. However, the unique contribution of this book lies, it seems, somewhere other than the essays on these topics.



We believe that the center of gravity for this book lies in its discussion of topo-philosophy,\footnote{This view is also hinted at by the author, with him saying that ``the third chapter [the chapter discussing topo-philosophy] is essential for the book'' 
%\label{ref:RND97lJ3HNuVa}(Skowron, 2021, p.xix).
\parencite[][p.xix]{skowron_czesc_2021}.%
} so we expect Skowron to introduce us to topo-philosophy and explain what topo-philosophy is, who has engaged with it, and where it may go in future.



Now, why might topo-philosophy be interesting and worthy of attention? The answer to this question is rather long but rewarding. We are told the following: ``Philosophy, in particular its theoretical part, is too difficult to be apprehended with common sense and everyday reasoning'' 
%\label{ref:RNDXH91RTKmZu}(Skowron, 2021, p.xvi).
\parencite[][p.xvi]{skowron_czesc_2021}. %
 So, what is needed to address this? The author states that a~deep understanding of philosophical ideas requires a~deep understanding of the fundamental philosophical structure for concepts like that of the whole and parts, of unity, of foundation, of place, and of autonomy. Topo-philosophy---as a~fusion of topology, topo-ontology, mereology, and philosophy---offers tools for analyzing these complex philosophical structures by juxtaposing them with concepts from topology, such as topological spaces, connectedness, borders, subspaces, density, dimensions, and metrics. Now, let us attempt a~simpler explanation.



Philosophy is about ideas and their structures, while topology is about the properties of a~geometric object that are preserved under continuous deformations, mereology is the study of parts and the wholes they form and topo-ontology---fusion of topology, mereology and ontology---is about topological-like structures of ontological concepts. This means that topo-philosophy is about topological representations of philosophical ideas that go beyond mere ontology. In the author's own words, ``philosophy using spatio-topological concepts is denoted as topo-philosophy'' 
%\label{ref:RNDFItaH2Hc3G}(Skowron, 2021, p.xi).
\parencite[][p.xi]{skowron_czesc_2021}.%




A~two other explanations of topo-philosophy can be found in the book: (1) ``[…] topo-philosophy belongs to mathematical philosophy or some philosophy that uses the language of mathematics to express philosophical concepts, with the proviso that topo-philosophy uses the language and concepts of topology'' 
%\label{ref:RNDYgDZC75Zk0}(Skowron, 2021, p.153).
\parencite[][p.153]{skowron_czesc_2021}. %
 Alternatively, (2) ``Topo-philosophy is based on the judicious application of ideas of geometry [esprit de géométrie is Skowron's suggestion]'' to philosophy (p.169). Geometry always involves an ordering of things, and topo-philosophy is simply doing the same in ordering the conceptual space of philosophy (p.171). ``Judicious application'' may be the key phrase here, because ``topologization'' must be done with ``esprit de finesse'' (again Skowron's suggestion), otherwise it may lose its power to give insight into non-topological ideas and morph into a~barren abstract discourse.



Thus, how to ``topologize'' philosophy can be learned by studying applications of topo-philosophy, for which Skowron provides ample examples. Possibly the most accessible one, thanks to it having already enjoyed wider recognition, is the catastrophe theory of Rene Thom. Nevertheless, Skowron discusses applications of topo-philosophy in epistemology, physics, robotics, data analysis, and models of the mind and the central nervous system. Indeed, topo-philosophy is really coming out into the open.



We see the emerging applications of topo-philosophy in research for AI, information, and deep neural networks (DNN), which are topics not covered in this book. Quantified theories of information have a~topological side in terms of topological information and information geometry. Information geometry was defined by its founder Shun-ichi Amari 
%\label{ref:RNDNOYbRcPFMx}(2016)
\parencite*[][]{amari_information_2016} %
 as ``[…] a~method of exploring the world of information by means of modern geometry.'' Information geometry studies information science---which is an umbrella term grouping statistics, information theory, signal processing, machine learning, and AI 
%\label{ref:RNDZ16DsgJt2D}(Nielsen, 2020)
\parencite[][]{nielsen_elementary_2020}%
---through geometry. Information geometry provides a~context-free, pure method for studying relations like the distance between, for example, probability distributions. Information science can be viewed as the science of deriving models from data, which is often presented as the geometry of decision making, such as through curve fitting and classification 
%\label{ref:RNDDFJNifdSvJ}(Nielsen, 2020; 2022).
\parencites[][]{nielsen_elementary_2020}[][]{nielsen_many_2022}. %
 Topological information views information geometry as being topological. Thus, information is topological in the sense that the relations between systems that manipulate and exchange information can be captured through topological relations.



A~topological representation of information and computing allows for Turing machines and computing to be generalized to information manipulation on tangle machines.\footnote{``Tangle machines are topologically inspired diagrammatic models. Their novel feature is their natural notion of equivalence. Equivalent tangle machines may differ locally, but globally they share the same information content. The goal of tangle machine equivalence is to provide a~context-independent method to select, from among many ways to perform a~task, the ‘best' way to perform the task.'' 
%\label{ref:RNDR0SxnolEhe}(Carmi and Moskovich, 2015, p.1).
\parencite[][p.1]{carmi_tangle_2015}.%
} (For more about information topology, see the works 
%\label{ref:RND5ptQtRkOYk}(Moskovich and Carmi, 2014; Carmi and Moskovich, 2015)
\parencites[][]{moskovich_tangle_2014}[][]{carmi_tangle_2015}%
). The advantages of information geometry and topological information lie in their power to capture various forms of information processing (e.g., information science, decision science) in context-free formal systems based on geometry or topology, thus allowing for results to be generalized from a~specific domain.



Of course, if the topological perspective is so revealing, we may wonder why we did not realize this before. Indeed, Skowron's book is an eye opener to some extent.



However, focusing on topo-philosophy may not do Skowron's work justice, because it is only a~small part of his book. Substantial parts are devoted to reviewing topological research, mereological concepts, mereo-topology, and historical notes. How then should we view these sections? One way is to regard them as a~sort of background introduction to topo-philosophy, but why? Well, if you want to learn about topo-philosophy, you need to understand some basic tenets of mereology, topology, mereo-topology, and topo-ontology, so these sections are helpful as a~reference. It is certainly useful to have them in one place.



One could also forget the notion that the book is about topo-philosophy (the subtitle of the book suggests an introduction to topo-ontology) and treat it as a~series of detailed essays on topological and mereological concepts, with them being connected by the overall theme of topo-ontology, of which a~discussion of topo-philosophy is an integral part. From this viewpoint, Chapter 3 being about topo-philosophy is not central, as we previously presumed. Instead, all the chapters are equally important, and the message is the entire book, which elaborates on the main title and subtitle.



Thus, one may think of the book as a~review of the main tenets of topo-philosophy (unfortunately quite short) together with a~background discussion of topology, mereology, mereo-topology, and so on. Alternatively, one may introduce the book as a~review of the main tenets of topology, mereology, mereo-topology, and so on together with a~side discussion of topo-philosophy(appropriately quite short).



The problem with this second option, however, is that it takes the punch away from the book in terms of its novelty, because topology, mereology, and mereo-topology are rather well-known, well-studied topics.\footnote{Substantial resources on these topics are available. In fact, Skowron provides ample references for all the presented ideas, both historical and modern, so his book is a~self-contained, comprehensive source of knowledge for the discussed topics.} In contrast, topo-philosophy seems fairly novel,\footnote{A~Google search for topo-philosophy directed us to a~Wiki page on El topo: El Topo is a~1970 Mexican acid Western art film based on ``symbolism and Eastern philosophy,'' a~topic certainly outside the scope of Skorwon's book (accessed at https://en.wikipedia.org/wiki/El\_Topo). In addition, topo-philosophy does not register in the Google Ngram Viewer, so this concept has been banished into the Internet's conceptual never-never land. In contrast, the presence of topology, mereology and mereo-topology is well established. This also means that LLMs like GTP-X will not be writing essays on topo-philosophy anytime soon, but it may do so for topology or mereology (\textit{signa temporum}).} despite its deep historical roots, so as something rather unique, topo-philosophy would be a~good choice to serve as the fulcrum for the book, as we originally suggested.



There are a~few more impressions from reading the book. The book is certainly not an easy read, and the presentation of topology, mereology, and mereo-topology is relatively advanced. For an expert, the book offers a~fairly comprehensive review of these topics. In contrast, if one wants to learn about topology, mereology, or mereo-topology, these sections in the book are not the place to start. As we said earlier, it is a~rather advanced introduction. In other words, the book provides a~formal introduction to the topics and is rather shy on conceptual or intuitive perspectives. (For an easier ride into topology, see, for example, the work of Earl 
%\label{ref:RNDxE7Hpp2xnW}(2019)
\parencite*[][]{earl_topology_2019} %
 and the philosophy of mereology by Lando 
%\label{ref:RNDVWJEpuRF4z}(2017).
\parencite*[][]{lando_mereology_2017}.%
) Skowron is well aware of this, however, and from time to time, he shows a~lighter side (Socrates' sting). Overall, though, the thorough, formal approach makes the book a~hard nut to crack. Every author has to make choices, and this book was certainly not intended for display on airport bookstands.



There are also a~few minor things that catch the eye: (1) The claims for the ``entropy of philosophical systems'' 
%\label{ref:RNDzlZG4W5EOg}(Skowron, 2021, p.172)
\parencite[][p.172]{skowron_czesc_2021} %
 and ``entropy as a~measure of unpredictability'' (after Hutchins 
%\label{ref:RND4nadzPgTrP}(2012)
\parencite*[][]{hutchins_concepts_2012}%
), seem to be a~misadventure, albeit one that is quite popular in philosophy. Thermodynamic entropy is a~well-understood physical phenomenon that has little to do with the state of philosophical systems. Any application of thermodynamic entropy concept outside of its proper context, while quite common 
%\label{ref:RNDyOyW11cEuR}(see e.g. Müller, 2007),
\parencite[see e.g.][]{muller_history_2007}, %
 are misleading.\footnote{``For level-headed physicists, entropy---or order and disorder---is nothing by itself. It has to be seen and discussed in conjunction with temperature and heat, and energy and work. And, if there is to be an extrapolation of entropy to a~foreign field, it must be accompanied by the appropriate extrapolations of temperature and heat and work. If we wish, we can now assign an entropy to the message which Shakespeare sent us when he wrote Hamlet: We look up the probability of each letter of the English alphabet, count how often they occur in Hamlet and calculate Inf (Shannon's information entropy). People do that and we may suppose that they know why. Ingenious as this joke may be, it provides no more than amusement.'' 
%\label{ref:RNDpGCEazyNqm}(Müller, 2007, pp.133–134).
\parencite[][pp.133–134]{muller_history_2007}. %
 } (2) The book would benefit from a~more extended synthesis of the discussed ideas. We have a~short synthetic view of what the topo-philosophical method may be but more would benefit the book. (3) The connection between English sources (many quoted key works are in English) and the Polish text would be greatly facilitated if the author provided a~lexicon of English technical terms rendered in Polish. (4) The book may also benefit by focusing on topo-philosophy and its main actors, objectives, and applications from a~historical perspective. As we have said, topo-philosophy is where the novelty of this book appears to dwell, so why not dedicate the book to it at the expense of thinning out the contextual parts? In addition, a~more focused book would be more amenable to being published in English, which I~think would be worth doing. (5) Moreover, an English edition of the entire book, or selected parts thereof, would bring some interesting works from Polish philosophers to a~wider audience, so it is certainly worth considering.



Overall, the book is a~well-executed foray into topo-ontology or topo-philosophy, depending on whichever lens you prefer to use. More specifically, whatever perspective you may adopt, Skowron offers a~much-needed review of the main discussions, players, applications, and perspectives related to topo-philosophy, something that is hard to find collected in one place, so this is certainly a~plus. What the reader may wish to see, however, is more of the author's synthesis for the presented ideas, expanded beyond ``Towards general topology of object and its parts'' in section 7. One may also follow up on Skowron's ideas in his recently published paper ‘A metaphysical foundation for mathematical philosophy'' 
%\label{ref:RNDh35Q7fwlFn}(Wójtowicz and Skowron, 2022).
\parencite[][]{wojtowicz_metaphysical_2022}.%




References



Amari, S., 2016. \textit{Information Geometry and Its Applications}. Applied Mathematical Sciences. [online] Tokyo: Springer Japan. https://doi.org/10.1007/978-4-431-55978-8.



Carmi, A.Y. and Moskovich, D., 2015. Tangle machines. \textit{Proceedings of the Royal Society A: Mathematical, Physical and Engineering Sciences}, [online] 471(2179), p.20150111. https://doi.org/10.1098/rspa.2015.0111.



Earl, R., 2019. \textit{Topology: A~Very Short Introduction}. Very Short Introductions. Oxford, New York: Oxford University Press.



Eckstein, M. and Skowron, B., 2020. ``Is logic a~physical variable?'' Introduction to the Special Issue. \textit{Philosophical Problems in Science (Zagadnienia Filozoficzne w~Nauce)}, [online] (69), pp.7–13. Available at: {\textless}https://zfn.edu.pl/index.php/zfn/article/view/536{\textgreater} [Accessed 18 September 2023].



Hutchins, E., 2012. Concepts in Practice as Sources of Order. \textit{Mind, Culture, and Activity}, [online] 19(3), pp.314–323. https://doi.org/10.1080/10749039.2012.694006.



Krzanowski, R., 2020. Contemporary Polish ontology. Where it is and where it is going. \textit{Philosophical Problems in Science (Zagadnienia Filozoficzne w~Nauce)}, [online] (69), pp.294–298. Available at: {\textless}https://zfn.edu.pl/index.php/zfn/article/view/531{\textgreater} [Accessed 18 September 2023].



Lando, G., 2017. \textit{Mereology: a~philosophical introduction}. London; New York: Bloomsbury academic.



Moskovich, D. and Carmi, A.Y., 2014. \textit{Tangle Machines II: Invariants}. https://doi.org/10.48550/arXiv.1404.2863.



Müller, I., 2007. \textit{A~history of thermodynamics: the doctrine of energy and entropy}. Berlin; New York: Springer.



Nielsen, F., 2020. An Elementary Introduction to Information Geometry. \textit{Entropy}, [online] 22(10), p.1100. https://doi.org/10.3390/e22101100.



Nielsen, F., 2022. The many faces of information geometry. \textit{Notices of the American Mathematical Society}, [online] 69(1), pp.36–45. https://doi.org/10.1090/noti2403.



Skowron, B. ed., 2020. \textit{Contemporary Polish Ontology}. Philosophical Analysis. [online] Berlin; Boston: De Gruyter. https://doi.org/10.1515/9783110669411.



Skowron, B., 2021. \textit{Część i~całość: w~stronę topoontologii}. 1st ed. Warszawa: Oficyna Wydawnicza Politechniki Warszawskiej.



Wójtowicz, K. and Skowron, B., 2022. A~metaphysical foundation for mathematical philosophy. \textit{Synthese}, [online] 200(4), p.299. https://doi.org/10.1007/s11229-022-03760-5.

\end{document}

