\begin{newrevengenv}{Andrzej Anderwald}
	{The interdisciplinary profile of theology---fashion or necessity?}
	{The interdisciplinary profile of theology---fashion or necessity?}
	{The interdisciplinary profile of theology---fashion or necessity?}
	{University of Opole}
	{Wojciech P. Grygiel, Damian Wąsek, \textit{Teologia ewolucyjna. Założenia---problemy---hipotezy}, Copernicus Center Press, Kraków 2022, ss.~268.}









\lettrine[loversize=0.13,lines=2,lraise=-0.03,nindent=0em,findent=0.2pt]%
{S}tudies of the complexity of reality, which are carried out at the intersection of different areas of knowledge, are nowadays gaining more and more supporters among researchers. Three directions of these studies dominate today: multidisciplinarity, intradiscilpilnarity and interdisciplinarity. The last and the most frequently used means taking up common research projects by scientists coming from different disciplines. This occurs at each stage of the project beginning with the formulation of a~research problem, then proposing appropriate hypotheses and ultimately interpreting data that were obtained 
%\label{ref:RNDVphmznouhr}(Mette, 1996).
\parencite[][]{kasper_interdisziplinaritat_1996}. %
 Since the Second Vatican Council, theology has also become more open to cooperation with other sciences, including not only the humanities, but various disciplines of empirical sciences such as physics, chemistry, biology, psychology, sociology and cognitive science. The need of this kind of cooperation is signaled at present not only by the representatives of theology but by those involved in empirical sciences as well. Some researchers even go so far as to make the future of theology and its presence at universities dependent on the interdisciplinary direction of research. However, the interdisciplinary openness of theology raises a~number of questions: Does opening theology to other disciplines not threaten to break the internal identity of its content? Why should a~theologian listen to the voices of representatives of other sciences since many of them are not interested in his research discipline? Does theology have something to say to other sciences in an interdisciplinary dialogue? Is theology itself interested in opening up to new \textit{loci theologici alieni}? Can the interdisciplinary nature of theology help in creating an integral concept of man and the world? Can interdisciplinary theological research contribute to the clarification and transmission of faith among people with scientific and technical mentality?



An unequivocally positive answer to the questions posed is provided by the authors of the book \textit{Evolutionary Theology}: Wojciech Grygiel, a~natural philosopher, chemist, theologian, and Damian Wąsek, a~theologian. As representatives of various disciplines through their joint work they give a~concrete example of the interdisciplinary cooperation.\footnote{It is worthy of note that this is not the first interdisciplinary work by these authors, for example, see 
%\label{ref:RNDqynK0y5WAi}(Wąsek, 2018; Grygiel, 2020; 2021)
\parencites[][]{wasek_teologia_2018}[][]{Wasek2021}[][]{Grygiel2019IntelligentDesign}[][]{grygiel_cognitive_2021}.%
} Their project is methodological in nature: it is to ``show how the development of science can entail the development of theology, and what assumptions must be met to result in a~constantly deepening insight into the divine essence'' 
%\label{ref:RNDE8DsEpdEoG}(Grygiel and Wąsek, 2022, p.12).
\parencite[][p.12]{grygiel_teologia_2022}. %
 The book consists of two main parts: \textit{Assumptions} (pp. 15–151) and \textit{Problems and Hypotheses} (pp. 152–236).



The first part, consisting of five chapters, discusses the methodological assumptions of evolutionary theology. Chapter 1 under the title \textit{Revelation: between formulas and relation} 
%\label{ref:RNDxETKKVO3rf}(2022, pp.16–37),
\parencite*[][pp.16–37]{grygiel_teologia_2022}, %
 considered by the authors to be the most important for capturing the essence of evolutionary theology, presents a~dynamic concept of theology of revelation in the perspective of its historical development. This approach shows, with all its internal dynamics, the importance of the concept of revelation in the construction of evolutionary theology. The category linking the conducted considerations is the relationship between man and the self-giving God taking place against the background of changing images of the world. Chapter 2 entitled \textit{Truths of Faith: Between Immutability and Evolution} 
%\label{ref:RNDclC81FaA6k}(2022, pp.38–60)
\parencite*[][pp.38–60]{grygiel_teologia_2022} %
 presents the tension between what is essential, unchangeable and changeable in the interpretation of the truths of the Christian faith. The latter conditioned by the time context, especially the specific image of the world related to it, may be subject to change, reinterpretation or even correction. In reference to theological thought, e.g., John Henry Newman 
%\label{ref:RNDyceyT4uvZM}(2022, pp.43–44)
\parencite*[][pp.43–44]{grygiel_teologia_2022} %
 and Karl Rahner 
%\label{ref:RNDbF9ZGhYqvX}(2022, pp.50–51)
\parencite*[][pp.50–51]{grygiel_teologia_2022} %
 the authors clearly point to the evolution of the Christian doctrine taking place in compliance with certain rules. ``The doctrine---as they write---is not about accepting historical formulations, but their inner essence. At the same time, one should be aware that this ``explication'' cannot be expressed in an ahistorical way'' 
%\label{ref:RNDhGHJc8jW1T}(2022, p.51).
\parencite*[][p.51]{grygiel_teologia_2022}. %
 Unlike the descriptive titles of the previous chapters, the third chapter with a~metaphorical title \textit{Theology as a~Work for Orchestra} 
%\label{ref:RNDZia7eHmRpK}(2022, pp.61–84)
\parencite*[][pp.61–84]{grygiel_teologia_2022} %
 deals with the classification of the theological places. Theology distinguishes two types of its sources: proper places (\textit{loci proprii}) and auxiliary ones (\textit{loci alieni}). It is the latter, based solely on human authority, that is the subject of the analysis conducted in this chapter. The authors are particularly interested in defining more precisely the criteria for interdisciplinary dialogue involving theology 
%\label{ref:RNDSGQiotOG68}(2022, pp.77–84),
\parencite*[][pp.77–84]{grygiel_teologia_2022}, %
 so strongly related to the topic of \textit{loci alieni}. This issue seems crucial in developing a~methodology for evolutionary theology. The fourth chapter, \textit{From a~static to a~dynamic image of the world} 
%\label{ref:RNDNnpUyAFePa}(2022, pp.85–121),
\parencite*[][pp.85–121]{grygiel_teologia_2022}, %
 deals with the analysis of various ways of interaction between the language of theological statements and the current image of the world. The adoption of the hermeneutic category of the image of the world, as proposed here, makes it possible to better capture and understand the complex contextuality of not only scientific, but also religious and theological beliefs 
%\label{ref:RNDNhVUSVs8wr}(2022, pp.86–87).
\parencite*[][pp.86–87]{grygiel_teologia_2022}. %
 In a~broader historical perspective, as indicated by the title of the chapter, the authors successfully indicate both the ongoing transformations of the image of the world effected by the new discoveries in physics and biology, as well as the related reinterpretations at the level of the philosophy of science 
%\label{ref:RNDT6sd5WE23e}(e.g., 2022, pp.119–120).
\parencite[e.g.,][pp.119–120]{grygiel_teologia_2022}. %
 In the fifth chapter, \textit{In the flow of logos} 
%\label{ref:RNDewyvRmpxim}(2022, pp.122–151),
\parencite*[][pp.122–151]{grygiel_teologia_2022}, %
 one more argument for the need to develop evolutionary theology is presented. This way of practicing theology in the evolving Universe strives not only to protect against its marginalization by taking into account the current image of the world in theological reflection, but ``is primarily aimed at a~much deeper insight into the mysteries of who God is in essence'' 
%\label{ref:RNDy9u7sMVAmL}(2022, p.122).
\parencite*[][p.122]{grygiel_teologia_2022}. %
 This chapter is a~supplement to the methodological assumptions of evolutionary theology and it is particularly important in setting the epistemic boundaries that prevent from making unjustified extrapolations between natural and theological cognition. ``The existence of such a~border---as we read---is necessary for revelation to make sense, that is, for there needs to be room for the transcendent voice of God who speaks from beyond the immanence of the Logos to its interior'' 
%\label{ref:RNDxY0xYjgLSs}(2022, p.146).
\parencite*[][p.146]{grygiel_teologia_2022}.%




The second part of the book, which consists of three chapters, deals with the application of the discussed assumptions of evolutionary theology in practice. The authors present ``several selected reinterpretation problems [...], aptly illustrating how these assumptions ‘work' in specific cases'' 
%\label{ref:RNDSIqqgNOitu}(Grygiel and Wąsek, 2022, p.13).
\parencite[][p.13]{grygiel_teologia_2022}. %
 And so, in chapter six, \textit{Adam, where are you? Evil and The Original Sin} 
%\label{ref:RNDV7N5FPRMMr}(2022, pp.153–173)
\parencite*[][pp.153–173]{grygiel_teologia_2022} %
 they synthetically discuss the problem of evil and suffering as well as the classical doctrine of the original sin in the perspective of the evolving image of the world. The presented reinterpretation of the doctrine of original sin is a~strong argument for the necessity of practicing theology in the perspective of evolutionary theology 
%\label{ref:RNDcYPgZrVAFx}(2022, pp.168–170).
\parencite*[][pp.168–170]{grygiel_teologia_2022}. %
 Chapter Seven: \textit{Soul: Between Nature and Divine Intervention} 
%\label{ref:RNDcjoMkY7CHL}(2022, pp.174–196)
\parencite*[][pp.174–196]{grygiel_teologia_2022} %
 deals with the analysis of the current topic of the relationship between theology and neuroscience: ``between the biological reality shaped in the process of evolutionary development, and the one that is associated in traditional theology with direct divine intervention'' 
%\label{ref:RNDqet6MjSDMs}(Grygiel and Wąsek, 2022, p.175).
\parencite[][p.175]{grygiel_teologia_2022}. %
 The emphasis is placed here on showing the consequences of this type of relationship both for the reinterpretation of the theological concept of the soul and the very contribution of neuroscience to the theological discourse. In chapter eight, \textit{In the Footsteps of Agency: Cognitive Religious Studies} 
%\label{ref:RNDQxnhakDYlS}(2022, pp.197–233),
\parencite*[][pp.197–233]{grygiel_teologia_2022}, %
 the authors take up a~new research concerning the phenomenon of religion within the cognitive sciences. In their considerations, they strive to capture and show the impact of the scientific knowledge on the formation of religious beliefs. Specific issues are analyzed in turn, such as: the origin of religious beliefs, the doctrine of intelligent design and miraculous events. Some doubts may be raised by the presented analysis of a~miracle which links its recognition as God's way of acting in the world with interpretations inspired by the thought of St. Augustine and St. Thomas Aquinas 
%\label{ref:RND3qBtJC4zp2}(2022, pp.224–230).
\parencite*[][pp.224–230]{grygiel_teologia_2022}. %
 It seems that reference to the modern semantic concept of a~miracle, especially in the layer of its cognition including scientific and religious knowledge, gives the possibility of the interpretation of a~miracle better harmonizing with the evolving image of the world.



The second part of the discussed book is not only an example of a~practical application of the methodological assumptions of evolutionary theology presented in the first part but it also provides specific arguments against the anti-Christian theses of Richard Dawkins. They have been quite widespread recently mainly through his \textit{The God Delusion} \parencite*{Dawkins2006TheGodDelusion} as well as through the naturalistic ideas propagated by the supporters of the new atheism.\footnote{E.g., Sam Harris
%\label{ref:RNDIVsMG8jXye}(2004);
\parencites*[][]{harris_end_2004, harris_letter_2006}, %
% Sam Harris
%%\label{ref:RNDwReNGTtUnv}(2006);
%\parencite*[][]{harris_letter_2006}; %
 Christopher Hitchens 
%\label{ref:RND8HPwdUIG4i}(2009).
\parencite*[][]{hitchens_god_2007}.%
}



In conclusion, the book \textit{Evolutionary Theology} is a~successful study that shows how interdisciplinarity in theological research leads to a~departure from the one-dimensional scientific paradigm and gives the opportunity to develop a~holistic view of nature and man. Especially from the perspective of the fundamental theology interdisciplinarity is not a~fashion but rather a~necessity and an expression of the understanding the complexity of reality which no single science is able to grasp integrally. The research project of the book, its structure in which it combines issues in the field of systematic theology and philosophy of science as well as science itself shows the importance and necessity of the interdisciplinary research for the development of modern theology. The interest of a~modern man in the scientific knowledge is an expression of a~certain sign of the times in which his expectations, needs and requirements are revealed. Also, the reviewed book is a~positive example in search of the new forms to integrate faith and reason as part of the dialogue between theology and other sciences. It shows how to defend the rationality of the Christian faith as it confronts the claims of the contemporary science. The book adds its voice to the attempts of providing this king of defense which are present in the Anglosaxon literature by such authors as John Polkinghorne, Alister McGrath, Gary Keogh, Anderw Pinsent, Markus Holden, and in German by Jürgen Moltmann, Christian Link, Dieter Hattrup, Urlich Lüke and Alexander Loichinger.



A~reliable interdisciplinary exchange may not only lead to discovering the boundaries of one's own scientific discipline, but also to an increase in methodological awareness. Such discoveries result in the mutual cleansing of past errors and guard against unwarranted extrapolations so that theology does not turn into pseudoscience and science into unconscious theology. The reliance on the contribution of the authors' own scientific community to interdisciplinary research, which fits into the research perspective of evolutionary theology, should also be positively assessed. Grygiel and Wąsek not only refer to the achievements of the great precursors and initiators of the Copernicus Center of the Interdisciplinary Studies, Michał Heller and Józef Życiński or their own publications, but they also incorporate the achievements of philosophers and theologians associated with the Pontifical University of John Paul II, such as: S. Wszołek, J. Bremer, J. Mączka, Z. Liana, R.J. Woźniak, Ł. Kamykowski or T. Dzidek.



Based on the above considerations, I~recommend the book \textit{Evolutionary Theology} not only to anyone who is interested in the interdisciplinary dialogue and who wishes to do theology within the changing image of the world but to anyone who is looking for the justification of a~personal Christian creed against the claims of mentality dominated by the scientific thinking.



\vspace{15mm}%
{\subsubsectit{\hfill Abstract}}\\
{This review pertains to the book \textit{Evolutionary Theology} (\textit{Teologia ewolucyjna}) written by Wojciech P. Grygiel and Damian Wąsek. The book presents a distinct and modern viewpoint on theology by offering a~comprehensive analysis of the characteristics of theological language and utilizing it to reevaluate certain theological beliefs, such as the concept of original sin, within the framework of the ever-changing understanding of the Universe. This approach contributes significantly to the restoration of theology's credibility in modern culture by bridging the gap between science and theology.}\par%
\vspace{2mm}%
{\subsubsectit{\hfill Keywords}}\\
{evolutionary theology, interdisciplinarity, science and theology.}%

\enlargethispage{1.5\baselineskip}

\end{newrevengenv}

