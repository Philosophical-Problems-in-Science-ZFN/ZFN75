\begin{artengenv}{Roman Krzanowski}
	{From philosophy in science to information in nature: Michael Heller's ideas}
	{From philosophy in science to information in nature\ldots}
	{From philosophy in science to information in nature: Michael Heller's ideas}
	{University of Sydney}
	{This paper discusses the concept of information that was formulated by Michael (Michał) Heller. Heller---a philosopher, theoretical physicist, cosmologist, and theologian---provided a~complex image of information and its role in nature, one that is rarely found in studies of information. Heller posited that the laws of nature may be interpreted as information, or as providing information, with this being a~complementary view to scientific structuralism (not discussed in this paper). According to Heller, the informational content of a~structure (in nature) is inversely proportional to that structure's degree of freedom. The more constrained or complex, while also being less likely to exist, a~structure is, the more information it contains. In Heller's view, the concept of information presented in the Shannon's Theory of Communication (ToC) is inadequate for expressing the notion of information beyond the concept of a~numerical measure of a~signal structure. Information in Heller's research comes very close to the concepts of Jacquette's and Perzanowski's combinatorial ontology (the concepts not discussed in this paper) and the general theory of information (GTI) of Mark Burgin, although Heller himself did not make these connections.
	}
	{natural information, physical information, information in nature, information in cosmology, Michał Heller, Mark Burgin.}



\section{Introduction}

\lettrine[loversize=0.13,lines=2,lraise=-0.03,nindent=0em,findent=0.2pt]%
{A}{}~modern concept of information (and its quantification) in science and technology was introduced (not created) into the scientific and technical discourse in the mid-20\textsuperscript{th} century by Shannon 
%\label{ref:RNDutSohEfMiR}(1948),
\parencite*[][]{shannon_mathematical_1948}, %
 Shannon and Weaver 
%\label{ref:RND47G0bEhlq4}(1949; 1964),
\parencites*[][]{ShannonWeaver1949}[][]{ShannonWeaver1998}, %
 and Weaver 
%\label{ref:RNDMKzwoQeQhw}(1949),
\parencite*[][]{weaver_mathematics_1949}, %
 and a~flood of research publications on information followed 
%\label{ref:RNDEZA06K6FfX}(e.g., Seising, 2009).
\parencite[e.g][]{carvalho_60_2009}. %
 Nevertheless, after decades of continuous efforts, we still only have a~rather vague understanding of what information is.\footnote{For historical, pre-Shannon, notes on the concept of information, see Vreeken 
%\label{ref:RNDde911hRbTQ}(2005),
\parencite*[][]{vreeken_history_2005}, %
 Adriaans 
%\label{ref:RNDqAGX1Tl5WN}(2020),
\parencite*[][]{adriaans_information_2020}, %
 or Gleick 
%\label{ref:RNDYlgtArikh4}(2011).
\parencite*[][]{gleick_information_2011}.%
} Instead of one definition, we have many 
%\label{ref:RNDNGqtAmEOxQ}(e.g., Adriaans, 2020; Krzanowski, 2022).
\parencites[e.g][]{adriaans_information_2020}[][]{krzanowski_ontological_2022}. %
 Most discussions of information are limited to a~specific context, such as biological information, information in communication, pragmatic information, semantic information, symbolic information, synthetic information, physical information, quantum information, natural information, environmental information, or structural information with variants in each of the classes, although this list is not exhaustive.



On a~few occasions, in an attempt to express information more comprehensively as a~fundamental aspect of reality (an intuition shared with our pre-Socratic, religious colleagues, and some physicist with the bent for metaphysics),\footnote{The papers were published in an edited volume by Davies and Gregersen 
%\label{ref:RNDch8toC7Nqn}(2010).
\parencite*[][]{Davies2010-DAVIAT-5}.%
} researchers have formulated enigmatic \textit{koans} like ``everything is Information'' 
%\label{ref:RNDzV4R7lzlJw}(Jones, 2018),
\parencite[][]{jones_everything_2018}, %
 ``information is the difference that makes a~difference'' 
%\label{ref:RNDl6YHtSaJdX}(see Sloman, 2018),
\parencite[see][]{sloman_what_2018}, %
 or ``It from Bit'' 
%\label{ref:RNDEHJLLaxOot}(Wheeler, 1989).
\parencite[][]{wheeler_information_1989}. %
 Different versions of these have become entrenched in popular culture, yet these sayings do not explain much.\footnote{The fact that these phrases have been entrenched in popular culture does not make them truer. It makes them what they are---a staple of popular culture. Further, one of these koans, a~well-known ``It from Bit'' 
%\label{ref:RNDvdJsZXicmR}(Wheeler, 1989)
\parencite[][]{wheeler_information_1989} %
 implying human effect on QM has been proven wrong in the Delayed Choice Quantum Eraser experiment, the point which, of course, popular publications miss to the detriment of the scientific truth. As other ‘koans' do not pretend to express scientific truths but intuitions, they continue their lives in commons, unchallenged (commons understood as in https://onthecommons.org/).} They serve as useful quips in TED talks or alike, but they are without much impact beyond 
%\label{ref:RNDwREDVIhdRV}(see e.g., Tetlow, 2017).
\parencite[see e.g][]{tetlow_phil_2017}.%
\footnote{In the author's view, enthusiasm about the apparent deep meaning of these koans, quite widespread in popular publications, has not been reflected in advanced discussions on philosophy of information.}



Heller as a~philosopher, theoretical physicist, cosmologist, and theologian provided a~complex image of information and illustrated its role in cosmology, something that is rarely found in studies of information. In what follows, we discuss Heller's account of information, present his extant claims and views about Shannon's information entropy, and present the enigmatic idea of harmony between abstract mathematical structures and nature. We also discuss how Heller's concepts of information fit into the wider modern discussion about information, including the GTI and the idea of the latent order of nature.\footnote{A~part of this paper has been published as D. Phil. thesis 
%\label{ref:RNDzmTe0LbiAE}(Krzanowski, 2022).
\parencite[][]{krzanowski_ontological_2022}.%
}



A~word of caution: Heller's ideas on information do not form a~comprehensive theory of information like Shannon's' TOC, Floridi's \textit{General Definition of Information} (GDI) 
%\label{ref:RNDVZBXwgxHZk}(see e.g., Floridi, 2011)
\parencite[see e.g][]{floridi_philosophy_2011} %
 or Burgin's \textit{General Theory of Information} (GTI) 
%\label{ref:RNDXZvEWepqnw}(Burgin, 2003).
\parencite[][]{burgin_information_2003}. %
 They are dispersed throughout his papers on cosmology and philosophy of science and are more akin to Heraclitus or pre-Socratic fragments than to Shannon's, Floridi's, or Burgin's comprehensive theories. Thus, they have to be in some way weaved out of the larger context. Interpretation of such dispersed fragments is riddled with dangers. On one hand, we want to understand what Heller is telling us about information. On the other hand, we do not want to over-interpret his ideas, as it has been done (sometimes) with pre-Socratics. Therefore, the following presentation of Heller's thoughts on information may be seen by some as incomplete. But, we prefer the presentation to be incomplete in this sense, rather than incorrect, stating what Heller said but not what Heller might have said. Thus, the reader will often find our comments on Heller's fragments ending with the pose, ``Heller does not clarify this intuition further,'' and so we don't do it either.



\section{Heller and Information}

Michael Heller's views on information resulted from his studies of the fundamental structures of the cosmos (i.e., the Universe), its mathematical models, and the properties of nature (i.e., physical phenomena).\footnote{All Heller's quoted writings here have been translated from Polish into English by the author.} In a~series of observations, Heller outlines his vision of information in nature. The first fragment (1) comes from Heller's book \textit{The Introduction to the Philosophy of Science}:



\begin{quote}
(1) The informational interpretation of the laws of nature may be seen as a~complement rather than a~competing option to scientific structuralism 
%\label{ref:RNDcwX7611rKL}(Heller, 2009, pp.62–63).
\parencite[][pp.62–63]{heller_filozofia_2009}.%
\end{quote}




Heller posits that the laws of nature may be interpreted as information, or as providing information, in a~view that complements scientific structuralism.\footnote{Heller's views about laws of nature and structuralism may be found in 
%\label{ref:RNDmvWcWqZOaK}(Heller, 2009).
\parencite[][]{heller_filozofia_2009}.%
} What ``complements'' means here is unclear, but it may be interpreted as saying that there is no dichotomy between the structural and information views of nature, with structure and information both being characteristic of nature.



In fragment (2), Heller interprets Shannon's theory of communication and claims that the increase in the information content of a~structure is inversely proportional to the structure's degrees of freedom.\footnote{The degrees of freedom is the number of independent variables (dimensions) the (any) system may be characterized by or exist within.}



\begin{quote}
(2) According to the modern theory of information, the increase in informational content arises in transition from a~set with a~larger number of degrees of freedom to a~more limited set. For example, the informational content of a~set of all letters will increase for a~set of letters that expresses some sentence 
%\label{ref:RNDkJ6OzGmvCJ}(Heller, 2009, pp.62–63).
\parencite[][pp.62–63]{heller_filozofia_2009}.%

\end{quote}



Heller observes how the laws of nature impose constraints on nature's structures, so they control, in a~way, what can and what cannot be (i.e., not everything is possible in physics).\footnote{An interesting interpretation of the relation between the laws of nature and the organization of natural world is suggested by Laughlin. He writes that ``At the most fundamental level, the laws of physics are laid out in plain sight for everyone to see. Yet you cannot generally predict things with these equations […] [however, there are] collective principles of organization encrypted into these equations'' 
%\label{ref:RNDnzcDoo3DGv}(Laughlin, 2008, p.36).
\parencite[][p.36]{laughlin_crime_2008}. %
 Thus, you may say that the laws of physics define principles of organization or that information is expressed through the laws of physics. It is, however, a~very farfetched conjecture.} What is possible is limited to the very large number of combinations of fundamental elements, so it is constrained by physical laws. The presence of quantum or discrete building blocks then makes the universe possible. This view is also reflected in the models of the universe in combinatorial ontologies, or ancient atomism.



For Heller, the laws of nature act like information (fragments (3) and (4)) in determining and constraining what is possible.



\begin{quote}
(3) Thus, information increases when the number of degrees of freedom decreases. (4) Limited sets (i.e., sets with constraints imposed on them) are nothing but certain structures, and every structure has certain information. The more restrictions that a~given structure possesses, the more information it contains 
%\label{ref:RNDtiBLjkb0Ib}(Heller, 2009, pp.62–63).
\parencite[][pp.62–63]{heller_filozofia_2009}.%

\end{quote}



The more constrained or complex structures are, and therefore less likely to exist, the more information they contain, based on Shannon's law. Thus, do the structures (in nature) code information (fragment (5)) or express information?



\begin{quote}
(5) As the world is a~certain structure, it contains information, because this structure-world encodes information. This information is decoded by science and formulated as the laws of nature 
%\label{ref:RNDeo2HJcV2zl}(Heller, 2009, pp.62–63).
\parencite[][pp.62–63]{heller_filozofia_2009}.%
\end{quote}




Would Heller suggest here that the laws of nature are information, or is information merely their expression? Alternatively, maybe it is a~chicken-and-the-egg problem. Nevertheless, we do not get an answer to this question in Heller's writings.



This interpretation of nature, information, structures, and natural laws is further discussed in Heller's article titled ``Nauka i~wyobraźnia'' [Science and Imagination] 
%\label{ref:RNDgUo1ws5gLi}(Heller, 1995).
\parencite[][]{heller_nauka_1995}. %
 In fragments (6,7, and 8), Heller positions structures and laws of nature as information.



\begin{quote}
(6) Modern theoretical physics suggests that the world does not possess structure but is a~structure. (7) This structure contains encoded information or is information. (8) Science decodes its fragments by fitting mathematical structures to the structures of the cosmos 
%\label{ref:RND3ipRg5vybk}(Heller, 1995, p.170).
\parencite[][p.170]{heller_nauka_1995}.%
\end{quote}




This information, as natural laws, is partially decoded and expressed in scientific laws. While scientific laws do represent a~fragment, or an aspect, of cosmic structures, even though they are obviously much less complex than the natural structures, Heller does not explain in what sense the laws of nature are natural structures.



In fragments (9) and (10), Heller states that while the laws of nature and structures are not isomorphic, they act in concert with nature, which perhaps refers to a~sort of codependency.



\begin{quote}
(9) The decoded fragments of information are denoted as scientific theories or models of nature. (10) The mathematical structures of our theories and the structure of the cosmos are not isomorphic, but there is a~strange resonance, a~harmony between them. Because of this, resonance–harmony theories are grossly simplified in comparison to the structures of the cosmos, but they harmonize with the world, reproducing some of its [structural] properties 
%\label{ref:RNDmBexqPBPLt}(Heller, 1995, p.170).
\parencite[][p.170]{heller_nauka_1995}.%
\end{quote}




In other words, the laws of nature are the causes, or the results of, nature's properties to some degree, although to what degree we are not sure. The point behind this remark is that laws and natural structures are not the same but somewhat codependent. Heller refers to the similarity between nature and abstract mathematical structures as harmony. Harmony, as proposed by Heller, is an intriguing (strange) property of the abstract models of the cosmos. According to Heller, the mathematical models of nature are highly simplified, with respect to the complexity of nature, and formalized. In other words, they have a~high level of abstraction. They are not of the same ``nature'' as physical entities, so how are they able to reflect some of nature's properties quite accurately? In these abstractions, one may be tempted to see Platonic forms and nature as their realization, and such a~view would certainly explain this strange harmony. This would be the position of modern Platonism or mathematical Platonism, which by the way has little to do with the ontology of Plato 
%\label{ref:RNDGVNX0QOpyL}(e.g., Linnebo, 2018).
\parencite[e.g][]{linnebo_platonism_2018}.%




Further explanations for the concepts of nature, structures, information, and form can be found in Heller's paper titled ``Evolution of the concept of mass'' 
%\label{ref:RND6tth03hi1y}(Heller, 1987).
\parencite[][]{heller_ewolucja_1987}. %
 In fragments (11) and (12), Heller posits that information can be thought of as a~foundational element of nature instead of matter.



\begin{quote}
(11) As one must have some image of the world, the image of matter as foundational ``stuff'' must be substituted with another one. The image of the world not as a~material composite but as a~pure form would correspond much better with the findings of modern physics. (12) All models of the cosmos constructed by modern physics are abstract mathematical models. They do not have anything else but shape and structure (i.e., purely formal schema)
%\label{ref:RNDJxzdODhhrY}(Heller, 1987).
\parencite[][]{heller_ewolucja_1987}.%
\end{quote}




In particular, modern physics does model the universe as mathematical formulas through shapes/structures without content. In this view, information is expressed in, or by, the ``empty'' mathematical structures, or these structures are information. Nevertheless, Heller does not clarify this intuition further.



In fragments (13) and (14), Heller suggests that even if there is something beyond these ``Platonic'' structures, modern science is unable to detect it.



\begin{quote}
(13) Even if the real world contains something beyond the form, the modern methods of physics cannot detect it; this something slips through the net of mathematical–empirical methods. (14) In this sense, the world of physics is a~pure form 
%\label{ref:RNDXPWsSRu86n}(Heller, 1987).
\parencite[][]{heller_ewolucja_1987}.%
\end{quote}

This statement approaches the position of epistemic structuralism (i.e., taking the mathematical version of structuralism) in claiming that the structures of nature are mathematical structures of which nothing else (i.e., ontology) cannot be known.



In fragments (15 and16), Heller posits that this concept of information differs from the concept of information arising in the Shannon–Weaver–Hartley theory of communication (ToC).



\begin{quote}
(15) The same concept can be expressed as follows: If we define information as the constraint on degrees of freedom (possibilities), each law of physics is information as it limits the possibilities of nature. (16) One may think that the ``stuff'' of the universe is nothing else but information. But our current understanding of information is purely formal (e.g., Shannon-Hartley theory of information). Thus, information is reduced here to structure rather than to what this structure is filled with. In this view, the structure of the world is an information code, or encoded information, and the role of science is to decipher this code 
%\label{ref:RNDK8FyFTcsRd}(Heller, 1987).
\parencite[][]{heller_ewolucja_1987}.%
\end{quote}




This theory perceives structure as something for encoding something rather than as the ``stuff of the universe.'' Thus, the concept of information in the ToC is inadequate for expressing the notion of information beyond the concept of a~numerical value. In fact, the ToC does not define information, as some have mistakenly concluded, but rather measures it. To put it more precisely, the function defined by Shannon is referred to as a~measure of information (i.e., information entropy), with the elementary unit of information being a~digital bit (0/1). Indeed, Shannon's measure of information (i.e., the entropy of information) does not define information, just as the definition of a~kilogram does not define what mass is. The entropy of information merely quantifies a~specific property of a~modulated physical phenomenon (i.e., a~signal) under certain assumptions of syntax. Thus, it no more defines information than the definition of a~kilogram defines what mass is. It is instead simply quantifying a~certain property of a~certain physical phenomenon (a signal) under certain assumptions. Thinking of the ToC this way is less prone to misinterpretations and may be closer to Shannon's original intention.



\section{Heller on Information in Perspective}

If we were to consider the most insightful ideas from Heller, what would they be? The statement that ``the concept of information in the ToC is inadequate for expressing the notion of information beyond the concept of a~number'' would certainly count as one. Most studies of information in any domain base their concepts of information on Shannon's information metrics (i.e., information entropy). Few people, including Shannon himself 
%\label{ref:RNDPWpbz9tYpZ}(Shannon, 1956),
\parencite[][]{shannon_bandwagon_1956}, %
 foresaw this profusion of concepts stemming from his idea and warned against this. Indeed, these ``Shannon's extensions'' are often over-interpretations (of the original intent) or to put it more bluntly, misinterpretations of the original idea and purpose. Shannon developed his ToC as a~theory of communication for measuring the efficiency of a~communication channel in the presence of noise and little more than this. Shannon's information entropy, in Heller's view, is a~metric for certain observable structures that depending on what information is, may or may not contain information. As it happens, if we ask in what sense is this information, we generally get lost in explanations, or mathematics.\footnote{Any measure of information based on shape/form does not measure information but rather its effect in nature. In addition, any measure of information based on shape/form/morphology actually contains/conceals a~time variable, as pointed out by Burgin 
%\label{ref:RNDDCTQkaqTGz}(2010),
\parencite*[][]{burgin_theory_2010}, %
 so such measures should be indexed by time. For example, Shannon's information entropy ``IE'' should be rewritten as ``IEt''.}



The next insight from Heller's work would be the notion that information is somewhat expressed as the natural (and by extension any) structures and laws of nature while being neither of these. According to Heller, the structures only encode or express information. Information lies beyond the visible and is expressed in, or by, ``empty'' mathematical structures, or these structures are information. Interestingly, Heller never associates information with meaning, such as knowledge or data, as many do 
%\label{ref:RNDkQicQ824sl}(e.g., Losee, 1997; Sveiby, 1998; Casagrande, 1999; Dretske, 1999; Floridi, 2010; 2011; 2019; Lenski, 2010; Vernon, 2014).
\parencites[e.g][]{losee_discipline_1997}[][]{sveiby_what_1998}[][]{casagrande_information_1999}[][]{dretske_knowledge_1999}[][]{floridi_information_2010}[][]{floridi_philosophy_2011}[][]{floridi_semantic_2019}[][]{lenski_information_2010}[][]{vernon_artificial_2014}. %
 However, Heller's information in the physical world is just form or form behind form, with meaning as in knowledge coming from, and with, us.



In Heller's view, with ``information expressed in or by ‘empty' mathematical structures,'' information comes close to Platonic or platonic forms,\footnote{The term ``Platonic'' refers to the original teachings of Plato himself, while the term ``platonic'' refers to modern versions of Plato's metaphysics.} a~metaphysical position that has a~ring of truth to it, but this does not go down well with hardline physicalists. Nevertheless, the fact is that Burgin's theory of information (GTI) is arguably the most comprehensive conceptualization of information proposed so far 
%\label{ref:RNDBpmiZTvtcr}(Burgin, 2003; 2010; 2017; Burgin and Feistel, 2017; Burgin and Mikkilineni, 2022),
\parencites[][]{burgin_information_2003}[][]{burgin_theory_2010}[][]{burgin_structural_2017}[][]{burgin_structural_2017}[][]{burgin_is_2022}, %
 and it includes Heller's metaphysical aspect of information in some form, thereby granting Heller's intuitions legitimacy of sorts.



In the GTI, information is stratified according to the global structure of the world, as represented by the Existential Triad, which comprises the world's top-level components as a~unified whole that reflects the unity of the world. This triadic structure is rooted in the long-standing traditions of Plato and Aristotle, and it comprises three components: the Physical (i.e., material) World, the Mental World, and the World of Structures 
%\label{ref:RND9v6oWGxvXU}(Burgin, 2010; 2017).
\parencites[][]{burgin_theory_2010}[][]{burgin_structural_2017}. %
 The Physical World represents the physical reality that is studied by natural and technological sciences, while the Mental World encompasses different forms and levels of mentality. Finally, the World of Structures comprises various kinds of ideal structures. The Existential Triad involves differentiating information into two fundamental classes: ontological information (i.e., information in nature) and mental information.



A~more detailed explanation of the GTI can be gained from Burgin's works, as cited above. Due to its metaphysical import, the GTI may not be to everyone's liking, but it does not make the theory itself any less comprehensive or wrong; philosophy is not a~beauty contest, even if it seems to be so from time to time. Further, the fact that the GTI is not known outside of the narrow circle of experts in the philosophy of information does not take away anything from its import; the veracity of scientific theories is not voted in or out by a~democratic process or won in a~popularity contest (a point that some people may miss). Moreover, in the authors view, we do not have anything better than the GTI theory, at least for now.



\section{Beyond Heller's Information}

Out of Heller's fragment (11) and the works of other cosmologists' (e.g., Wheeler, Reeves),\footnote{Not surprisingly, visions of information as a~fundamental element of nature did not originate from computer or data scientists or communication and networking engineers but rather people working intimately with information and nature.} which envision information as a~fundamental element of nature, grew the idea that information cannot be identical to, or identified with, the external form or shape of an object, structure as such, syntax, or even semantics because these ``things'' are temporal and ephemeral, whereas a~fundamental element of nature should have a~more stable existence.\footnote{The stability in time of physical objects, which is denoted as persistence, is the property of something to exist through time simpliciter. All physical things, including the Universe itself, persist in that they come into existence, exist for a~certain time (possibly changing forms on the way), and disappear (as in Heraclitian flux), at least this is the view of The Standard Model of Cosmology. (See the discussion about the SMC in, for example, the work of Smeenk and Ellis 
%\label{ref:RNDBoKzKWhArt}(2017),
\parencite*[][]{smeenk_philosophy_2017}, %
 Scott 
%\label{ref:RNDiLQuQPN39N}(2018),
\parencite*[][]{scott_standard_2018}, %
 and Page 
%\label{ref:RNDjvTHoAeCPd}(2020).
\parencite*[][]{page_little_2020}.%
) } These material forms (the external form or shape of an object ) should be better regarded as the medium through which information discloses itself to us, Heller's position, rather than information itself. To address this insight, Heller proposed that information is ``an abstract form'' or ``something beyond the form,'' which verges on the Platonic realm.



In the GTI, information is conceptualized as nature's potential to form complexes or low-entropy structures 
%\label{ref:RNDugKxZRF2Vz}(see Krzanowski, 2023)
\parencite[see][]{krzanowski_inquiry_2023}%
\footnote{See ft. 14.} or information as a~latent order in nature.\footnote{The term ``latent order'' should always be interpreted as the ``latent order or the potential of nature to create complex morphologies.'' Wheeler denotes this latent order, it seems, as a~principle of organization 
%\label{ref:RNDC5IqTaonUY}(Wheeler, 1989).
\parencite[][]{wheeler_information_1989}.%
} The concept of information as the potential of nature to create low-entropy (thermodynamic entropy) complexes (structures) appears to resemble the concept of Aristotelian potency, but the precise nature of this apparent similarity needs further research. Several recent studies have implied the existence in nature of the potentiality, which is also referred to as self-organization, to create forms or complexes 
%\label{ref:RNDI3yTakFRnC}(e.g., Eigen and Winkler, 1993).
\parencite[e.g][]{eigen_laws_1993}. %
 The self-organization property of nature is observable in everything from snowflake structures to organic life and the cosmos 
%\label{ref:RND0hZBPGH0UX}(e.g., Reeves, 1986; Schrodinger, 2012).
\parencites[e.g][]{reeves_heure_1986}[][]{schrodinger_what_2012}.%
\footnote{We regard snowflakes as low-entropy complexes that epitomize the persistence of natural objects or naturally organized complexes 
%\label{ref:RND25uEQGaUv3}(Reeves, 1986).
\parencite[][]{reeves_heure_1986}. %
 Forming complexes (i.e., ice crystals) that later disintegrate exemplifies nature's flux and the transition from low–high–low organizational states. Under specific conditions, nature forms low-entropy systems in local violation of the second law of entropy. Complex, highly organized natural systems are characterized by low entropy, while chaotic systems with simpler organization are high-entropy systems. This process for forming low-entropy systems can go on for as long as the required conditions are satisfied. For an extended discussion of low-entropy complexes and information, see the work of Krzanowski 
%\label{ref:RND8VjaSuEddP}(2023).
\parencite*[][]{krzanowski_inquiry_2023}.%
} Nevertheless, we should add that potentiality in its modern form does not attribute Aristotelian \textit{telos} to nature. Information as nature's potency or power is a~rather poorly explored topic and it should therefore be the subject of a~separate study. (See the discussion about nature's potencies in the work of Bird 
%\label{ref:RNDbMjZFqEieE}(2007)
\parencite*[][]{bird_natures_2007} %
 or Austin and Marmodoro 
%\label{ref:RNDOI7j9FRSFC}(2017).
\parencite*[][]{simpson_structural_2017}.%
)



\section{Conclusion}

Heller's intuitions about information in nature are not part of the mainstream information research, fortunately, otherwise we would have few reasons to talk about his work. Heller's intuitions belong to studies into the deep foundations of reality and border (for some) on metaphysics. It is certainly a~path less travelled, one reserved rather for a~minority of more open minds. With this comes the (sort of) penalty of not being frequently referred to, albeit with the delight of exploring the deep unknown. Then again, is this not where the real pleasures of science and philosophy reside?



Being a~hard–core scientist, Heller never abandoned the philosophical perspective 
%\label{ref:RNDJ6puDgrvvG}(called by himself ‘philosophy in science', see Heller, 2019; Polak, 2019; 2022)
\parencites[called by himself ‘philosophy in sciencesee][]{heller_how_2019}[][]{polak_philosophy_2019}[][]{polak_beyond_2022} %
 however at the cost of introducing metaphysical ambiguities. We may argue that Heller did not clarify his ideas about information and that some of his claims are enigmatic (e.g., ``laws of nature are information, or information is their expression only,'' ``structures code information or express information,'' ``in what sense are laws of nature natural structures,'' and ``information is expressed in or by ‘empty' mathematical structures or these structures are information''). This leaves the reader feeling somewhat uneasy. Yet the concepts Heller was grappling with are not well understood, and even now, nobody has proposed any better elucidations for them. At least with Heller, our ignorance and ambiguities about information and the foundations of reality have been explicated. Why we did not try to interpret Heller's ideas on information further? As we have said in the introduction, we try to report what Heller said, not what his claims might have implied.



The connection between Heller and the GTI, the most comprehensive formulation for the nature of information we have, adds some importance to Heller's perspectives (it shows that Heller's ideas on information fits well into a~larger comprehensive theory), but it also legitimizes the GTI itself. This is because Heller's perspective is built upon a~deep understanding of the foundation of nature and physics.\footnote{For more popular publications by Heller on the cosmos, science, and the foundations of the universe, see 
%\label{ref:RNDC1KC1z2LCO}(Heller, 2008a; 2008b; 2011; 2012; 2013a; 2013b; 2017; 2020).
\parencites[][]{heller_ostateczne_2008}[][]{heller_podgladanie_2008}[][]{heller_philosophy_2011}[][]{heller_matematyka_2012}[][]{heller_filozofia_2013}[][]{heller_logos_2013}[][]{heller_przestrzenie_2017}[][]{heller_jedna_2020}. %
 For the full list of Heller's 200+ scientific publications, see http://www.obi.opoka.org/heller/ or https://www.faraday.cam.ac.uk/about/people/prof-michal-heller/.} The GTI, meanwhile, is a~complex construct, and comprehensive as it is, it is the best we currently have, having been built by an exquisite philosopher and mathematician extraordinaire, not through a~deep study of nature, as was the case with Heller's ideas,\footnote{This point is important. Philosophy, mathematics, cosmology, and sciences in general all attempt to address fundamental questions using their own different methodologies, and often they diverge in their conclusions. Nevertheless, when their conclusions agree in some cases, it significantly strengthens the results of their inquiries.} but rather through the deep conceptual analysis.



The possible role of information in nature has been discussed in several studies. The researchers who have conceptualized information as something more fundamental in nature (like Heller proposed) rather than just an idea or knowledge over the past 50 years includes von Weizsäcker 
%\label{ref:RNDmw6MivsyS4}(1971),
\parencite*[][]{weizsacker_einheit_1971}, %
 Burgin 
%\label{ref:RND5f9144pqSu}(2003; 2010; 2017),
\parencites*[][]{burgin_information_2003}[][]{burgin_theory_2010}[][]{burgin_structural_2017}, %
 Burgin and Feistel 
%\label{ref:RNDe3JVKsbhZ3}(2017),
\parencite*[][]{burgin_structural_2017}, %
 Burgin and Mikkilineni 
%\label{ref:RNDWYgBzw5fYd}(2022),
\parencite*[][]{burgin_is_2022}, %
 Turek 
%\label{ref:RNDrCFSbyFfhv}(1978; 1981),
\parencites*[][]{turek_filozoficzne_1978}[][]{turek_rozwazania_1981}, %
 Collier 
%\label{ref:RND8u2CUeD5lm}(1990),
\parencite*[][]{hanson_intrinsic_1990}, %
 Reeves 
%\label{ref:RNDMmBf53y93k}(1986),
\parencite*[][]{reeves_heure_1986}, %
 Stonier 
%\label{ref:RNDXGTASYBeZs}(1990),
\parencite*[][]{stonier_information_1990}, %
 Devlin 
%\label{ref:RNDIGTEPE7yb3}(1991),
\parencite*[][]{devlin_logic_1991}, %
 De Mul 
%\label{ref:RNDWzR3xMHLdp}(1999),
\parencite*[][]{mul_informatization_1999}, %
 Polikghorne 
%\label{ref:RNDtyNxeC5qMA}(2000),
\parencite*[][]{Polkinghorne2000}, %
 von Baeyer 
%\label{ref:RNDFzGClcz9KO}(2005),
\parencite*[][]{baeyer_information_2005}, %
 Seife 
%\label{ref:RNDhNSj49azPO}(2006),
\parencite*[][]{seife_decoding_2006}, %
 Dodig-Crnkovic 
%\label{ref:RNDaeMUBdQOYa}(2012),
\parencite*[][]{dodig-crnkovic_alan_2012}, %
 Hidalgo 
%\label{ref:RNDjH4xR2nYQ0}(2015),
\parencite*[][]{hidalgo_why_2015}, %
 Wilczek 
%\label{ref:RNDKolD0LZuJV}(2015),
\parencite*[][]{wilczek_beautiful_2015}, %
 Carrol 
%\label{ref:RNDAvu70WunTm}(2017),
\parencite*[][]{carroll_big_2017}, %
 Rovelli 
%\label{ref:RND8YMGc1S60E}(2016),
\parencite*[][]{rovelli_reality_2016}, %
 Davies 
%\label{ref:RNDrf0LOWU1iF}(2019),
\parencite*[][]{davies_demon_2019}, %
 Sole and Elena 
%\label{ref:RNDFlA1ltHVlo}(2019),
\parencite*[][]{sole_viruses_2019}, %
 Schroeder 
%\label{ref:RNDx0C3aJJ6RT}(2005; 2017),
\parencites*[][]{schroeder_philosophical_2005}[][]{schroeder_structural_2017}, %
 Wheeler 
%\label{ref:RNDtMBizZtCna}(1989),
\parencite*[][]{wheeler_information_1989}, %
 Landauer 
%\label{ref:RNDpjEakRB0SK}(1961; 1991; 1996),
\parencites*[][]{landauer_irreversibility_1961}[][]{landauer_information_1991}[][]{landauer_physical_1996}, %
 and Krzanowski 
%\label{ref:RNDusb7eOdEEu}(2022).
\parencite*[][]{krzanowski_ontological_2022}. %
 This list is certainly not exhaustive, but it offers a~comprehensive overview of the recent (going back to the1960s) seminal discussions on this topic.



Heller's writings about information should be seen on a~par with the work of these authors, and should enter the canon of works on this topic, because his insights and intuitions not only confirm their studies but offer a~perspective about the role of information in nature that is grounded in cosmology and physics rather than just in conceptualizations and philosophy, as is often the case with works on information.



\end{artengenv}

