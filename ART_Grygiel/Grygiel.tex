\begin{artengenv}{Wojciech P. Grygiel}
	{The applicability of the concept of the field of rationality in the explanation of the fundamental role of symmetries in physics}
	{The applicability of the concept of the field of rationality\ldots}
	{The applicability of the concept of the field of rationality in the explanation of the fundamental role of symmetries in physics}
	{The Pontifical University of John Paul II in Krakow}
	{The introduction of the concept of the field of rationality and its correlates (the field of potentiality and the formal field) by Józef Życiński and Michał (Michael) Heller opened up space for the philosophical explanation of the unreasonable effectiveness of mathematics in capturing regularities built into the physical reality. The presented study is a~response to the clear incentive of these authors towards the development of the understanding and applicability of these concepts. It is argued that identifying symmetries within the field of rationality not only helps to articulate the fundamental role of symmetries in physics but it provides a~better grasp on the issue of potentialities for the emergence of complexity in the Universe. Also, some global properties of this field can be more deeply comprehended. By indicating the drawbacks and limitations of this approach, perspectives for further inquiry into the meaning and usefulness are suggested.
	}
	{symmetry, ontology, potentiality, emergence, field of rationality, field of potentiality.}




\section*{Introduction}

\lettrine[loversize=0.13,lines=2,lraise=-0.03,nindent=0em,findent=0.2pt]%
{T}{}he concept of the \textit{field of rationality} has been proposed independently by Józef Życiński and Michał (Michael) Heller in order to address two fundamental questions within the philosophical reflection on the nature and method of mathematics and physics: (1) how mathematical objects and structures exist and (2) why mathematics is so effective in the physical sciences.\footnote{An extensive overview of the origin and the development of the concept of the field of rationality can be found in: 
%\label{ref:RNDTasydVpCOn}(Pabjan, 2011; Grygiel, 2022).
\parencites[][]{pabjan_jozefa_2011}[][]{grygiel_critical_2022}. %
 } The development of the contemporary physics has revealed that the formalisms of the fundamental physical theories rely on symmetry manifested by the appropriate symmetry groups. Also, symmetry is a~principal tool by which the unification of physics has become possible thereby making the Universe intelligible at an unprecedented scale 
%\label{ref:RNDc3r1pKrunE}(e.g., Gross, 1996).
\parencite[e.g][]{gross_role_1996}. %
 This outcome has found its vocal expression in a~phrase coined by Wolfgang Pauli who referred to the ubiquity of symmetry in physics as \textit{Gruppenpest} (the plague of symmetry). The importance of deepened philosophical analysis of why the type of symmetries known as gauge symmetries is so effective in physics has been emphasized by Michael Redhead 
%\label{ref:RNDLaXUK12aTb}(2003, p.138)
\parencite*[][p.138]{brading_interpretation_2003} %
 in the following assertion: ``The gauge principle is generally regarded as the most fundamental cornerstone of modern theoretical physics. In my view its elucidation is the most pressing problem in current philosophy of physics''. The philosophical concerns regarding symmetries in physical theories continue to spark interest and discussions from a~wide range of perspectives 
%\label{ref:RND5F99cb5p0B}(e.g., Dardashti, Frisch and Valente, 2021).
\parencite[e.g][]{dardashti_editorial_2021}.%




The aim of this study is to show how the understanding of the internal structure and the global properties of the field of rationality can be deepened by taking into account that symmetries play such an extremely important role in physics. By identifying symmetries within the field of rationality a~metaphysical argument for this state of affairs will become available. The need for this deepening has been clearly expressed by Heller 
%\label{ref:RNDtoVj61hHvL}(2014, p.442)
\parencite*[][p.442]{heller_field_2014} %
 in his assertion that ``the idea never went beyond its seminal stage'' and still remains ``fuzzy''. The additional advantage of identifying symmetries within the field of rationality is that one can better explicate the nature of potentialities for the emergence of physical structures in the course of the Universe's history commencing at the moment of the Big Bang.



The objective of this study will be carried out in fours steps. Firstly, an introduction to the origins and the meaning of the field of rationality as well as its derivatives referred to as the formal field and the field of potentialities will be offered. A~special emphasis will be made on how Życiński attempted to capture the process of the emergence of the physical structures in the Universe as the actualization of potentialities latent in the field and what are the possible shortcomings of this attempt. Secondly, the specificity of the formalisms of the symmetry based physical theories will serve as a~premise to propose a~relation between the formal field and the field of rationality and to introduce the concept of the \textit{field of symmetries}. Thirdly, the formal field as well as the connection between symmetry and structure will be utilized in advancing better understanding of potentialities and the ensuing dynamics leading to the emergence of structures in the Universe. Some useful references to the contemporary discussions on potentialities will be made. Fourthly and lastly, keeping in mind that the inquiry is intended more as an exploration of the possible interpretative perspectives of the field formal field and the field of rationality, some suggestions concerning further investigative efforts will be offered. Since identifying symmetry groups within the field of rationality implies a~decidedly realist position in regards to the status of symmetries within the fundamental fabric of the Universe, this study explores a~new dimension of metaphysical issues that arise in the context of contemporary science.



\section*{The field}

The concept of the \textit{field of rationality} was originally proposed by Józef Życiński and introduced with detailed justification in 
%\label{ref:RNDasUdwPRpoz}(Życiński, 1987).
\parencite[][]{zycinski_filozoficzne_1987}. %
 In a~nutshell, this field comprises all possible mathematical structures as well as all possible relations of inference between them and some section of this field provides a~matrix for the physical functioning of the Universe. This clearly reflects the fact that only a~small portion of mathematics turns out to be relevant from the point of view of physical applications. As long as this field is considered from purely formal point of view only, Heller prefers to call it the \textit{formal field} and to link the field of rationality with the ontological claim positing it as an existing entity that justifies the possibility mathematics as the activity of the human mind 
%\label{ref:RND0VuHOaLI42}(Heller, 1997, p.238).
\parencite[][p.238]{heller_uchwycic_1997}. %
 This, of course, reveals Platonic preferences of Heller and Życiński to which they openly subscribe 
%\label{ref:RNDtslsXM3aZc}(e.g., Życiński, 2013).
\parencite[e.g][]{zycinski_swiat_2013}. %
 Unfortunately, both these authors remain somewhat ambiguous whether the field of rationality should refer to the world of mathematics as a~whole or to its portion that is physically relevant only. Since it is the ontological interpretation of the field of mathematical structures that shows the desired explanative power in regards to the possibility of mathematics and its applicability in physics, for the purpose of the conceptual clarity the Platonic world of all possible mathematical structures will be referred to as the \textit{formal field} and its physically applicable portion as the \textit{field of rationality}. The distinct ontological character of these two fields finds its natural environment in the Platonic ontology of the three worlds of \textit{math}, \textit{physics} and \textit{mind} proposed by British mathematician and theoretical physicist, Roger Penrose 
%\label{ref:RNDas9enwZcu6}(e.g., 2004, pp.17–21).
\parencite[e.g][pp.17–21]{penrose_road_2004}. %
 In this ontology, the physical world emerges in its entirety from the objectively existing Platonic world of mathematical structures. Undoubtedly, the Platonic interpretation of the formal field of mathematical structures reinforces a~strong metaphysical claim but, at the same time, it does justice to the preferred standpoint of mathematicians treating the object of their study as a~objectively existing reality which they do not construct but discover 
%\label{ref:RNDaudESttG59}(e.g., Penrose, 2004, p.13).
\parencite[e.g][p.13]{penrose_road_2004}.%




While the above paragraph shows only a~general statement of what the field of rationality is, Życiński took up the challenge to delve deeper into its nature. In his view, the key role of the field of rationality is to capture the fact that ``the fundamental level of reality is constituted by an abstract network of formal relations and the reality of the observed physical substrate is secondary with respect to the formal relations whose existence we discover in the physical processes which are concrete exemplifications of these structures'' 
%\label{ref:RNDJHh6rXyfWf}(Życiński, 1995, p.102)
\parencite[][p.102]{zycinski_status_1995}%
\footnote{Translated from Polish by Wojciech P. Grygiel.}. In order to provide a~suitable illustration of this assertion, Życiński resorted to quantum field theory and, in particular, to the metaphor based on the process of formation of particles as a~result of the excitation of the lowest energy field, that is, the vacuum. Following the suggestion of American particle physicist, Heinz Pagels 
%\label{ref:RNDe0mjGIoHGN}(1983, p.245),
\parencite*[][p.245]{pagels_cosmic_1983}, %
 Życiński treated the vacuum as a~reservoir of potentialities out of which physical structures could emerge in the evolution of the Universe and, ultimately, find their exemplification in concrete physical systems. And this is the very reason why he proposed to regard the field of rationality as the \textit{field of potentiality}.



His favorite examples of the emergent structures were the Kepler laws of the planetary motions and the Mendeleev's periodic table which---in his opinion---should have both already existed in the early Universe prior to the appearance of planets and chemical elements. He maintains that although these laws must have been somehow coded in the structure of the Universe so their actualization in concrete objects occurred strictly by natural powers, there must remain a~``radical separation'' between these two domains of existence 
%\label{ref:RNDVxi7af1B4I}(Życiński, 2006, pp.53–54).
\parencite[][pp.53–54]{zycinski_pole_2006}. %
 In other words, on one hand he wished to secure the workings of the physical causality in effecting this actualization and yet to preserve some form of otherness of the field of rationality to sensibly articulate the idea of potentiality.



It is not difficult to see that this ambiguity makes Życiński's argumentation inconclusive and that he never came up with a~satisfactory way out of it. Initially, he opted for the Platonic metaphysical view of the field of rationality based on the atemporal character of the abstract structures comprising the field of rationality. While this dualist stance allowed for a~clearer articulation of their potentiality with respect to the domain of physicality, it effectively prevented their causal activity in this domain. Życiński 
%\label{ref:RNDwYpzXVPnw5}(2006, pp.58–59)
\parencite*[][pp.58–59]{zycinski_pole_2006} %
 has eventually abandoned the Platonic view of the field of rationality in favor of its ontological interpretation by naming the field of rationality the \textit{nomic structure} of the Universe (from Greek \textit{nomos} = law) which reflects much closer relationship of this field with the laws that govern the Universe. In his introduction to Życiński's \textit{Świat matematyki i~jej materialnych cieni} Heller parallels this conceptual change with the transformation of the philosophical school of Plato in which the ostensibly dualist metaphysics has been converted into ontology by Plato's successors: Speusipius and Xenoctares 
%\label{ref:RNDwiWKS4P6fW}(Heller, 2013)
\parencite[][]{zycinski_wstep_2013}%
\footnote{For an in-depth analysis of the transformation of the Platonic School see: 
%\label{ref:RNDPCM02Qqsyq}(Dembiński, 2010; see also 2015; 2019).
\parencites[][]{dembinski_pozny_2010}[see also 201][]{dembinski_2015}[][]{Dembinski_2019}.%
}. Heller opines that this is precisely where the final ontological stance of Życiński qualifies and where the idea of the mathematicity of the Universe has its roots.



Życiński's ontological turn finds its corroboration in the approach to quantum gravity pursued by Heller and his collaborators with the use of the non-commutative geometries 
%\label{ref:RNDmvaE0YkT1G}(Heller and Sasin, 1998; Heller, 2002, pp.115–122).
\parencites[][]{heller_emergence_1998}[][pp.115–122]{heller_poczatek_2002}. %
 This approach leads to the elimination of the notion of space and time on the fundamental level of the physical reality thereby offsetting the dichotomy between the atemporal and the temporal as means of delineating what is abstract and ideal and what is concrete. Consequently, atemporality ceases to be the attribute of the abstract Platonic world but shifts over to the domain of the physical and can enter into the causal interactions with the concrete. Contrary to the Platonic stance, this situation neutralizes the barrier for the physical causation in actualizing potentialities but, by this very fact, it makes the articulation of potentiality more difficult.



By bringing up only a~handful of examples illustrating the usefulness of the concept of the field of rationality Życiński \textit{de facto} provides only some local characteristics of this field without much attention its more fundamental global properties. However, intimations of this kind of description appear in his insistence that the field of rationality as a~whole imposes constraints on the ontology of the Universe rendering some phenomena and processes impossible 
%\label{ref:RNDB9JNXg3FWw}(Życiński, 1987, p.180).
\parencite[][p.180]{zycinski_filozoficzne_1987}. %
 According to Życiński, the existence of the field as a~constraint manifests itself in the unchangeability of the physical constants, stability of the physical processes and---most importantly---symmetries and their invariants. In order to substantiate this claim he recalls Pagels' observation that the majority of the history of modern physics are the discoveries of new symmetries 
%\label{ref:RNDMcoybGvbJ1}(Pagels, 1983, p.296).
\parencite[][p.296]{pagels_cosmic_1983}. %
 Engaging the field of rationality to explain the role of symmetries as the cornerstone of contemporary physics accords with Życiński's philosophical intuitions and his endorsement of this line of argumentation can be taken for granted.



It turns out that Życiński's incentive to investigate the global properties of the field of rationality echoed in a~study carried out by Heller in which he does not commence from the field's physical concretizations but he reaches out to the nature of mathematics itself by turning to a~highly abstract mathematical theory known as the \textit{category theory} 
%\label{ref:RNDtcWhxwm5uG}(Heller, 2014).
\parencite[][]{heller_field_2014}. %
 The category theory perceives the different branches of mathematics like calculus or linear algebra as separate categories whereby it provides an overview ``from above'' and reveals possible connections among them. Since a~separate category may be selected to represent a~section of the field of rationality that constitutes a~matrix for the functioning of a~given region of the physical reality, the field of rationality can be matched with the \textit{field of categories}. Heller's assertion that the question ``why is the Universe mathematical'' should be rephrased into ``why is the Universe categorical'' suggests that the field of rationality is rather meant to indicate the collection of physically relevant mathematical structures only. Although there are studies which indicate deep connection between symmetry and categories 
%\label{ref:RNDhirb5PGCKx}(e.g., Heunen, Landsman and Spitters, 2008),
\parencite[e.g][]{heunen_principle_2008}, %
 the approach taken up in this study will be \textit{aposterioric} in the sense that it will attempt to infer more on the global nature of the field of rationality from the well established fact of the ubiquity of symmetry in physical theories.



\section*{Symmetries in the Field}

The first indication that there may exist connections between the field of rationality and symmetry can be found in the philosophical understanding of the term \textit{rationality}. The term itself has diverse meanings deriving from the Latin term \textit{ratio} and it may stand for reason, relation as well mathematical proportion. This coincides with the original understanding of symmetry developed in the ancient Greece which reflects the etymology of the term as the \textit{common measure} and which precedes the group theoretical account of symmetry. As emphasized by Brading and Castellani, symmetry remains closely linked with unity which in the ancient meaning is effected by proportion and in the modern by the symmetry operations belonging to a~precisely defined transformation group. They assert that ``the way which this unity is realized on one hand, and how the equal and different elements are chosen on the other, determines the resulting symmetry and in what exactly it consists'' 
%\label{ref:RND18USI560eY}(Brading and Castellani, 2003, p.3).
\parencite[][p.3]{brading_introduction_2003}. %
 This, in turn, correlates with the \textit{normative} character of symmetry, namely, that the invariance with respect to a~group of transformations imparts significant restrictions on the theory's form as well as on the form of its equations 
%\label{ref:RNDiy3EYNiFP1}(Brading and Castellani, 2003, p.13).
\parencite[][p.13]{brading_introduction_2003}.%




The next important piece of information on how to locate symmetries in the field of rationality comes from Heller's attempt to compare the process of the formation of a~physically meaningful representation of an abstract group with the commencement of its existence in the philosophical sense of the term. He grounds this inference in the analogy to St. Anselm's proof of the existence of God on the premise that there occurs a~transition from the formal order to the order of real physical existence 
%\label{ref:RNDeqpa82wk2Z}(Heller, 2003, p.63).
\parencite[][p.63]{heller_teilhards_2003}. %
 As an illustration Heller offers the example of the irreducible unitary representations of the Poincaré group which describe properties of all existing elementary particles and fields. Considered in themselves, groups are but sets of abstract objects defined by the group operation satisfying the group axioms. The Lie groups, which are continuous groups playing key role in physical applications and to which the Poincaré group belongs, are additionally equipped with differentiable manifolds 
%\label{ref:RNDpMeBflyFCe}(e.g., Schwichtenberg, 2018, pp.47–54).
\parencite[e.g][pp.47–54]{schwichtenberg_physics_2018}. %
 However, these abstract objects begin to ``do physics'' once they are represented as group structure preserving operations on a~uniquely selected mathematical space most frequently considered as linear transformations of a~vector space. By using representation theory, one can study how a~given group operates on a~variety of vector spaces thereby generating distinct meaningful physical situations.



The simplest and quite illustrative examples in that regard are the SU(2) and SU(1,1) symmetries. Since both these symmetries offer powerful tools in advancing our understanding of the properties of quantum systems, they are undoubtedly important elements of the field of rationality. While there is only one unitary and finite dimensional representation of the SU(2) compact group, the SU(1,1) group, which is probably the simplest non-compact Lie group, has several unitary irreducible representations which refer to different families of coherent states and serve to study physically distinct systems 
%\label{ref:RNDy3tze9voO3}(e.g., Vourdas, 2006).
\parencite[e.g][]{vourdas_analytic_2006}. %
 Since the abstract structure of the SU(1,1) group leads to several distinct physical realizations, it seems rational to locate the abstract groups within the formal field while their physically pertinent representations, which are symmetries, should find their place in the field of rationality. Consequently, considering that the abstract groups may have representations that are not physical (e.g., non-unitary representations), one can postulate the existence of the \textit{field of symmetries} that constitutes the subfield of the formal field which contains all possible abstract groups and symmetries regardless of their physical relevance.



In order gain further insight into the relations between the formal field, the field of symmetries and the field of rationality, one needs to take into account three general features of physical theories that rely on symmetries. Firstly, the formalisms of these theories feature mathematical structures other than symmetry groups such as topology, manifolds or differential geometry. Secondly, physical theories contain symmetries that are physically irrelevant such as the symplectic structure of a~Hamiltonian, for instance. This state of affairs may have its source in the fact that physical theories put forward by physicists are but approximations of the structure of the physical reality and as such they may contain structural elements that do not pertain to reality but they are artifacts of the workings of the human mind. A~good example in this regard is given by the four possible formulations of quantum mechanics that are empirically but not mathematically equivalent: Hilbert spaces, Feynman path integrals, C*-algebras and density matrices. As Heller points out, these formulations are different representations of the quantum reality taken in an informal sense that they encode some of the structural features of this reality and only structural invariants of these representations refer to the fabric of the microworld (Heller 2011, pp.144-145). Whatever remains variant is relegated to the domain of the artifact of description. Interestingly enough, such inference is oftentimes given as the defining feature of symmetry whereby symmetries constitute mathematical tools which discriminate between what pertains to reality and what is a~surplus structure, that is, an artifact of a~theory. Paul Dirac 
%\label{ref:RNDn8o6PvCeHR}(1930, p.vii)
\parencite*[][p.vii]{dirac_principles_1930} %
 asserts the following:



\begin{quote}
[Nature's] fundamental laws control a~substratum of which we cannot form a~mental picture without introducing irrelevancies. The formulation of these laws requires the use of the mathematics of transformations.
\end{quote}



The third general feature of a~physical theory with symmetries has to do with the fact that although symmetries provide important constraints for the dynamical equations, they don't determine them uniquely and other factors need to be taken into account in their derivation. For instance, neutrino oscillation is a~phenomenon where an impact of symmetry on dynamic properties (equations of motion) becomes particularly visible. The three-flavor neutrino oscillation can be effectively described as a~3-level system of a~dynamics generated by a~highly non-trivial Hamiltonian directly related to Pontecorvo–Maki–Nakagawa–Sakata mixing matrix relating mass and flavor states 
%\label{ref:RNDg2Px9jkBNv}(e.g., Banerjee et al., 2015; Bilenky, 2016).
\parencites[e.g][]{banerjee_quantum-information_2015}[][]{bilenky_neutrino_2016}. %
 A~form of this matrix depends on the CP symmetry constraining neutrino properties. If the CP symmetry is violated---as it seems to be the case according to the recent experiments 
%\label{ref:RNDoEyGAHb5Ka}(e.g., The T2K Collaboration, 2020)
\parencite[e.g][]{the_t2k_collaboration_constraint_2020}%
---neutrino and its antiparticle become distinguishable and evolve in time with different Hamiltonians generating their evolution. One can identify measurable properties of the neutrino by indicating particular form of the time evolution and its symmetry 
%\label{ref:RND0J2TVDLx4v}(e.g., Richter, Dziewit and Dajka, 2017).
\parencite[e.g][]{richter_leggett-garg_2017}.%




Although by taking into account the ubiquity of symmetries in physics one can be initially tempted to match the field of rationality with the field of symmetries, considerations presented above show that the situation is more complex and a~more nuanced approach needs to be adopted. It has been already suggested that the abstract groups and symmetries belong to the formal field and that this field contains all possible mathematical structures. It turns out naming the field of rationality ``the field'' has yet another advantage because the precise mathematical definition of a~field associates a~certain quantity with each of its points. By way of analogy, a~particular instance of rationality such as those indicated by Życiński can be linked with a~corresponding point of the field. Such a~point stands for a~section of the fundamental ontic structure of the Universe represented by physical theories. Taking into account the ubiquity of symmetries in physics a~conjecture can be put forward that a~symmetry group is located in the neighborhood of the points of the field of rationality and it may constrain structures proper to a~given point. As a~result, a~symmetry group will turn up in the physical theory that describes reality's structure at this point and it will exert influence on the properties of the systems subject to the regime of this theory and its equations. Ultimately, physically relevant symmetries present in the field of rationality seem to form a~non trivial cross section of the field of symmetries, that is a~part of the formal field, with the field of rationality. Unfortunately, at this stage of analysis it is not possible to explain why this cross section contains the symmetries that it does and not any other. One may also legitimately doubt whether, beyond a~mere statement, such an explanation is even possible.



\section*{Exploring potentiality}

A~close corollary of identifying symmetries within the field of rationality is the possibility of clarifying Życiński's ambiguity in regards to the nature of potentialities latent in the field of rationality. It turns out that one can think of these potentialities in two different ways based on how the ``radical separation'' between the abstract and the concrete comes about. The first way arises in some accordance with Życiński's original metaphysical outlook where the field of rationality containing abstract structures was placed in the Platonic world of ideas thereby generating the much desired ``radical separation'' between the abstract and the concrete. It is not hard too see that the proposed placing of the abstract groups such as SU(2) and SU(1,1) in the formal field and not in the field of rationality does justice to this radical separation when the formal field corresponds to the Platonic universe of mathematics. With the obvious reservation of how such abstract groups can exert their causal influence in the physical domain, this separation has to serve as the only reason for now why these groups should be regarded as potencies that become actualized in the form of the properties of fields and particles when unitarily represented in concrete linear spaces.



Keeping in mind that symmetries impose restrictions on the properties of the physical objects they describe, it is worthwhile to point an important difference between the two abstract groups. In contradistinction to SU(2), the SU(1,1) has several physically meaningful representations suggesting that its abstract structure is refracted in a~number concrete physical realizations whereby Życiński's demand of one abstract structure underpinning a~number of concretes is fulfilled. In a~way, the number of physically relevant representations could become a~measure of how potent a~given abstract group is in giving rise to real physical systems. Also, this kind of potency accounts for the physical character of the unbroken symmetries.



The second way of associating potentiality with symmetries has to do with the processes of symmetry breaking. Let us start with the difference between symmetry and design. The opinion that symmetry is a~key element of the design of the Universe has been expressed by American physicist Anthony Zee 
%\label{ref:RNDZbyP4liQAQ}(2007, pp.3, 283).
\parencite*[][pp.3]{zee_fearful_2007}. %
 It has been critically analyzed by American philosopher of science, Peter Kosso, who suggested an intuitive origin of this assertion based on the geometric symmetries of the geometrical objects. In his effort to dismantle this intuition, Kosso 
%\label{ref:RNDVdTQzjaIDm}(2003, p.421)
\parencite*[][p.421]{brading_symmetry_2003} %
 gave a~simple but telling example of juxtaposing a~messy and ordered room. While in a~messy room one can quite easily shift items around without upsetting its invariant structure and frustrating its owner, an ordered room does not admit of practically any displacements of its furnishings that would escape the attention of the one who arranged them. Kosso concluded that the messy room has more symmetry and less design while the ordered less symmetry and more design. Consequently, design means not symmetry but the breaking of symmetry suggesting that producing a~design connotes rather having intentional control over the choice of the desired symmetries than being subjected to a~constraint. As a~confirmation of his conclusions Kosso recalls Steven Weinberg's example of a~chair constructed out of atoms where each atom is rotationally symmetric but the chair itself is not. In other words, the building of a~chair by its designer has led to the decrease of symmetry.



As Debs and Redhead 
%\label{ref:RNDJK6XBF4i3x}(2007, pp.37–39)
\parencite*[][pp.37–39]{debs_objectivity_2007} %
 point out in a~rather informal and intuitive way, symmetry and invariance are complementary ideas bound by the relation of \textit{duality}. In mathematics duality is known to be a~broad concept and its precise definition is given when duality is applied to specific cases, for just that context. The main idea contained in duality is that it points to a~deeper structure that manifests itself in twofold manner as ``two sides of the same coin''. Debs and Redhead do not pursue any rigorous identification of an underlying structure that symmetry and invariance may represent but they wish to articulate the \textit{interchangeability} of these concepts with special emphasis on their \textit{reciprocality}. In particular, they refer to the fact that the higher the symmetry group of a~structure, the more changes it can endure indicating that it is less constrained because it contains less invariants. So if the symmetry group gets smaller, the number of invariants grows and the structure becomes richer (more rigid). In other words, the decrease of the size of the symmetry group, that is, the symmetry breaking, leads to the emergence of more complex structures resulting in the growth of complexity. Manchak and Barrett 
%\label{ref:RNDhWjIkHbTxQ}(2023)
\parencite*[][]{manchak_hierarchy_2023} %
 demonstrate that this relation bears more nuanced character but its informal treatment should suffice for the purpose of this study.



A~good example of the relationship between the operations of symmetry and the invariant structures are the different geometries with the Euclidean being the most rigid that is having the greatest number of invariants and the smallest symmetry group, through affine geometry where the requirement of constant length is loosened and only the parallel lines are preserved. Yet less structure comes with the projective geometry. The ``softest'' structure is topology whose invariant is the Euler number and any transformation is allowed that preserves continuity, that is, the structure of the neighborhoods of points. Ripping the structure apart would mean changing topology and breaking the structure's symmetry.



It is commonly known that the structuring and diversification of the physical reality occurs by means of the processes of symmetry breaking. Peter W. Anderson 
%\label{ref:RNDENP0PCKgPT}(1972, p.395)
\parencite*[][p.395]{anderson_more_1972} %
 offers an example the formation of a~crystal which leads to the lowering of the symmetry: ``the general rule, however, even in the case of a~crystal, is that the large system is less symmetrical than the underlying structure would suggest: symmetrical as it is, a~symmetrical crystal is less symmetrical than perfect homogeneity''. The nature of symmetry breaking has received an extensive treatment in physics leading to the identification of two basic mechanisms through which symmetry might be broken: \textit{explicit} and \textit{spontaneous} 
%\label{ref:RNDLhZNZoiTWM}(e.g., Castellani, 2003).
\parencite[e.g][]{brading_meaning_2003}. %
 The mechanism of the spontaneous symmetry breaking occurs when the lowest energy symmetrical solution becomes unstable under small perturbations as some parameter approaches a~critical value resulting in a~new asymmetric but stable lowest energy state. Inasmuch as Życiński's illustration of the actualization of the potentialities in the field of rationality by means of the excitation of a~vacuum could with some reservations reflect the mechanism of the spontaneous symmetry breaking (e.g., the excitation of the quantum harmonic oscillator), phase transitions yield a~much better example in this regard. A~system that is capable of undergoing a~phase transition could be regarded as having potentialities at its disposal to assume a~more ordered state due to symmetry breaking as a~certain external parameter is changed (i.e., decrease of temperature).



The presence of the groups of symmetries in the field of rationality allows for a~rather straightforward understanding of what it means that a~physical structure is contained in this field. Since following the explanation provided in a~previous section symmetries relate to the corresponding invariant structures via the relation of duality, a~concrete structure may be considered as encoded within the field of rationality by means of an appropriate subgroup of a~symmetry group that has been spontaneously broken. From a~more formal point of view, duality stands for a~mathematically precise relation between these two different structures suggesting that Życiński's postulate of the ``radical separation'' between the abstract and concrete finds its expression in this reciprocality. In summary, the actualization of a~physical structure that emerges from the field of rationality could be then understood as a~process of the lowering of a~symmetry present in this field where the original larger symmetry group connotes the potentiality to bring forth a~diversity of concrete structures which commence their physical existence as accessible for the scrutiny of the scientific method.



Also, the identification of symmetries in the field of rationality seems to offer ways of better insight into Życiński's claim that concrete physical systems are instantiations of the general physical laws that govern their dynamics (e.g., the Kepler laws). When symmetry is spontaneously broken, the solutions of the equations of motion are no longer invariant under the action of the equation's symmetries. Phrased differently, the world around us appears to us very asymmetric but it does not mean that the fundamental laws are not symmetric. Although the new lowest energy solutions are asymmetric, they are related through the action of symmetry transformations and the whole set maintains the symmetry of of a~given theory and its laws. Thus the lower symmetry solutions do not violate the symmetry properties of these laws. And conversely, the patterns exhibited by the behavior of nature provide clues to the symmetries that are being broken. The extent to which this mechanism is applicable to such instances as the planetary systems fulfilling the Kepler laws of motion would need much more detailed analysis that remains beyond the confines of this study.



The identification of symmetries within the field of potentialities finds its additional justification in a~path that is in some sense reverse to that of symmetry breaking, namely, a~path that hypothetically leads back to a~structure that has the potency of producing every possible complexity in the Universe. In addressing this issue Heller 
%\label{ref:RNDDwatscuKRd}(1997, p.232)
\parencite*[][p.232]{heller_uchwycic_1997} %
 asserts the following:



\begin{quote}
Everything points to the fact that at the beginning there was supersymmetry---an extremely rich and geometrically simple mathematical structure. The subsequent symmetry breakings (the separation of each of the four interactions) gave rise to increasing diversity. The dream of the theory of everything is the dream of discovering of the mathematical structure from which everything has its origin.\footnote{Translated from Polish by Wojciech P. Grygiel.}
\end{quote}



An attentive reader will quickly notice that in this quote Heller points to the reciprocal relation between symmetry and invariance as applied to the early stages of the Universe. In order to unify bosons and fermions, supersymmetry requires a~sufficiently large symmetry group which should in turn yield relatively few invariants thereby making the corresponding geometry simple. This observation signals an interesting connection between unification and potentiality in light of which a~unified theory would encode potentialities towards a~larger number of possible concretizations. For instance, such increased potentiality could manifest itself in a~theory unifying gravity with the three other interactions because, as Heller 
%\label{ref:RNDRbv1mOwLkq}(2002, p.63)
\parencite*[][p.63]{heller_poczatek_2002} %
 admits: ``it is very difficult to find a~symmetry rich enough to combine the spatiotemporal symmetry of gravitation with the dynamic symmetry of other interactions''.



It turns out the the issue of potentiality is one of the central ones in contemporary metaphysics and it concerns the ongoing discussion on the nature of powers and dispositions and these concepts are used into the explanation of what the laws of nature are 
%\label{ref:RNDCofJCKyIFw}(e.g., Friend and Kimpton-Nye, 2023).
\parencite[e.g][]{friend_dispositions_2023}. %
 In most general terms, to attribute a~disposition to a~thing means that if certain conditions are fulfilled, then that thing will behave in a~certain way, or produce a~certain effect---that is, that a~certain outcome will occur. For instance, a~negatively charged particle is an entity that, if brought together with another negatively charged particle, it will experience a~repulsive force. As French 
%\label{ref:RNDqDUG4rC1oG}(2020)
\parencite*[][]{french_doing_2020} %
 clearly shows, while the articulation of dispositions and powers in regards to objects of everyday experience is a~fairly straightforward task, the shift to the domain of the abstract mathematical formalisms of the symmetry based physical theories presents a~considerable challenge. In this regard one can legitimately ask what is the metaphysical significance of the fact that, for example, the spinor representation of the Poincaré group encodes the properties of electrons and quarks. Chances are that the application of the concepts of the formal field and the field of rationality may turn out instrumental in sorting out these difficulties. In order to accomplish that, however, a~separate detailed study will need to follow.



\section*{Conclusions}

In the conclusion of the presented inquiry it is worth to bring out that the identification of symmetries within the field of rationality---much the same as the postulate of the field itself---are philosophical interpretations. This means that they cannot influence the progress of physics but they provide answers to why this progress is possible. In other words, they do not modify or oppose the formalisms of the physical theories but they address questions which cannot be posed within their mathematical frameworks. Nevertheless, it is crucial to recall that the efficacy of the proposed interpretation relies on an \textit{a~posteriori} observation derived from the practical aspects of theoretical physics, revealing that symmetry serves as a~fundamental underpinning in all physical theories. The major contribution of the inquiry consists in that, by relying on this observation, a~novel insight into the global structure of the formal field and the field of rationality has been obtained. Moreover, the identification of symmetries within these domains fortifies a~robust realist standpoint concerning their ontological status, thereby opening up avenues for exploring their metaphysical significance. What might escape even the most sophisticated metaphysical consideration is why the cross section of the field of symmetries with the field of rationality contains these and not other symmetries that are physically relevant.



The identification of symmetries within the field of rationality and its suggested justification carry a~number of shortcomings and are in need of further development to address their full philosophical import. For instance, no reference was made to the different kinds of symmetries that enter into the theoretical frameworks (external (i.e., spatio-temporal), internal, gauge). Moreover, in light of the works of Heller on the application of the non-commutative geometry in the pursuit of the theory of quantum gravity that has been already mentioned of this study, some promising results can be obtained when the concept of a~group is generalized with that of a~\textit{grupoid} 
%\label{ref:RNDoXePTONV7o}(e.g., Heller, 2006).
\parencite[e.g][]{heller_evolution_2006}. %
 This indicates that identifying symmetry groups within the field of rationality may bear an approximate character only.



One can rightly expect that the development of Heller's idea of interpreting the field of rationality as the field of categories will provide further support for the meaningfulness of the field of rationality but at some point it might face its conceptual limitations as well. What appears promising from the point of view of this study is that some deep connections have been identified between categories and symmetry suggesting that that the field of categories may relate to the field of symmetries in a~yet unknown way 
%\label{ref:RNDvHaQvjXKm3}(e.g., Heunen, Landsman and Spitters, 2008).
\parencite[e.g][]{heunen_principle_2008}. %
 Consequently, the process of ``unfuzzying'' of the field of rationality remains a~challenge as one needs to constantly re-represent it with the use of more abstract conceptual frameworks allowing for the gradual unveiling of its nature. Ultimately, however, one cannot exclude that the intuitively understood duality expressing the relation of reciprocity between symmetry and invariance will reveal its full mathematical meaning suggesting that they are but two sides of the same coin and that the field of rationality is but another means by which the human mind strives to decipher the mystery of the Universe.


\enlargethispage{1.5\baselineskip}
\paragraph{Acknowledgments.}
The author wishes to express his thanks to the reviewers for their incisive comments and to Professor Jerzy Dajka from The Department of Physics of the University of Silesia in Poland for his help in sorting out complex formal technicalities.



\end{artengenv}
