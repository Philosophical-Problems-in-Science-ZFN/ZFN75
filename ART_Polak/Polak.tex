\setcounter{secnumdepth}{0}





\begin{document}

Philosophy in Technology: Objectives, Questions, Methods, and Issues





Paweł Polak (UPJPII)



\section{Abstract}

Philosophy in technology is a~research program that studies the philosophical roots of engineering and technology. By virtue of their education, technologists believe that the limits, goals, possibilities, and effects of technology on society and humankind are exclusively technological problems, so their solutions must lie exclusively in technology. In contrast, philosophy in technology asserts that the resolutions to these problems need to be rooted in an understanding of their philosophical origins. In this program paper, we define the objectives of philosophy in technology together with the kinds of questions it explores, the methods it uses, and its differences to the philosophy of technology.



Keywords: philosophy in technology, philosophy of technology, engineering perspective, semantic gap, philosophy in science, theology



The man who has no tincture of philosophy goes through life imprisoned in the prejudices derived from common sense, from the habitual beliefs of his age or his nation, and from convictions which have grown up in his mind without the co-operation or consent of his deliberate reason. To such a~man the world tends to become definite, finite, obvious; common objects rouse no questions, and unfamiliar possibilities are contemptuously rejected\label{ref:RNDJ0Uq3xFY8B}\textup{(Russell, 1912, p.243)}\textup{.}



[Philosophy] removes the somewhat arrogant dogmatism of those who have never travelled into the region of liberating doubt \label{ref:RNDiWmEalmrpO}\textup{(Russell, 1912, pp.243–244)}.



\section{Introduction: The Need for a~New Approach for Reflecting on Technology}

\footnotetext{ This paper is based on the paper co-authored with Roman Krzanowski presented at the conference ``Philosophy in Technology 2.0'' (Wroclaw University of Technology \& Polish Academy of Arts and Sciences). This text is an extended and modified version of my part of the joint publication. I~would like to thank Roman Krzanowski for the discussions, inspiration, and contributions to the joint publication. Of course, all errors and mistakes in this text are my own.}

The modern world bears the stamp of the science and technology that has shaped culture and given it an extraordinarily dynamic development. This trend is so deep and persistent that people uses the modern products of technology to express and promote themselves, with some even spinning the most extreme anti-rationalist, anti-developmental ideas. A~deeper philosophical reflection is therefore needed for the technology that forms the fabric of modern culture and determines our future models of life. We believe that the philosophy of technology has raised many important questions to date, but it has almost completely ignored the specific role that philosophy plays in the development of technology. To fill this void, we here propose a~program that we call ``philosophy in technology.'' We picked this name because we want to pay greater attention to philosophy that is ``internal'' to technology. Technology sometimes benefits directly from philosophical concepts, but the roles played by philosophy are more diverse, with them ranging from fundamental ideas and assumptions to the philosophical roles of technology itself. 
%\label{ref:RND4DM4Dc1Gp4}(For more on the metaphysical roles of technology see, for example, Bolter, 1984.)
\parencite[for exampl][]{}%




The following section begins by discussing the roots of ``philosophy in technology'' based on the idea of adapting the existing methodology of ``philosophy in science.'' Next, we contrast philosophy in technology with the philosophy of technology. In Section 5, we move on to discussing the main tenets of philosophy in technology as a~research program before we outline the methodological assumptions of philosophy in technology in Section 6. Next, Section 7 presents how a~philosophy in technology agenda may be useful for technology–theology relations. Section 8 then finally summarizes our observations about philosophy in technology and suggests a~need for an open dialogue between philosophers and technologists, even though they are not as far apart as many seem to think.



This text is programmatic for developing a~philosophical inquiry in such an important contemporary direction. As such, many topics are treated only sketchily, and the analyses are far from complete. This work aims to point out a~new direction for research, and subsequent works should fill in the identified gaps.



\section{Historical Background: The Shift from Science to Technology}

Contemporary technology is so closely related to science that we even use the term technoscience to reflect the deep interdependence between science and modern technology 
%\label{ref:RNDm32eyhF9Mi}(Hottois, 2023).
\parencite[][]{hottois_technoscience_2023}.%
\footnote{It is worth to mention that the relationship between science and technology has been strengthening since the emergence of engineering (polytechnic) sciences in the 18th century 
%\label{ref:RNDzbOaiPepkv}(Rodzeń, 2019, p.669).
\parencite[][p.669]{janeczek_nauka_2019}.%
} We focus here mainly on technology because the philosophy of science is at a~much more developed level, so we need to pay more attention to technology. Due to the strong connections between science and technology, we believe that we could benefit from some philosophical considerations about science, namely the metaphilosophical concept of ``philosophy in science.''\footnote{Keeping in mind the important differences between science and technology. For a~good synthetic account of the relationship between technology and science, see, for example, 
%\label{ref:RNDOSWOhNI3qP}(Franssen, Lokhorst and van de Poel, 2023).
\parencite[][]{franssen_philosophy_2023}.%
}



Now, what does the concept of ``philosophy in technology'' have in common with Michel Heller's well-known metaphilosophical concept of ``philosophy in science'' 
%\label{ref:RNDL3K3SwgmBf}(Heller, 2019; see also Polak, 2019)?
\parencites[][]{heller_how_2019}[see also][]{polak_philosophy_2019}? %
 Is it just a~play on words, or is it a~deeper result of the development of the philosophical school known as the Krakow School of Philosophy in Science 
%\label{ref:RNDFQCdL47BPA}(Polak and Trombik, 2022)?
\parencite[][]{polak_krakow_2022}? %
 We believe that we can adapt this existing metaphilosophical concept to illuminate the most important contemporary aspects of technology. While we were inspired by Heller's concept, it has also been greatly modified due to the differences between science and technology and the different historical backgrounds.



If we consider that good philosophy should shed some light on the current pressing problems faced by humanity, then ``philosophy in science'' was primarily an attempt to respond to the broad cultural crisis caused by the extreme positivist interpretations imposed on the sciences. This program was initiated by Michel Heller almost fifty years ago, and its name, which is used literally in the English version, has accompanied the journal ZFN since its first issues.\footnote{ZFN is an acronym of this journal's Polish title ``Zagadnienia Filozoficzne w~Nauce,'' which translates as ``Philosophical Problems in Science.''} The program has proven fruitful on many levels 
%\label{ref:RNDA6tNqkBrtA}(e.g., Brożek, Mączka and Grygiel, 2011; Polak, Mączka and Grygiel, 2017),
\parencites[e.g][]{brozek_philosophy_2011}[][]{polak_oblicza_2017}, %
 and it has served as a~bridge for developing a~dialogue between the fields of science and faith 
%\label{ref:RNDUruS65ukra}(Polak and Rodzeń, 2023).
\parencite[][]{polak_theory_2023}.%




Today, it is worth taking a~broader look at this philosophical program from the perspective of 75 issues of the journal. What are its prospects now? Does it still have a~\textit{raison d'être}? After all, the philosophy of positivism is already a~part of the history of philosophy, and the groundbreaking theories of the natural sciences are now standard topics for philosophers.



As science continues to provide new intellectual challenges, we believe that philosophy in science is still necessary. These days, however, we do not focus exclusively on physics, like positivism did in the past, because the range of sciences that significantly affect modern culture is now much broader. Indeed, it includes the humanities and the social sciences, such as economics, which has found an important place in ZFN, as well as technology.



\section{Technology as a~Philosophical Challenge}

Technology occupies a~special place among all the challenges facing modern society. It is broadly related to science in the sense that it makes extensive use of scientific developments, yet the problems posed by technological development cannot be reduced to the scientific problems associated with it. Indeed, they emphasize different goals: Science's goals are cognitive in nature (i.e., gaining knowledge), while technology has practical goals (i.e., taking actions).\footnote{In fact, the matter of relationships is more complex, but strong reductionist positions are difficult to maintain, see e.g. 
%\label{ref:RND5tIt2ZZuIB}(Franssen, Lokhorst and van de Poel, 2023, sec.2.1.-2.2.).
\parencite[][sec.2.1.-2.2.]{franssen_philosophy_2023}.%
}



Social media alienation of the individual 
%\label{ref:RNDH7lu5j71VV}(Reveley, 2013),
\parencite[][]{reveley_understanding_2013}, %
 digital surveillance 
%\label{ref:RNDPzpXA2ke0h}(Galič, Timan and Koops, 2017; Selinger and Rhee, 2021),
\parencites[][]{galic_bentham_2017}[][]{selinger_normalizing_2021}, %
 the undermining of democracy 
%\label{ref:RNDx8uEgRdtQu}(Olaniran and Williams, 2020),
\parencite[][]{olaniran_social_2020}, %
 and censorship 
%\label{ref:RNDQ3v9jty3M6}(Cobbe, 2021)
\parencite[][]{cobbe_algorithmic_2021} %
 are just some of the current problems that technology is accused of causing. Technology was supposed to be the embodiment of scientific rationality and provide tangible proof of the effectiveness and usefulness of science, but in reality, it has turned out to be far more complex and problematic than earlier philosophers thought it would. It is therefore difficult to understand the modern world without reference to both science and technology. Thus, the original ``philosophy in science'' program needs to be supplemented by a~complementary program for technology, which we will call ``philosophy in technology.'' These research programs share many metaphilosophical issues, but there are also some important differences between them. It therefore seems high time that we attempt to better define what philosophy in technology is and what it could be, because this should also help us gain a~better understanding of what technoscience could be.



Philosophy in technology explores the philosophical roots of technology.\footnote{By this we mean the process of creating technology, and in particular the process of ``design as decision making'' 
%\label{ref:RNDfPqxJakITo}(Franssen, Lokhorst and van de Poel, 2023).
\parencite[][]{franssen_philosophy_2023}. %
 For most engineers, a~concept ‘philosophy in technology' may seem strange, since neither in their studies nor in their engineering practice do they generally refer overtly to philosophical concepts. However, the lack of awareness of references to philosophy does not mean that philosophical issues are absent from engineering or that they are neutral---rather, it points to the shortcomings and problematic nature of such a~model of engineering education. We must note that there is already an emerging group of engineers who recognize the importance of philosophical issues. Dias 
%\label{ref:RNDygQT8YoJhl}(2019),
\parencite*[][]{dias_philosophy_2019}, %
 for example, is an excellent testimony to the beginning of the process of change. The analyses presented there of the role of philosophy in relation to technology and engineering are basically in line with the program of philosophy in technology presented here. Another example of the use of philosophical concepts directly in technology is also provided by Smolnik 
%\label{ref:RNDOHaNa2jaCS}(2017; 2018),
\parencites**[][]{smolnik_comparative_2017}[][]{smolnik_praxiological_2018}, %
 who shows the use of philosophical praxeology in systems engineering.} It is not concerned with any particular technical domain but rather with how different technologies can benefit from purely philosophical concepts, how technological domains often unwittingly adapt traditional philosophical concepts to meet their needs, and how from an abstract metaphysical, ontological, or axiological perspective, philosophy shapes and defines what technology does, how it develops, and how it evolves.\footnote{\textrm{It should be noted that we take a~broad view of technology here, as it is one of the oldest areas of human activity and has a~rich history of development }\label{ref:RNDG7pZYB1a7R}\textrm{(Hughes, 2005; Arthur, 2009)}\textrm{. Given the limited scope of the article, we refer here only to the most recent technologies, which we have chosen because of their current cultural significance. This does not mean, however, that philosophy in technology is limited to the narrow field of new digital technologies. It applies in principle to any technology, although the readability of the philosophical issues involved in a~given technology may vary greatly depending on the field.}}



Philosophy in technology also highlights the semantic gap between the concepts used by technology and the concepts that are understood in philosophy. We argue here that this semantic gap has become a~source of confusion that leads to misunderstandings between philosophers, the general population, and technologists. It also serves to downplay or exaggerate the risks and threats posed by technological development.



\section{Philosophy in Technology versus the Philosophy of Technology}

Philosophy of technology can be viewed from many perspectives. As we see it, it can be seen as (1) a~systematic clarification of the nature of technology as an element and product of human culture. Alternatively, it can be regarded as (2) a~systematic investigation of the practices involved in inventing, designing, engineering, and making technological artifacts or (3) a~systematic reflection on a~technology's consequences for humanity.



What distinguishes the pre-existing philosophy of technology is the external (from technology) perspective that it adopts and its aims. Technical systems, networks of interactions, artifacts, and so on are analyzed ``from the outside,'' as a~given object of philosophical reflection. In other words, they are considered from a~chosen philosophical perspective, imposing chosen philosophical view on technology. In its broadest sense, technology is therefore simply an object of reflection when attempting to formulate certain general relationships. A~typical aim of philosophy of technology is to understand the philosophical implications of technology and its products.



Philosophy in technology, in contrast, takes an ``internal'' perspective, because we are interested in the philosophy that underlies a~particular technology. In other words, we want to reconstruct and consider the philosophy that is embodied in the technology.\footnote{\textrm{Evidently, such reconstruction is always biased by certain }\textrm{\textit{a~priori}}\textrm{ accepted philosophical concepts, but these can be reasonably modified in the course of critical discussion (see below).}} We stress here that the mere ideological declarations of the technology's creators are, at most, of secondary importance, because what we are interested in here is what a~technology actually does and the philosophical basis for it.



The aim of philosophy in technology is to understand what philosophical concepts, assumptions, and values have been used in the process of creating a~particular technology, technical solution or artefact. In doing so, we hope to gain a~better understanding of the object of study. In other words, philosophy in technology is an important preparation for philosophy of technology. Even more important is the practical purpose---to raise awareness of the role of philosophy for engineering and to remove philosophical obstacles to the development of technology. Examples of such blocking effects of philosophical concepts on the development of AI can be found, see for example 
%\label{ref:RNDaJaJ2anBaB}(Smith, 2019; Krzanowski, 2021; Wooldridge, 2021).
\parencites[][]{smith_promise_2019}[][]{krzanowski_road_2021}[][]{wooldridge_road_2021}.%




Thus, philosophy in technology (1) searches for the implicit philosophical grounding for technology and engineering and the role it plays in shaping technological solutions; (2) explicates the ontological, metaphysical, axiological, and methodological dimensions of technology; and (3) clarifies the semantic gap between technical and philosophical concepts and attempts to bring them together under one perspective. The latter endeavor could involve concepts such as agents, autonomy, intelligence, the mind, ethics, justification, responsibility, phenomenology, selfhood, personhood, knowledge, wisdom, privacy, power, right vs. wrong, ontology, truth conditions, verification, and so on, although the list is potentially endless.



If we compare philosophy in technology with well-known concepts, such as Carl Mitcham's distinction between the engineering philosophy of technology and the humanistic philosophy of technology, we see that they are orthogonal. According to Mitcham 
%\label{ref:RNDhvaU2eqT5c}(1994, p.62):
\parencite*[][p.62]{mitcham_thinking_1994}:%




Engineering philosophy of technology begins with the justification of technology or an analysis of the nature of technology itself---its concepts, its methods, its cognitive structures, and objective manifestations. It then proceeds to find that nature is manifested throughout human affairs and, indeed, even seeks to explain both the nonhuman and the human worlds in technological terms. [...] Humanities [...] philosophy of technology seeks by contrast insight into the meaning of technology---its relation to the transtechnical: art and literature, ethics and politics, religion.



Philosophy in technology is located somewhere between the world of engineers and the world of humanists, but it takes a~different perspective. It looks for the philosophy that is involved in technology rather than reflecting on the nature of technology or its relation to trans-technical spheres. It analyzes how engineers use philosophical concepts and what the broader philosophical implications are of using these transformed concepts. The aims of philosophy in technology also differ from Mitcham's two variations of the philosophy of technology. So, what are the specific details of this new approach?



\section{Philosophy in Technology as a~Research Program}

Philosophy in technology is not a~given set of philosophical propositions to be shared and incorporated into the development of technology. This program is a~critical study of the philosophical foundations of technology, and its purpose is to critically discuss these foundations in order to benefit technology primarily but also philosophy itself. This will enable technology to free itself from ideological traps, purify itself of erroneous or harmful elements, and provide developmental impulses. For philosophy, it opens up a~new field of inquiry and prompts it to contribute to the development of our techno-scientific civilization.



Thus, philosophy in technology is a~metaphilosophical concept, one based on concepts of critical rationalism that have been adapted from the Kraków School of Philosophy of Science 
%\label{ref:RNDlk1NKaFwmE}(Polak and Trombik, 2022).
\parencite[][]{polak_krakow_2022}.%




Philosophy in technology therefore attempts to clarify the philosophical roots of technology by (a) explaining how philosophy is present in technology and engineering (e.g., fundamental philosophical assumptions, the philosophical concepts involved, the axiology of decisions); (b) identifying the role that philosophy plays in technology and engineering (i.e., philosophy for technology and engineering); (c) stimulating a~discussion of the philosophical foundations and implications of new technologies, such as to minimize any existential threats; (d) using philosophical reflection to shape a~more humanistic technology; and (e) opening up the technical perspective to philosophical analysis.



In order to deepen our discussion about the philosophical foundations of technology, we need involve not just philosophers but also the representatives of technology. This will not be possible without a~change in both parties' mutual attitudes, so it is also necessary to look for new ways of teaching philosophy at technical universities in order to bridge the gap between these two fields.



\section{Methodological Remarks}

As a~research program, philosophy in technology was created as an extension and adaptation of the concept of philosophy in science, which Michel Heller developed in the 1980s primarily to analyze the relationship between philosophy and physics. This concept has since proven to be very useful for highlighting the relevance of philosophy not just to physics but also other natural sciences 
%\label{ref:RNDVkK51mwtvT}(e.g., Brożek, Mączka and Grygiel, 2011; Polak, Mączka and Grygiel, 2017).
\parencites[e.g][]{brozek_philosophy_2011}[][]{polak_oblicza_2017}. %
 However, reflections on the problems of modern technology have made it evident that an analogous concept is needed to analyze the relationship between philosophy and technology, but what methods should philosophy in technology apply? We have already mentioned that a~discussion of the philosophical foundations of technology should be rooted in the critical rationalism framework of the Krakow school of philosophy of science, which was inspired mainly by the thinking of Karl Popper. We list some proposals below, but the list remains open for further discussion.\footnote{It is worth noting that Tavani 
%\label{ref:RNDLFhvkJxZLj}(2013)
\parencite*[][]{tavani_ethics_2013} %
 independently proposed many similar aspects. He emphasized the role of critical reasoning skills when building an artificial ethical system.}



(I) Philosophy in technology is \textbf{a~reflection on the classical philosophical problems in technology}. It is analogous to philosophy in science because we propose tracing the presence and roles of the great classical philosophical questions in technology, such as the nature of free will, the mind, intelligence, autonomous agents, and so on, so that we may be able to identify and analyze references to classical philosophical concepts such as matter and time 
%\label{ref:RNDhexFp1myYp}(e.g., Bolter, 1984).
\parencite[e.g][]{}. %
 Technology is not just philosophy-laden---it also influences our thinking as a~source of models and metaphors. Understanding what the intellectual contribution of technology is to our comprehension of reality is an important task for philosophers, but it is one that is all too often quietly overlooked.



(II) Philosophy in technology explores \textbf{how classical philosophical concepts can be adapted to meet the needs of technology}. Of course, we are aware that it is generally not possible to apply classical concepts directly, because they were forged for different purposes and embedded in specific conceptual frameworks. For technology, we should therefore adapt classical concepts, keeping in mind that while they are indeed inspired by classical concepts, they are not equivalent to them. An example of this could be adapting Aristotelian phronetic ethics to machine ethics 
%\label{ref:RNDbiuxGW1s1Q}(Polak and Krzanowski, 2020b; 2020a).
\parencites[][]{polak_phronetic_2020}[][]{polak_ethics_2020}. %
 An important and interesting issue here is the task of \textbf{formalizing classical concepts,} so they can be made as specific as possible and translated into a~language that fits the pragmatics of a~technical implementation 
%\label{ref:RNDfLa2jAp0Oa}(e.g., Janusz, 2006; Tavani, 2013).
\parencites[e.g][]{darowski_relacja_2006}[][]{tavani_ethics_2013}.%




(III) Philosophy in technology is a~\textbf{disclosure and critical analysis of technology that exposes philosophical biases and assumptions}, reconstructs accepted philosophical concepts in technology and engineering 
%\label{ref:RNDoEDOjv2dcv}(e.g., Smith, 2019),
\parencite[e.g][]{smith_promise_2019}, %
 and clarifies the unclear use of concepts 
%\label{ref:RNDswcuKJXAtp}(e.g., Cervantes et al., 2019).
\parencite[e.g][]{cervantes_artificial_2019}. %
 Engineers who create and use technology refer to philosophy, and even if they are unaware of it, they rely on serious philosophical assumptions in their actions. They use these assumptions mostly subconsciously and uncritically, following the principles they have learned without usually caring about the far-reaching, non-technical consequences of their actions.\footnote{Bertrand Russell aptly pointed out this general problem over a~century ago. See the quote at the beginning of this article 
%\label{ref:RNDRQghSwDKdg}(Russell, 1912, pp.243–244).
\parencite[][pp.243–244]{russell_problems_1912}.%
} On the other hand, even when they are aware of the philosophical significance of the decisions they make, their lack of philosophical experience makes them exceptionally ill-equipped to avoid naive or extremely reductionist solutions 
%\label{ref:RNDwGgYe307qY}(cf., Gordon, 2019).
\parencite[cf][]{}.%




(IV) Philosophy in technology analyzes \textbf{the consequences of philosophical prejudices in technology}, thus determining their role in specific technical realizations and analyzing the consequences and possible postulates for any changes in the philosophical foundations 
%\label{ref:RND8G80tiuaAZ}(e.g., Smith, 2019; Suchacka, Muster and Wojewoda, 2021; Wieczorek and Jędrzejko, 2021).
\parencites[e.g][]{smith_promise_2019}[][]{suchacka_human_2021}[][]{wieczorek_conscience_2021}. %
 In this way, philosophy in technology contributes to the long-term beneficial development of humanity, and in this sense, it could just as easily be called ``philosophy for technology.''



\section{Framework for Technology–Theology (Technology-Religion) Relationships Analysis}

\footnotetext{I~would like to thank Jacek Rodzeń for his valuable comments on philosophy and technology, as well as for his lengthy discussions on the issue of the neo-Scholastic reinterpretation of science. }

Technology today plays various important roles in daily life, so the import of the philosophical aspects of technology stretches beyond philosophy itself. One important area is the impact of technology on religion and theology 
%\label{ref:RND6iHI23xmLt}(e.g., Rodzeń, 2016).
\parencite[e.g][]{salamon_religia_2016}.%




Contemporary discussions about technological impact on religious practice and religions include, for example, technological spiritual enhancement 
%\label{ref:RNDhDvMKKHNhx}(e.g., Wildman and Stockly, 2021)
\parencite[e.g][]{wildman_spirit_2021} %
 or the theological aspects of human-like robots 
%\label{ref:RNDE8Mj0eb1Iv}(Balle, 2022).
\parencite[][]{balle_theological_2022}. %
 The classical religions of today also face important challenges like secularization, and at the heart of such issues lies the question of the profound cultural changes brought about by the rapid development of technology. Will technology displace religion? How will the message of faith be shaped for people who are surrounded by the wondrous realm of technology, which often obscures reality.\footnote{Recall Baudrillard's concept of simulacra 
%\label{ref:RNDZ2pAlKAFSR}(Baudrillard, 1994).
\parencite[][]{baudrillard_simulacra_1994}.%
}



In the field of theology, we could observe that the cultural changes brought about by technology's exceptionally rapid development in the 21st century make the classical theological concepts unclear and incomprehensible to modern people, because these concepts were created within the context of a~completely different worldview. This is particularly evident in Catholic theology, which is based on the concepts, ideas, and worldview of medieval culture (e.g., the contribution of St. Thomas Aquinas). Attempts to reinterpret modern culture within this medieval conceptual framework began as early as the nineteenth century with Leo XIII's encyclical \textit{Aeterni Patris} (1879), but these were doomed to failure as evidenced by the problems with receiving the discoveries of modern science 
%\label{ref:RNDXipy1XACss}(Polak and Rodzeń, 2023).
\parencite[][]{polak_theory_2023}. %
 The same applies to the latest technologies and the culture based on them.



Any attempt to solve mentioned problems should begin with a~proper understanding of the realm of technology. If we understand the philosophical role of technology, it will become easier to understand how we can incorporate it into theology or religious practices. With a~proper understanding of technology, its goals, and the values it embodies, one can perhaps hope to navigate between the extremes of fanatical optimism about technology and a~fear-driven techno-skepticism, because both of these extremes pose a~risk to rational human beings and threaten to ideologize religion in the context of technology. After all, theology has always built its message on the existing philosophy through which a~given culture expresses itself.



In reality, technology is even closer to theology than it is to science due to its direct involvement in the sphere of human activity. Theology, after all, concerns itself with the practical life of people, albeit from the perspective of faith rather than technical action. However, the two fields are united by the question of a~person's practical life (\textit{praxis}), which is why a~mutual interaction between these spheres is inevitable.\footnote{Today's increasingly bold takeover of areas of faith by technology 
%\label{ref:RNDjyLc08C3bq}(see, e.g. Wildman and Stockly, 2021)
\parencite[see.g.][]{wildman_spirit_2021} %
 is an expression of the contemporary crisis of theology and religious faith as classically understood. It should be noted, however, that deep interactions between the spheres of faith and technology have been taking place for centuries and took a~particularly interesting form in the Middle Ages 
%\label{ref:RND5O7qGsB8yT}(Ovitt, 1987).
\parencite[][]{ovitt_restoration_1987}. %
 (I am especially grateful to Jacek Rodzeń for bringing this important issue of the proximity of science and technology to our attention).}



Due to its goals, philosophy in technology can serve as a~convenient platform for a~dialogue to take place between modern technology and theology. It could provide theology with the concepts and elements of the current worldview that are needed to modernize the theological vision. In turn, by analyzing the ``inner'' philosophy of technology, we can hope that theology will not isolate itself from this sphere and instead become more sensitive to the important problems that condition the development of modern technology (e.g., axiology). From the point of view of technology, thanks to such a~high-level dialogue, the far-reaching effects of technology, which go far beyond purely technical applications, will become clearer. In other words, the dialogue between theology and technology represents an important step toward the humanization of technology. Moreover, if theology does not wish to be reduced to a~blind, irrational opponent of technology, it should engage in such a~dialogue. This dialogue seems feasible because an analogous process has already developed at the interface of science and theology, one where the concept of philosophy in science has played an important role.



\section{Conclusions}

The new digital technologies place many demands on engineering, including some of a~non-technical nature. In the past, classical engineering operated within requirements that were clearly defined, precise (i.e., a~permissible range of parameters was specified), and measurable (quantifiable). Today's engineering, in contrast, works with requirements of an extremely non-technical nature, such as requiring ethical or social behavior. Such problems should prompt engineers to automatically turn their attention to philosophy. While this may give the impression that only some recent technologies are directly related to philosophy, we can identify philosophical problems in other areas of technology. Some philosophical concepts were even directly applied in classical engineering.\footnote{An example of direct application of philosophical theories in ‘classical' engineering was analyzed by Maksymilian Smolnik, e.g. application of Tadeusz Kotarbiński's praxiological model for mechanical engineering 
%\label{ref:RNDEc2uzeNmGt}(Smolnik, 2018)
\parencite[][]{smolnik_praxiological_2018} %
 as well Józef Konieczny praxiological models 
%\label{ref:RND4NAUE7bCk2}(Smolnik, 2017).
\parencite[][]{smolnik_comparative_2017}.%
}



The lessons we can draw from this discussion are as follows:\footnote{The conclusions were formulated together with Roman Krzanowski.}



(1) Technology tends to substitute its own meaning for terms with traditional connotations in philosophy, but usually there is no awareness of what new meanings are being created. Indeed, the differences between the meanings of technological and philosophical terms are often so great that they may refer to completely different things, such as in the case of ethics, ethical behavior, justice, agency, autonomy, intelligence, the mind, and so on. This lies at the root of many significant misunderstandings, and this confusion with terms can even become a~tool for ideological manipulation.



(2) Changes in the meaning of concepts applied to technology can have serious consequences, not only within academic discussions but also for sociocultural change. Incorrect meanings also lead to a~myopic vision of technology.



(3) To better understand technology, we need to understand its foundations in terms of the philosophical concepts and assumptions of technology. We need a~full disclosure and critical analysis of technology to expose its philosophical biases and assumptions.



(4) For technological development, we need to understand how classical philosophical concepts can be adapted to meet the needs of technology.



(5) Philosophy in technology is also important for painting a~broader picture of the technology's impact. For example, it could serve as a~conceptual bridge for analyzing the relationship between technology and theology.



(6) There should be an open and frank dialogue where both sides (i.e., technologists with a~philosophical bent and philosophers with a~technological understanding) can freely exchange their ideas without fear of being dismissed as ignoramuses or simpletons.



By drawing attention to the important role of philosophy in technology, we hope to facilitate a~technological development that is better suited to the complex nature of \textit{us homo sapiens}. We also hope that it will mitigate, at least a~little, the scale of the crises that humanity is experiencing as a~result of the unusually rapid transformations affecting most areas of our lives.



\section{Acknowledgements}

I~would like to firstly thank Roman Krzanowski, with whose cooperation the idea of ``Philosophy in Technology'' finally matured. In many discussions, rehearsals, and joint papers, this vision was gradually clarified. Of course, all the errors and ambiguities in this present article are entirely my own. I~would also like to thank Roman for his efforts in creating the ``Philosophy in Technology'' conference series, because he is the true \textit{spiritus movens} behind this series.



I~would also like to thank Łukasz Mścisławski for organizing the second instance of the aforementioned conference and being kind enough to contribute many critical comments for my text that certainly helped to improve it.



I~am also very grateful to Jacek Rodzeń for the discussions on philosophy and technology that we have been having for more than a~decade. Many ideas were born under the influence of these discussions. I~would also like to thank Jacek for his valuable and profound comments on this text.



I~would also like to thank the anonymous reviewers for their valuable comments and inspiration for further research.



\section{References}

Arthur, W.B., 2009. \textit{The Nature of Technology}. New York; London: Free Press (Simon \& Schuster) \& Penguin.



Balle, S., 2022. Theological Dimensions of Humanlike Robots: A~Roadmap for Theological Inquiry. \textit{Theology and Science}, [online] pp.1–25. https://doi.org/10.1080/14746700.2022.2155916.



Baudrillard, J., 1994. \textit{Simulacra and Simulation}. Ann Arbor: University of Michigan Press.



Bolter, J.D., 1984. \textit{Turing's man: western culture in the computer age}. Chapel Hill: University of North Carolina Press.



Brożek, B., Mączka, J. and Grygiel, W.P. eds., 2011. \textit{Philosophy in Science: Methods and Applications}. Kraków: Copernicus Center Press[202F?]: Konsorcjum Akademickie. Wydawnictwo.



Cervantes, J.-A., López, S., Rodríguez, L.-F., Cervantes, S., Cervantes, F. and Ramos, F., 2019. Artificial Moral Agents: A~Survey of the Current Status. \textit{Science and Engineering Ethics}. [online] https://doi.org/10.1007/s11948-019-00151-x.



Cobbe, J., 2021. Algorithmic Censorship by Social Platforms: Power and Resistance. \textit{Philosophy \& Technology}, [online] 34(4), pp.739–766. https://doi.org/10.1007/s13347-020-00429-0.



Dias, P., 2019. \textit{Philosophy for Engineering: Practice, Context, Ethics, Models, Failure}. SpringerBriefs in Applied Sciences and Technology. [online] Singapore: Springer. https://doi.org/10.1007/978-981-15-1271-1.



Franssen, M., Lokhorst, G.-J. and van de Poel, I., 2023. Philosophy of Technology. In: E.N. Zalta and U. Nodelman, eds. \textit{The Stanford Encyclopedia of Philosophy}, Spring 2023. [online] Stanford, CA: Metaphysics Research Lab, Stanford University. Available at: {\textless}https://plato.stanford.edu/entries/technology/{\textgreater} [Accessed 2 October 2023].



Galič, M., Timan, T. and Koops, B.-J., 2017. Bentham, Deleuze and Beyond: An Overview of Surveillance Theories from the Panopticon to Participation. \textit{Philosophy \& Technology}, [online] 30(1), pp.9–37. https://doi.org/10.1007/s13347-016-0219-1.



Gordon, J.-S., 2019. Building Moral Robots: Ethical Pitfalls and Challenges. \textit{Science and Engineering Ethics}. [online] https://doi.org/10.1007/s11948-019-00084-5.



Heller, M., 2019. How is philosophy in science possible? \textit{Philosophical Problems in Science (Zagadnienia Filozoficzne w~Nauce)}, [online] (66), pp.231–249. Available at: {\textless}https://zfn.edu.pl/index.php/zfn/article/view/482{\textgreater} [Accessed 6 October 2021].



Hottois, G., 2023. \textit{Technoscience}. [online] Translated by J.A. Lynch. Encyclopedia of Science, Technology, and Ethics.. Encyclopedia.com. Available at: {\textless}https://www.encyclopedia.com/science/encyclopedias-almanacs-transcripts-and-maps/technoscience{\textgreater} [Accessed 2 October 2023].



Hughes, T.P., 2005. \textit{Human-Built World: How to Think about Technology and Culture}. science.culture. [online] Chicago, IL: University of Chicago Press. Available at: {\textless}https://press.uchicago.edu/ucp/books/book/chicago/H/bo3626290.html{\textgreater} [Accessed 4 December 2023].



Janusz, R., 2006. Relacja etyczno-psychologiczna w~ujęciu obiektowym. In: R. Darowski, ed. \textit{Philosophiae \& musicae: księga pamiątkowa z~okazji jubileuszu 75-lecia urodzin księdza profesora Stanisława Ziemiańskiego SJ}. Kraków: ‘Ignatianum' - Wydawnictwo WAM. pp.375–380.



Krzanowski, R., 2021. The road to conscious machines: AI through failed ideas. \textit{Philosophical Problems in Science (Zagadnienia Filozoficzne w~Nauce)}, [online] (70), pp.171–181. Available at: {\textless}https://zfn.edu.pl/index.php/zfn/article/view/556{\textgreater} [Accessed 28 December 2021].



Mitcham, C., 1994. \textit{Thinking Through Technology The Path Between Engineering And Philosophy}. [online] Chicago: University of Chicago Press. Available at: {\textless}http://archive.org/details/ThinkingThroughTechnologyThePathBetweenEngineeringAndPhilosophy{\textgreater} [Accessed 13 July 2023].



Olaniran, B. and Williams, I., 2020. Social Media Effects: Hijacking Democracy and Civility in Civic Engagement. \textit{Platforms, Protests, and the Challenge of Networked Democracy}, [online] pp.77–94. https://doi.org/10.1007/978-3-030-36525-7\_5.



Ovitt, G., 1987. \textit{The Restoration of Perfection: Labor and Technology in Medieval Culture}. New Brunswick: Rutgers University Press.



Polak, P., 2019. Philosophy in science: A~name with a~long intellectual tradition. \textit{Philosophical Problems in Science (Zagadnienia Filozoficzne w~Nauce)}, [online] (66), pp.251–270. Available at: {\textless}https://zfn.edu.pl/index.php/zfn/article/view/472{\textgreater} [Accessed 6 October 2021].



Polak, P. and Krzanowski, R., 2020a. Ethics in autonomous robots as philosophy in silico: The study case of phronetic machine ethics. \textit{Logos i~Ethos}, (52), pp.33–48. https://doi.org/10.15633/lie.3576.



Polak, P. and Krzanowski, R., 2020b. Phronetic Ethics in Social Robotics: A~New Approach to Building Ethical Robots. \textit{Studies in Logic, Grammar and Rhetoric}, [online] 63(1), pp.165–183. https://doi.org/10.2478/slgr-2020-0033.



Polak, P., Mączka, J. and Grygiel, W.P. eds., 2017. \textit{Oblicza filozofii w~nauce: księga pamiątkowa z~okazji 80. urodzin Michała Hellera}. Kraków: Copernicus Center Press.



Polak, P. and Rodzeń, J., 2023. The Theory of Relativity and Theology: The Neo-Thomist Science–Theology Separation vs. Michael Heller's Path to Dialogue. \textit{Theology and Science}, [online] 21(1), pp.157–174. https://doi.org/10.1080/14746700.2022.2155917.



Polak, P. and Trombik, K., 2022. The Kraków School of Philosophy in Science: Profiting from Two Traditions. \textit{Edukacja Filozoficzna}, (2(74)), pp.205–229. https://doi.org/10.14394/edufil.2022.0023.



Reveley, J., 2013. Understanding Social Media Use as Alienation: A~Review and Critique. \textit{E-Learning and Digital Media}, [online] 10(1), pp.83–94. https://doi.org/10.2304/elea.2013.10.1.83.



Rodzeń, J., 2016. Religia a~technika. In: J. Salamon, ed. \textit{Przewodnik po filozofii religii: nurt analityczny}, Przewodniki po Filozofii. Kraków: Wydawnictwo WAM. pp.403–419.



Rodzeń, J., 2019. Nauka a~technika (technonauka). In: S. Janeczek, M. Walczak, A. Starościc and Wydział Filozofii (Katolicki Uniwersytet Lubelski Jana Pawła II), eds. \textit{Metodologia nauk. Cz. 1: Czym jest nauka?}, Dydaktyka Filozofii. Lublin: Wydawnictwo KUL. pp.655–681.



Russell, B., 1912. \textit{The problems of philosophy}. [online] New York[202F?]: H. Holt. Available at: {\textless}http://archive.org/details/problemsofphilo00russuoft{\textgreater} [Accessed 17 July 2023].



Selinger, E. and Rhee, H.J. (Judy), 2021. Normalizing Surveillance. \textit{SATS}, [online] 22(1), pp.49–74. https://doi.org/10.1515/sats-2021-0002.



Smith, B.C., 2019. \textit{The Promise of Artificial Intelligence: Reckoning and Judgment}. Cambridge, MA: The MIT Press.



Smolnik, M., 2017. A~Comparative Analysis of Praxiological Networks and Selected IDEF Models. \textit{INNOVATOR. Journal of the European TRIZ Association}, [online] 4(2), pp.4–11. Available at: {\textless}https://www.etria.eu/innovator/ETRIAjournal2017vol04.pdf{\textgreater} [Accessed 9 October 2023].



Smolnik, M., 2018. A~praxiological model of creative actions in the field of mechanical engineering. In: S. Koziołek, L. Chechurin and M. Collan, eds. \textit{Advances and Impacts of the Theory of Inventive Problem Solving: The TRIZ Methodology, Tools and Case Studies}. [online] Cham: Springer International Publishing. pp.135–145. https://doi.org/10.1007/978-3-319-96532-1\_13.



Suchacka, M., Muster, R. and Wojewoda, M., 2021. Human and machine creativity: social and ethical aspects of the development of artificial intelligence. \textit{Creativity Studies}, [online] 14(2), pp.430–443. https://doi.org/10.3846/cs.2021.14316.



Tavani, H.T., 2013. \textit{Ethics and Technology: Controversies, Questions, and Strategies for Ethical Computing}. 4th ed. Hoboken, NJ: Wiley.



Wieczorek, K.T. and Jędrzejko, P., 2021. The Conscience of a~Machine? Artificial Intelligence and the Problem of Moral Responsibility. \textit{Er(r)go. Teoria - Literatura - Kultura}, [online] (42), pp.15–34. https://doi.org/10.31261/errgo.10418.



Wildman, W.J. and Stockly, K.J., 2021. \textit{Spirit Tech: The Brave New World of Consciousness Hacking and Enlightenment Engineering}. First edition ed. New York: St. Martin's Press.



Wooldridge, M., 2021. \textit{The Road to Conscious Machines: The Story of AI}. London, UK: Penguin.

\end{document}

