\begin{artengenv}{Tadeusz Sierotowicz}
	{Theology of science: Its collocation and critical role for understanding of limits of theological and scientific investigations}
	{Theology of science\ldots}
	{Theology of science: Its collocation and critical role for understanding of limits of theological and scientific investigations}
	{ISR-Bolzano,\\IISS ``Gandhi'' Merano}
	{The paper presents a~brief outline of the Michał Heller's programme of theology of science, with a~specific attention to its collocation and critical role with respect to both theology and science. The former consideration is based on a~third domain of truths (Hans Urs von Balthasar), while the latter is inspired by Józef Tischner's presentation of religious thinking. Theology of science as such will be described with the reference to Larry Laudan's approach, considered here as a~very useful and pragmatic tool for the description of basic concepts of this theology.
	}
	{theology of science, third domain of the truths, research tradition, Michał Heller, Hans Urs von Balthasar, Józef Tischner, Larry Laudan.}





\section{Introduction }

\lettrine[loversize=0.13,lines=2,lraise=-0.03,nindent=0em,findent=0.2pt]%
{I}{}n the vast area of study designated ``faith and reason'' the theology of science occupies a~special position. While considered a~branch of theology, the theology of science has a~specific topic of study, namely science, which do not belong to theology proper. This situation raises a~number of uncertainties, including questions regarding its methodology and locus of enquiry. In this essay, I~will focus my attention on two issues: the collocation of theology of science in the realm of theological investigation, and the purpose it serves for both theology and science. In fact, the theology of science serves in communicating faith to a~secular world and in developing a~reasonable and informed faith. But not only, as it will be explained later.



I~will follow the approach of Donald Lococo developed in his \textit{Life in One Breath: Meditations on Science and Christology} 
%\label{ref:RND9gungMSkGl}(2021).
\parencite*[][]{lococo_life_2021}. %
 As he writes, modern reflections on theology and science evince a~``large lacunae, owing to the near ablation from consideration'' of some of the ``most significant twentieth-century Catholic theologians, namely Balthasar and Karl Rahner''.\footnote{It is worth to note, that Karl Rahner thought is also present in Michael Heller's programme of theology of science described below 
%\label{ref:RNDAVFdVHLsHU}(cf., Macek, 2014, pp.80–81; Maziarka, 2016, p.13).
\parencites[cf][pp.80–81]{macek_teologia_2014}[][p.13]{maziarka_w_2016}. %
 } ``[O]ne can hardly conceive of building on the theology of any denomination without paying attention to its most deeply influential thinkers'' 
%\label{ref:RNDmdG2RhaCy2}(Lococo, 2021, p.11).
\parencite[][p.11]{lococo_life_2021}.%




Lococo mentions these two names only. Of course, in order to give a~fuller account of the development of theology in the context that interests me, the list of names should be considerably longer, including theologians as Ratzinger, Guardini or Teilhard de Chardin. As this text is not intended to be a~review paper, but aims to formulate a~working hypothesis inspired by Balthasar's approach, his works will provide the basis for further considerations. As far as the role of science in religious thinking and the limits of the theology of science are concerned, I~will be guided by the thought of Józef Tischner (Polish priest and an eminent philosopher), whose ideas on religious thinking were developed in the Kraków academic milieu, not without a~dialogue with the thought of Michael (Michał) Heller. At the same time it is worthwhile to remember the possible correlation between Balthasar's and Tischner's thinking 
%\label{ref:RNDgtYyfLBgCP}(see Wołowski, 2019).
\parencite[see][]{wolowski_problem_2019}.%




With these thoughts in mind, I~will state my point of view as follows. As to the understanding of theology of science itself, I~will follow Michael Heller's approach, briefly outlined in the first section.\footnote{Donald J. Lococo has observed that ``over the last quarter-century and more, the relationship between science and faith has been addressed by numerous scholars, resulting in the publication of a~surfeit of books, many with titles so similar that it is difficult to distinguish between them'' 
%\label{ref:RNDK1kUdqaI1u}(Lococo, 2021, p.10).
\parencite[][p.10]{lococo_life_2021}. %
 Rather than attempting to summarize the immense number of resources available, I~will focus specifically on Michael Heller's approach to the theology of science. An extensive bibliography appears in the next section. For an understanding of the difference between the conjunctions And and Of in the aggregation of theology and science see 
%\label{ref:RNDdstarT2BLU}(Tyson, 2022, pp.1–4);
\parencite[][pp.1–4]{tyson_christian_2022}; %
 for other programmes of theology of science see, for example 
%\label{ref:RNDbo15qFh3RT}(Lococo, 2002; 2021; Rodzeń, 2021; Tyson, 2022; Wilkinson, Harrison and Tyson, 2022).
\parencites[][]{lococo_towards_2002}[][]{lococo_life_2021}[][]{rodzen_teologia_2021}[][]{tyson_christian_2022}[][]{wilkinson_after_2022}.%
} Then, I~will enquire into the theology of Hans Urs von Balthasar, a~pre-eminent theologian in the Catholic tradition, looking for the answer to the question about the collocation of theology of science in the domain of theological research (section 2). Next, Michael Heller's programme will be further examined, and framed, in the broader context of Larry Laudan's research tradition (section 3).\footnote{The first paper on that topic was published by Michael Heller in 1982 and by Józef Życiński in 1984 
%\label{ref:RNDsiDwMoTqAq}(Heller, 1982; Życiński, 1984; see Polak, 2015; Rodzeń, 2021).
\parencites[][]{heller_stworzenie_1982}[][]{zycinski_w_1984}[see][]{polak_teologia_2015}[][]{rodzen_teologia_2021}. %
 M. Heller's writings on theology of science goes back to 1992 
%\label{ref:RNDBBxH30MYSL}(for an overview see, Oleksowicz, 2020, pp.759–760).
\parencite[for an overview se][pp.759–760]{oleksowicz_we_2020}. %
 As to the main bibliography, see 
%\label{ref:RNDYGcSrQlTg0}(Heller, 1996; 2015; Macek, 2014; Mączka and Urbańczyk, 2015; Maziarka, 2016; Polak, 2016; Oleksowicz, 2020; Rodzeń, 2021).
\parencites[][]{heller_new_1996}[][]{maczka_wstep_2015}[][]{macek_teologia_2014}[][]{maczka_teologia_2015}[][]{maziarka_w_2016}[][]{polak_teologia_2016}[][]{oleksowicz_we_2020}[][]{rodzen_teologia_2021}. %
 As to the science and religion dialogue in the Kraków School, see 
%\label{ref:RNDibxffqWZbw}(Brożek and Heller, 2015; Obolevitch, 2015; Polak and Rodzeń, 2021; 2023).
\parencites[][]{brozek_science_2015}[][]{obolevitch_relationship_2015}[][]{polak_science-religion_2021}[][]{polak_theory_2023}.%
} In the conclusion, Józef Tischner view of religious thinking will be questioned in order to describe the role of theology of science in developing a~reasonable faith and in understanding of limits of both science and theology (section 4).



\section{Michael Heller's theology of science programme}

This essay has its \textit{raison d'être} in the faith of the Church. Before proceeding, an important clarification must be made. The main participants in the conversation reported in this essay belong to the circle of Catholic Church. Coherently, the views expressed by the Christian Catholic theology represent what can be considered ``the First Truth Discourse'' on God and His Revelation.\footnote{I~use this expression following 
%\label{ref:RNDsAiBIiwJxZ}(Tyson, 2022, pp.26–39).
\parencite[][pp.26–39]{tyson_christian_2022}. %
 } Thus, it presupposes the existence of God, who reveals Himself, and the legitimacy of theology which ``begins with the self-revelation of the triune God in the Incarnation of the divine Logos, the Word, the Son, and the expositor [\textit{Auslegei}] of the Father'' 
%\label{ref:RND3JFTplAl6q}(Balthasar, 2004, p.11).
\parencite[][p.11]{balthasar_theo-logic_2004}. %
 Stated otherwise, the essay has its locus in theology, which refers to talk about God and God's Word of Revelation in the Catholic Church 
%\label{ref:RNDTzN1ooihOw}(an expression, \textit{mutatis mutandis}, of Barth, 2010, p.2).
\parencite[an expressioof][p.2]{barth_church_2010}.%




The purpose of Revelation is not to communicate truths about the natural world that satisfy the innate curiosity of the human being, but above all to show the path leading to salvation.\footnote{Cardinal Baronio has expressed this idea very clearly: ``The intention of the Holy Spirit is to teach us how to go to heaven, and not how the heavens go'' 
%\label{ref:RNDfkZ7pNppDW}(McMullin, 1999, p.185).
\parencite[][p.185]{mcmullin_augustine_1999}. %
 } Revelation is not informative in the way that ordinary knowledge is informative. It does not add to our list of facts about the world or the universe in which we live. Instead, it is existential in the sense that it concerns the deepest dimension of human existence and gives direction and meaning to human life. For this reason, the knowledge gained through revelation cannot be combined with the findings of science, which focus on the material universe and do not touch the deeper question of meaning. As John Paul II puts it, addressing a~group of scientists and researchers, ``Divine Revelation, of which the Church is the guarantor and witness, does not in itself entail any scientific theory of the universe, and the assistance of the Holy Spirit does not guarantee the explanations we propose regarding the physical constitution of reality'' 
%\label{ref:RNDAWwOmegoOr}(John Paul II, 1983).
\parencite[][]{john_paul_ii_address_1983}.%




Nevertheless, within the framework of theology, it is possible to reflect critically on those truths of Revelation which allow for a~deeper understanding of science as a~specifically human activity dedicated to the world created by God. This critical reflection is the very purpose of that branch of theology that might be called ``theology of science''.



According to Michael Heller, the theology of science is a~branch of theology that engages the experimental sciences, their existence, foundations, methods and results, with the understanding that the experimental sciences study the world created by God. As a~branch of theology, the theology of science has all the characteristics of theology as a~discipline. Its context for reflection is the life of the believer, the Church, and its methods and sources are not extraneous to those used in other theological disciplines. Consequently, a~theology of science can be thought as an authentic research tradition within Catholic theology.\footnote{In the words of Michael Heller, ``the purpose of the theology of science is the same as that of all theology, but always with reference to the specific object as it is proper for a~given theological discipline''. Therefore, ``the theology of science is dedicated to a~critical reflection on those data of Revelation which allow us to contemplate the sciences as a~specific human activity'' of exploring the world created by God 
%\label{ref:RNDEMnlehO8ff}(Heller, 1996, pp.97 and 99).
\parencite[][pp.97 and 99]{heller_new_1996}.%
}



The basic premise of the theology of science is thus one that has already been put forward: the statement that the universe was created by God. It should be specified here that, for theologians, the concept of the universe encompasses all that has been created by God. Of course, the universe of science and the universe of theology are not identical. The former pertains to the material world while the universe of theology goes beyond the material or visible world. However, while the two realms are separate, theology cannot bring forward theses that contradict those advocated by the sciences. It cannot, therefore, enter arbitrarily into the specific domain of the experimental sciences.



The thesis that the world, and indeed the universe, came into existence through God's special design has to be completed by the thesis which affirms the absolute dependence of everything that exists on the Creator. Traditional theology, following in the footsteps of traditional philosophy, thus used to speak of the ``contingency of the world''. The thesis that the world is utterly dependent on God not only for its creation but also for its continued existence is one of the essential elements of Christian doctrine concerning creation; however, the way God interacts with world is not a~question that will be addressed in this essay.



Rather, I~will focus on the rationality and comprehensibility of the world, a~primary focus within the theology of science. As Heller writes, ``with the theology of creation is connected another problem, the problem of the rationality of the world. […] by the rationality of the world I~mean that property of the world by which it can be studied rationally. This investigation of the world belongs to the domain of science and the accomplishments within the sciences are the best attestation to the rationality of the world. From a~theological perspective, the rationality of the world is the mark of the Creator's rationality'' 
%\label{ref:RND5Pt6JTkPyq}(Heller, 2015, p.21).
\parencite[][p.21]{maczka_wstep_2015}. %
 This theme is frequently highlighted in the theology of science. In the Christian doctrine of creation, it belongs to a~study of the Logos-Word. Olaf Pedersen writes that ``the identification of the divine logos with Christ [...] make it possible to connect in a~fundamental way faith in Christ with the quest for understanding the inherent rationality of nature, or even to see this rationality as a~sign of God's immanence in the world'' 
%\label{ref:RNDwFFgy3RA9U}(Pedersen, 1990, p.147).
\parencite[][p.147]{pedersen_historical_1990}.%




Finally, the question of values needs to be mentioned in connection with the theology of science. It is well known that the method of the experimental sciences is insensitive to values: normative and value statements do not belong to the language of the experimental sciences. This thesis has been put forward since at least the time of the Vienna Circle formed in the 1920s. It does not mean, however, that the material world has nothing to do with values. On the contrary, from the standpoint of theology, the creation of the world is essential for the realization of God's project of love and salvation. This project takes into account not only everything that the experimental sciences seek to discover and investigate, but also what is called a~``value system'', that is an axiology. Hence, reflection on the experimental sciences from an axiological point of view is also one of the tasks of the theology of science.



\section{A~Third domain of truths }

Clear from what has been so far written is that the theology of science belongs to the discipline of theology and shares with science an interest in the natural world, albeit from a~particular perspective, which is different from that of the experimental sciences. Michał Heller and his commentators emphasise that it is a~perspective which considers the world as created by God. Therefore, the theology of creation is considered a~pillar of the theology of science. But what is the precise meaning of that statement? What is the specific, material object of the theology of science, which, while guaranteeing its belonging to the field of theological enquiry as such, nevertheless distinguishes it from other theological disciplines and from the sciences as well? The question pertains, on the one hand, to the place of the theology of science within theology and, on the other, to the relation of theology of science to the natural sciences. In short, the question is about the specific domain (the material object) of the theology of science. To properly belong to theology and science this domain must fulfil the following conditions: (1) it must belong to the domain of theology as such; (2) it must also belong to the domain of the sciences; (3) it must allow theology of science to be considered a~distinct theological discipline, distinct from the sciences; and, last but not least (4) it must ensure the autonomy of theology and science.\footnote{The fourth condition may appear not obvious. Some scholars consider it a~``myth'' 
%\label{ref:RND4gavVkK159}(as Paul Tyson in his book on theology of science: Tyson, 2022, chap.9.1.).
\parencite[as Paul Tyson in his book on theology of science:][chap.9.1.]{tyson_christian_2022}. %
 Nevertheless, other researchers like 
%\label{ref:RNDCSZx97AYPT}(Lococo, 2021)
\parencite[][]{lococo_life_2021} %
 and the scholars from the so called Kraków School 
%\label{ref:RND65Ck0OFvEK}(Obolevitch, 2015; Polak, 2015; see also Macek, 2014)
\parencites[][]{obolevitch_relationship_2015}[][]{polak_teologia_2015}[see also][]{macek_teologia_2014} %
 hold up the theses of autonomy.}



One of the possible solutions to the problem suggested by this list of criteria has been suggested by Szczurek 
%\label{ref:RNDFEv8g2AgQ1}(2015, pp.133–134).
\parencite*[][pp.133–134]{maczka_teologia_2015-1}. %
 In his essay on the structure of theology of science, he advocates that theology of science is an authentically theological discipline working with scientific results as interpreted by the philosophy of science in the light of Revelation and the Ultimate Aim of the man. Interesting as this thesis may be, Szczurek's suggestion can be further elaborated and slightly changed since it identifies the formal object of theology of science with science as seen by philosophy of science. Consequently, it presents the theology of science in the guise of philosophy of science, that is, as another way of meditating science and its achievements. Whether a~more radical interpretation of theology of science that does not collapse the discipline into one that already exists remains to be investigated. Some stimulating remarks which outline a~possible, more profound, I~even dare to say -- ontological -- insight, can be found in the works of Hans Urs von Balthasar, mainly in the first volume of his \textit{TheoLogic} 
%\label{ref:RNDU5G7NWTLbq}(Balthasar, 2000).
\parencite[][]{balthasar_theo-logic_2000}. %
 Let us follow his train of thoughts.



According to Balthasar, ``the world as it concretely exists is one that is always already related positively or negatively to the God of grace and supernatural revelation''. Consequently, ``the world, considered as an object of knowledge, is always already embedded in this supernatural sphere, and, in the same way, man's cognitive powers operate either under the positive sign of faith or under the negative sign of unbelief'' 
%\label{ref:RNDoszRtdcIfY}(Balthasar, 2000, p.11).
\parencite[][p.11]{balthasar_theo-logic_2000}. %
 The author of \textit{TheoLogic} emphasizes that the natural fundamental structures of the world and knowledge are by no means eliminated or altered in their essence by their inclusion in the supernatural sphere. Therefore, philosophical thought, in its capacity for abstraction, can probe them apart from conscious reflection on their supernatural imbuement. However, as philosophical thought probes the concrete object of enquiry deeper and deeper, it begins to encounter an increasing amount of theological data. This is so, because ``the supernatural takes root in the deepest structures of being, leavens them through and through, and permeates them like a~breath of an omnipresent aroma''. For that reason, Balthasar asserts that it is impossible not to include theological data in thinking about the nature of things: ``it is not only impossible, it would be sheer folly to attempt at all costs to banish and uproot this fragrance of supernatural truth from philosophical research; the supernatural has too strongly impregnated nature so deeply that there is simply no way to reconstruct it in its pure state'' 
%\label{ref:RNDEfMdb8iJYG}(Balthasar, 2000, p.12).
\parencite[][p.12]{balthasar_theo-logic_2000}.%




Balthasar proceeds to describe three ways in which theological data is embedded in concrete philosophical thought. There is, of course, the unconscious assimilation of such data in philosophical enquiry (Balthasar gives the example of Plato). Then there is a~kind of secularization of theological data, whereby the data is given the status of rational, properly human truths (e.g. modern rationalism and existentialism). The first way, however, is no longer accessible given our knowledge of the incarnation, and the second way entails a~prejudice against divine Revelation, which can hardly be justified theologically and is therefore unsuitable for a~theology of science. There remains a~third way: ``to describe the truth of the world in its prevalently worldly character, without, however, ruling out the possibility that the truth we are describing in fact includes elements that are immediately of divine, supernatural provenance''. According to this statement, between the two domains of the natural and the supernatural, we need to postulate what Balthasar, following Romano Guardini, calls ``a third domain of truths, that genuinely belong to creaturely nature yet do not emerge into the light of consciousness until they are illuminated by a~ray of the supernatural'' 
%\label{ref:RNDMUG7H6s4qP}(Balthasar, 2000, p.12).
\parencite[][p.12]{balthasar_theo-logic_2000}.%




This third domain of truths is constituted by truths ``visible'' only under certain conditions, that is only when illuminated by ``a supernatural ray''. Which truths belong to this domain? Balthasar indicates, as an example, the First Vatican Council teaching that natural reason suffices ``to know with certainty the one true God as our Creator and Lord through creatures'' 
%\label{ref:RNDj6SAMRICLY}(Balthasar, 2000, p.12).
\parencite[][p.12]{balthasar_theo-logic_2000}. %
 This truth could be the foundation for a~specific, material object of theology of science. As a~matter of fact, it satisfies all four criteria stated at the beginning of this section. Indeed, it is the supernatural light (theology) that illuminates the natural world (of science). What is so illuminated by that supernatural light is what theology of science explores. Given this approach, it follows as a~matter of course that theology and science remain effectively autonomous in their specific fields\footnote{Paul Tyson, in his remarkable book on theology of science, tries to rethink ``the very idea of ‘science' and ‘religion'''. His way of thinking is that of a~hermeneutic spiral: to think what is ``unfamiliar'' (religion), starting with what is familiar (science). It entails a~new integration between understanding (religion) and knowledge (science), and -- what is more important here -- enables ``to define Christian theology within the truth categories of modern science'' 
%\label{ref:RNDPZHjcLNJtk}(Tyson, 2022, p.9).
\parencite[][p.9]{tyson_christian_2022}. %
 Consequently, he meticulously constructs an ``Integrative Zone of Knowledge and Understanding'', where such definition could be achieved 
%\label{ref:RNDIo9hVpQikB}(Tyson, 2022, chap.9).
\parencite[][chap.9]{tyson_christian_2022}. %
 Tyson's approach to a~theology of science is very stimulating. Nevertheless, it gives an impression of infringing slightly the autonomy of science from theology, as it seems to attribute in some sense a~priority to knowledge (science). Needless to say, his concept of an Integrative Zone does not correspond to the Balthasar's idea of a~third domain if truths. }.



\section{Theology of Science as Research Tradition}

Having described the specific object of theology of science, I'll rest for a~moment my case to present Heller's theology of science as a~research tradition. One can find a~useful guide in the model of science proposed by Larry Laudan.\footnote{The reference to Larry Laudan's approach is purely pragmatic as it offers useful linguistic tools for the description of basic concepts of theology of science.} His approach situates itself in the mainstream of the philosophy of science set forth by Thomas Kuhn and Imre Lakatos. Laudan's model, which as a~basic unit of the description of the development of science accepts the so-called research traditions, interprets science as intellectual activity of solving problems of different kind. A~research tradition is a~``group of general assumptions concerning the objects and processes in the field of research and the assumptions concerning the methods that should be applied in order to solve problems and to construct new theories in this field'' 
%\label{ref:RNDP4kT2duLGA}(Laudan, 1977, p.81),
\parencite[][p.81]{laudan_progress_1977}, %
 or, in a~more synthetic way: a~research tradition is ``a set of ontological and methodological do's and don'ts'' 
%\label{ref:RNDPpMkbBKFua}(Laudan, 1977, pp.79–80).
\parencite[][pp.79–80]{laudan_progress_1977}.%




A~given research tradition consists of various theories (which are sometimes in conflict with each other). Among various research traditions in the same field of research, the more successful ones are those that leads to solving more different problems, and which imply fewer anomalies and unresolved problems. The full research tradition definition must also take into account ``certain metaphysical and methodological commitments, which, taken as a~whole, define a~particular tradition and distinguish it from other traditions''. One might introduce the following schematic description of research traditions:



$$\textrm{Research Tradition} \to (\textit{I}; \textit{O}; \textit{R}; \textit{M}; \{\textit{T}\}; \{\textit{p}\})$$





in which the individual symbols stand for, respectively:



\textit{I} - metaphysical and methodological commitments,



\textit{O}\&\textit{R} -- basic objects\&relationships,



\textit{M} -- methodology accepted in the particular research tradition,



\{\textit{T}\} –the set of theories proposed in the framework of the research tradition to solve the set of problems of the vital importance, and



\{\textit{p}\} -- problems occurring in the given field of reflection (at the first glance there are two kinds of problems: ``\textit{first order problems}; they are substantive questions about the object which constitute the domain of any given science'' 
%\label{ref:RNDZLDioMzJCv}(Laudan, 1977, p.15; Laudan's italics);
\parencite[][p.15, Laudan's italics]{laudan_progress_1977}; %
 and conceptual problems that relates to the theory itself 
%\label{ref:RNDcFfSjqVzJB}(Laudan, 1977, chap.2)
\parencite[][chap.2]{laudan_progress_1977}%
).



Laudan believed that his approach could be applied, after making appropriate changes, to other fields of knowledge 
%\label{ref:RNDvSvf5OM7nS}(Laudan, 1977, pp.189–192).
\parencite[][pp.189–192]{laudan_progress_1977}. %
 Thus, Michał Heller's program of theology of science can be shortly narrate as a~specific theological research tradition operating in the area of theological research. If so, the meaning of symbols in the above-mentioned synthetic definition of research tradition could be as follows:



\textit{I} -- the existence of God as described in Christian Tradition (supernatural);



\textit{O}\&\textit{R} -- a~third domain of truths,



\textit{M} -- overall methodology of theology in the Christian Tradition,



\{\textit{T}\} -- e. g. evolution and creation as presented in
%Heller, 1996, pp. 81-103,
\parencite[][pp.81–103]{heller_new_1996}



\{\textit{p}\} -- \textit{first order problems}: contingency, comprehensibility of the world, creation, evolution 
%\label{ref:RNDfiGDiANilx}(for a~more detailed compilation, see: Macek, 2014, pp.67–137);
\parencite[for a~more detailed compilation, see:][pp.67–137]{macek_teologia_2014}; %
 \textit{conceptual problems}: (1) if theology of science is a~branch of theology, then all criteria of its evaluation are that of theology, and have nothing in common with science, (2) has theology of science bring any new solution to significant problems (or formulate any new problem), which without its contribution would not be known in theology or in science?



But, after all, who needs such a~research tradition? Doesn't it promise more than it can deliver, letting down theologians and scientists as unable to offer anything new to both theological and scientific reflection?\footnote{For a~critical appraisal of M. Heller's research tradition, see: 
%\label{ref:RNDaBjCNlFI0m}(Polak, 2016).
\parencite[][]{polak_teologia_2016}.%
} It seems that at least two reasons in favor of Heller's theology of science can be given. The first one is that of its contribution to the announcement of the Gospel. Here, the Message of John Paul II to George V. Coyne remains a~\textit{magna carta}. Just a~few passages from the Message to give an example of what is at stake here:



\begin{quote}
the Church and the scientific community will inevitably interact; their options do not include isolation. Christians will inevitably assimilate the prevailing ideas about the world, and today these are deeply shaped by science. The only question is whether they will do this critically or unreflectively, with depth and nuance or with a~shallowness that debases the Gospel and leaves us ashamed before history. […] Contemporary developments in science challenge theology far more deeply than did the introduction of Aristotle into Western Europe in the thirteenth century. Yet these developments also offer to theology a~potentially important resource. Just as Aristotelian philosophy, trough the ministry of such great scholars as St Thomas Aquinas, ultimately came to shape some of the most profound expressions of theological doctrine, so can we not hope that the sciences of today, along with all forms of human knowing, may invigorate and inform those parts of the theological enterprise that bear on the relation of nature, humanity and God? 
%\label{ref:RNDwWPUuTPsXi}(John Paul II, 1988).
\parencite[][]{john_paul_ii_letter_1988}.%
\end{quote}




An example of a~``potentially important resource'' could be beauty. As Lococo rightly writes, ``beauty and truth are linked in physical science, as is reason with our feelings'' 
%\label{ref:RNDZVUlwnRRaQ}(Lococo, 2021, p.61).
\parencite[][p.61]{lococo_life_2021}. %
 Of course, one cannot forget, that beauty is not a~scientific category. Nevertheless, the beauty of the first image of a~black hole 
%\label{ref:RNDkfhWC8se6X}(Szybka, 2020)
\parencite[][]{szybka_remarks_2020} %
 or of an electron micrograph ``makes us enthused that data gleaned from it will be significant'' 
%\label{ref:RNDrj0F0bgEOT}(Lococo, 2021, p.61).
\parencite[][p.61]{lococo_life_2021}. %
 Again, significant for what? Significant and valid, explains Lococo, ``to posit that the beauty-that-beings-are, is being-in-unity'' 
%\label{ref:RNDer14uuzj0I}(Lococo, 2021, p.62).
\parencite[][p.62]{lococo_life_2021}. %
 These considerations lead to Balthasar's theological syntheses offered in his treatise on theological aesthetics 
%\label{ref:RND1vmgwWHYSm}(Balthasar, 2009).
\parencite[][]{balthasar_glory_2009}.%




The second reason is the critical role played by theology of science towards both science and theology. John Paul II, in the quoted letter has stated that: ``Science can purify religion from error and superstition; religion can purify science from idolatry and false absolutes. Each can draw the other into a~wider world, a~world in which both can flourish'' 
%\label{ref:RNDT2GRkbg70G}(John Paul II, 1988).
\parencite[][]{john_paul_ii_letter_1988}. %
 In the conclusion, I~would like to examine this issue following Józef Tischner's approach to religious thinking, as it provides a~very profound insight into the question at hand. Of course, and it is to be stressed clearly, Tischner's thinking is rather weakly related to the theology of science. It has different object, vocabulary, philosophical roots -- shortly, it is a~pretty different research tradition 
%\label{ref:RNDAzR4glR189}(Sierotowicz, 2018).
\parencite[][]{sierotowicz_filozofia_2018}. %
 However, using the language which is typical for Tischner, the critical role of theology of science (that is not absolutizing both the scene and scientific rationality) can be described clearly enough.



\section{Conclusions: On the role of theology of science}

For Tischner religious thinking is the thinking of ``the man whose reason is seeking faith, and whose faith is seeking reason thinks in a~religious manner. His faith becomes manifest in his thinking, and his thinking becomes manifest in his faith'' 
%\label{ref:RNDgTb2DaGPgC}(Jagiełło, 2020, p.221).
\parencite[][p.221]{jagiello_jozef_2020}. %
 The religious thinking makes possible different, sometimes contradictory, theologies. But each theology exists because of religious thinking, not vice versa. Religious thinking, as with all thinking, is ``someone's thinking, thinking with \textit{someone} and thinking about \textit{something}. Thus, thinking has three dimensions: a~subjective dimension (I think), a~dialogic dimension (I think with you), an objective dimension (we think about it)'' 
%\label{ref:RNDPxkm6Nv3o8}(Jagiełło, 2020, p.224).
\parencite[][p.224]{jagiello_jozef_2020}. %
 Roughly speaking these dimensions correspond to Tischner's description of a~human being as a~dramatic existence: ``to be a~dramatic being is to: live in the present time, with other people around and the ground under one's feet. Man would not be a~dramatic existence but for these three factors: opening up to another man, opening up to a~scene of drama and to the passage of time'' 
%\label{ref:RNDiuKoFYuCsx}(Jagiełło, 2020, p.165).
\parencite[][p.165]{jagiello_jozef_2020}. %
 Religious thinking in its objective dimension turns to the stage of human drama:



\begin{quote}
For the people involved in living the drama, writes Tischner, the stage of life is above all a~plane of meetings and partings, a~sphere of freedom, in which man searches for a~home, bread and God, and where he finds a~graveyard. The stage is at man's \textit{feet}. […] Man experiences the stage by objectifying it, turning it into a~space filled with ‘objects', which he then arranges in a~variety of wholes that serve him looking for its essential design 
%\label{ref:RNDe1yjnsOJRq}(Jagiełło, 2020, p.166).
\parencite[][p.166]{jagiello_jozef_2020}.%
\end{quote}




However, in the context of religious thinking, is the objectified stage only at man's feet? That stage undergoes a~process of metaphorization. It turns into the metaphor of the true, proper reality. The stage as a~metaphor suggests movement from one domain of existence to another. This happens, when for example, somebody affirms ``my home is not a~true home, my true home has to be collocated in another world, and the same for happiness, love, real life'' 
%\label{ref:RNDzYUwLWkJ1b}(Tischner, 2011, p.388).
\parencite[][p.388]{tischner_myslenie_2011-1}. %
 Religious thinking is in opposition to all those interpretations of the scene that attribute absolute existence to what man's has under his feet. This way of looking at the scene binds all hope of human existence to the ``here and now'', attributing definitive existence to the scene. It thus becomes blind to the contingency and relative character of the scene. But above all, it forgets that the objectified world of man, the scene, and the only scene of the human drama, also manifests itself as a~metaphor of true existence 
%\label{ref:RNDCLoyQvx6ng}(Tischner, 2011, p.391).
\parencite[][p.391]{tischner_myslenie_2011-1}. %
 The non-absolute character of the scene is precisely where I~see the theology of science as occupying a~critical role, especially insofar as it points to the metaphorical character of the scene, and, consequently, to the limitation of the investigation dedicated exclusively to the scene (i.e., science).



On the other hand, the rapid development of the sciences and the increasingly profound understanding of the world of nature offered by experimental science, invites theology to adopt more than one metaphorical interpretation of the scene. Just to give an example of such interpretations, one can indicate the conviction, that the stage is the only intersubjective way to God or the belief that from the circumstance that our is the world of contingencies, follows the contingency of the world itself 
%\label{ref:RNDIKFvharL5V}(Tischner, 2011, pp.386–387).
\parencite[][pp.386–387]{tischner_myslenie_2011-1}.%
\footnote{For a~critical evaluation of these interpretations of the scene, see: 
%\label{ref:RND9o3DEW1KBs}(Johnson, 2019).
\parencite[][]{hunsinger_barth_2019}.%
}



These considerations permit to sum up the train of thought of the present paper. At first, the theology of science appears to be an authentic theological discipline, having as its basic objects and relationships of study the domain called by Hans Urs von Balthasar ``a third domain of truths''. The second point to be stressed is that the bond between science and theology within the theology of science appears both critical and bilateral. Besides, the theology of science preserves the rational character of both theology and science. In fact, science ``is never more reasonable than when it recognizes the limits of its methods, and never less so than when it presumes to be adequate to the full reality of the human and the divine.''\footnote{See J. McGrath in his introduction to 
%\label{ref:RNDqsiq6MpSCV}(Lococo, 2021, p.5).
\parencite[][p.5]{lococo_life_2021}. %
 } Rephrasing these words, one might say that theology is never more reasonable than when it recognizes the limits of its metaphors of the stage, and never less so than when it presumes to offer the unique metaphorization of the scene. The issues outlined above open up further research perspectives. To give just one example: a~systematic presentation of the science-faith/theology relationship in the works of Hans Urs von Balthasar. This topic seems urgent, as so far it has been almost completely ignored by researchers studying the Swiss theologian thought.



\end{artengenv}

