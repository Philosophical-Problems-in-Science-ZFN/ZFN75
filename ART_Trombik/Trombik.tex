\setcounter{secnumdepth}{0}





\begin{document}

Kamil Trombik (Uniwersytet Papieski Jana Pawła II w~Krakowie)



Andrzej Fuliński as a~Representative of the Concept of Philosophy in Science





SUMMARY



This paper analyzes selected issues related to the philosophy of the Krakow physicist Andrzej Fuliński. Since the 1970s, Fuliński has been strongly associated with the interdisciplinary milieu gathered around Heller and Życiński. His activity can therefore be considered within the context of the broader phenomenon known as the Krakow School of Philosophy in Science, which was founded by Heller and Życiński. This paper proposes the thesis that Fuliński's style of philosophy is connected with the concept of philosophy in science and tries to justify the thesis that Fuliński, due to his cooperation with the interdisciplinary milieu in Krakow and the specificity of his philosophical works, deserves to be regarded as a~representative of the Krakow School of Philosophy in Science.



Keywords: Andrzej Fuliński, the Krakow School of Philosophy in Science, philosophy in science, philosophy of nature, history of philosophy in Poland.



\section{Introduction}

The interdisciplinary approach to issues on the border between science and philosophy has become a~permanent part of Krakow's intellectual landscape, with an important element of this local tradition being the phenomenon of the so-called ``philosophizing scientists,'' who are researchers and thinkers who address problems specific to philosophy, especially the philosophy of science and the philosophy of nature, based on scientific investigations. Over the past century, these philosophizing scientists have included Marian Smoluchowski, Tadeusz Garbowski, Zygmunt Zawirski, and many others 
%\label{ref:RNDlA2xxfUKm7}(see, for example, Heller et al., 2007; Polak, 2011b; 2011a; 2018).
\parencites[sefor exampl][]{heller_krakowska_2007}[][]{polak_u_2011}[][]{polak_19th_2011}[][]{polak_tradycja_2018}. %
 Following World War II, cooperation between philosophers and scientists developed in Krakow, mainly among the friends of Karol Wojtyła 
%\label{ref:RNDSrab988fN6}(Heller and Mączka, 2006; Trombik, 2021; 2022).
\parencites[][]{heller_poczatki_2006}[][]{trombik_koncepcje_2021}[][]{trombik_stworzyc_2022}. %
 At the time, physicists associated with the Jagiellonian University were widely influential in this milieu, alongside others including Jerzy Janik and Andrzej Fuliński.



Following the election of Cardinal Wojtyła as pope, Janik and Fuliński remained active participants in local interdisciplinary projects, which were initiated from that point on by Michał (Michael) Heller and Józef Życiński. After 1978, both these scientists from the Jagiellonian University became involved in organizing seminars at Castel Gandolfo, which provided an opportunity for meetings and discussions between scientists, philosophers, and theologians throughout the pontificate of John Paul II. These meetings continued previous interdisciplinary conferences organized by Wojtyła during his time in Krakow in the 1960s and 1970s 
%\label{ref:RND4OJAhB0jgs}(Trombik, 2022).
\parencite[][]{trombik_stworzyc_2022}.%




Janik's work in philosophy has already had its initial reception within the Polish academic community 
%\label{ref:RNDcoIHcc2Db1}(Fuliński and Maślanka, 2015),
\parencite[][]{fulinski_profesor_2015}, %
 but the case is completely different with Fuliński's work. Nevertheless, this scientist's activity seems noteworthy for at least two reasons: First, it fits into the tradition of having a~dialogue between science and philosophy, something that was successfully achieved in the circle of the Polish Pope's associates. It therefore provides important evidence about the crossing of boundaries between natural sciences and philosophy that took place in Polish culture over the past several decades. Second, although Fuliński's academic achievements lie primarily in the area of statistical physics, he is not limited to the area of pure science. Taking various issues that are present in contemporary science as a~starting point, Fuliński has often expressed his philosophical competencies, as evidenced by the numerous, valued articles in which this physicist discussed various issues in the field of the philosophy of science and philosophy of nature.\footnote{Fuliński's papers were cited in various philosophical works, including a~coursebook on the philosophy of nature (Bugajak et al., 2009) and in books and papers by authors such as J. Życiński, A. Lemańska, K. Doliwa, A. Biegalska, S. Cisek, and J. Grzanka (e.g., Lemańska, 1996; Życiński, 1988, 1993, 2009, 2011; Biegalska, 2016; Doliwa, 2009).} It is worth mentioning that the term ``philosophical activity'' in Fuliński's case is not confined to a~short period---it accompanied him continuously for several decades. Moreover, in his articles, this physicist often returned to previously discussed philosophical issues, trying to philosophize within the context of the natural sciences at various stages on his scientific path.



Fuliński's ties to the interdisciplinary milieu centered around Heller and Zyciński, and this makes it possible to consider his activities within the context of the broader phenomenon known as the Krakow School of Philosophy in Science 
%\label{ref:RNDDzF5Xphpns}(Trombik, 2021, p.226; Polak and Trombik, 2022).
\parencites[][p.226]{trombik_koncepcje_2021}[][]{polak_krakow_2022}. %
 In this paper, I~propose that Fuliński's publications fit well with the style of practicing the philosophy of nature that was initiated by Heller. Moreover, I~believe that Fuliński himself, due to his cooperation with the local interdisciplinary milieu and the specificity of his philosophical works, deserves to be regarded as a~representative of the Krakow School of Philosophy in Science 
%\label{ref:RND8zxmRFLYrL}(see Polak and Trombik, 2022).
\parencite[see][]{polak_krakow_2022}.%




In the remainder of this article, I~will present Fuliński's profile and discuss a~selection of his philosophical views, with the focus being especially on those aspects of his philosophical activity that fit with the trend of philosophy in science 
%\label{ref:RNDrTT9EVYJ8P}(Polak, 2019; Trombik, 2021).
\parencites[][]{polak_philosophy_2019}[][]{trombik_koncepcje_2021}.%




\section{Between Kraków and Castel Gandolfo: Fuliński as a~Philosophizing Scientist}

Andrzej Fuliński began his academic career in Krakow. In 1955, he was awarded a~master's degree in theoretical chemistry at the Jagiellonian University before obtaining his doctorate five years later at the Polish Academy of Sciences in Warsaw. He later obtained his habilitation in 1966 at his \textit{alma mater}. Fuliński's scientific activity was highly appreciated by the academic community. In 1975, he became the head of the newly established Department of Statistical Physics at the Jagiellonian University. Fuliński, together with his colleagues, dealt primarily with describing the phenomena that occur in complex systems using broadly understood statistical physics methods. His research achievements resulted in, among other things, being awarded a~full professorship in 1980 and becoming director of the Institute of Physics at the Jagiellonian University.



A~period of increased scientific activity for Fuliński coincided with the initiatives of Michał Heller and Józef Życiński, who in Krakow had developed, on behalf of the Pontifical Academy of Theology, some large-scale interdisciplinary activities that had been previously initiated by Wojtyła 
%\label{ref:RNDRvVZszXRG0}(Trombik, 2022).
\parencite[][]{trombik_stworzyc_2022}. %
 Their areas of interest included, among other things, issues on the border between philosophy and physics, as well as the general methodology of science. They took up philosophical issues in the sciences and not just in their own publications. They also promoted and developed the idea of philosophy in science within the Center for Interdisciplinary Studies [in Polish Ośrodek Badań Interdyscyplinarnych (OBI)], which since the 1980s has been an important, although informal, institution aimed at deepening the dialogue between science and philosophy. This goal was achieved thanks to interdisciplinary meetings, conferences, and publications mainly appearing in the periodical \textit{Zagadnienia Filozoficzne w~Nauce} / \textit{Philosophical Problems in Science} 
%\label{ref:RNDT70NA0bkhO}(Heller and Mączka, 2006; Polak, 2019; Trombik, 2019).
\parencites[][]{heller_poczatki_2006}[][]{polak_philosophy_2019}[][]{trombik_origin_2019}.%




From the very beginning, Fuliński engaged in various interdisciplinary initiatives that were undertaken first by Wojtyła and then by Heller and Życiński. He participated in seminars, panel discussions, and conferences organized by the OBI 
%\label{ref:RNDo7weM8BKTD}(Liana and Mączka, 1999),
\parencite[][]{liana_z_1999}, %
 and he also took part in the ``Krakow Methodological Conferences'' that have replaced the earlier interdisciplinary meetings since the 1990s. Fuliński also regularly appeared at the Castel Gandolfo Seminars, which were held from 1980 at the summer residence of John Paul II. The Pope wanted these meetings to be a~continuation of the discussions on the border between science, religion, and philosophy that he had started with Krakow's scholars as early as the 1950s 
%\label{ref:RNDQ6OBEjinJo}(Janik, 1981, p.5; Nowina Konopka, 2020).
\parencites[][p.5]{janik_nauka_1981}[][]{nowina_konopka_kontakty_2020}. %
 Among physicists, Fuliński was, along with Janik, the most frequent participant in these seminars. During his stay in Castel Gandolfo, he had the opportunity to deliver a~number of papers, and his speeches were later published in the form of articles in special issues of \textit{Nauka–Religia–Dziej}e [Science–Religion–History], which has been in circulation since 1981.



Fuliński also published in the already mentioned \textit{Zagadnienia Filozoficzne w~Nauce} 
%\label{ref:RNDsCPEdhTFAz}(Fuliński, 2015; 2017).
\parencites[][]{fulinski_profesor_2015}[][]{fulinski_fluktuujacy_2017}. %
 His philosophical papers have also been published in magazines such as \textit{Znak} 
%\label{ref:RND3R9YuTdrnM}(Fuliński, 1993),
\parencite[][]{fulinski_o_1993}, %
 \textit{Studia Philosophiae Christianae} 
%\label{ref:RND0Kxb23GUyQ}(Fuliński, 1989),
\parencite[][]{fulinski_maszyna_1989}, %
 and \textit{Prace Komisji Filozofii Nauk Przyrodniczych} 
%\label{ref:RNDVq0vjcZb0Z}(Fuliński, 2010),
\parencite[][]{fulinski_czy_2010}, %
 as well as in post-conference materials published by the OBI 
%\label{ref:RND6HBKUnGiQn}(e.g., Fuliński, 1990c; 1991a; 2003).
\parencites[e.g][]{janik_glos_1990}[][]{heller_co_1991}[][]{heller_jednosc_2003-2}. %
 The topics of his work fit well with the issues raised by Heller, Życiński, and their students. Fuliński dealt with issues like the relationship between science and philosophy (together with an analysis of the ``two cultures'' phenomenon), the ontological aspects of physics, the problem of the mathematical universe, the issue of reductionism in science and philosophy, the issue of time, the issue of determinism, and the concept of chance. At the same time, these issues were vigorously discussed by the representatives of ``philosophy in science'' 
%\label{ref:RNDnISL43S4F6}(e.g., Trombik, 2021, pp.222–223),
\parencite[e.g][pp.222–223]{trombik_koncepcje_2021}, %
 and Fuliński himself regularly referred to the publications of Heller and Życiński in his works.



In the 1980s, Fuliński's cooperation with the OBI community deepened. The Krakow physicist even became one of the reviewers for Włodzimierz Skoczny's doctoral dissertation, which was titled ``Filozoficzne aspekty Zasady Antropicznej'' [``Philosophical Aspects of the Anthropic Principle''], written under the supervision of Życiński and defended at the Pontifical Academy of Theology in 1986. Fuliński was also keenly interested in the publications of Heller and Życiński. A~good example of this is his review of their book \textit{Wszechświat---maszyna czy myśl?} [\textit{The Universe---a machine or a~thought?}] that was published in the periodical \textit{Studia Philosophiae Christianae} 
%\label{ref:RNDkFSlc9WPFp}(Fuliński, 1989).
\parencite[][]{fulinski_maszyna_1989}. %
 It should also be noted that between 1988 and 1991---at the request of John Paul II and together with Heller, Życiński, and Zygmunt Kolenda---Fuliński prepared the work ``Reports on the socio-political situation in Poland'' 
%\label{ref:RNDfhCYv8sFZW}(Heller, 2020).
\parencite[][]{heller_jan_2020}.%
\footnote{The pope read these reports carefully. Fuliński recognized fragments of his observations on the current situation in Poland in the speeches of John Paul II during his pilgrimage to Poland in 1991.} This proves not only the enormous trust that the pope had in Fuliński but also the spirit of understanding and cooperation that existed between the Krakow physicist and the creators of philosophy in science, a~cooperation that continued into later years. Even over the last decade, Fuliński has repeatedly participated in various scientific initiatives of the Copernicus Center for Interdisciplinary Studies, an institution that was established by Heller after receiving the prestigious Templeton Prize in 2008, with this being a~21\textsuperscript{st} century continuation of the former OBI.



The indicated connections between Fuliński and the Krakow interdisciplinary milieu seem so important and so large scale that they provoke questions about the mutual dependencies that existed, including philosophical ones. When reconstructing Fuliński's views, it is worth noting their references to the concept of philosophy in science. Due to the limited length of this article, I~will limit myself here to discussing just some selected philosophical ideas in Fuliński's works, ones that will illustrate the mutual connections and dependencies, namely the issue of the relationship between science and philosophy (and also the relationship between science and religion), the problem of reductionism, and the dispute over the mathematical nature of the universe.



\section{Toward Interdisciplinary Research: Selected Philosophical Issues in Fuliński's Works}

The ``philosophy in science'' project, as initiated by Heller and Życiński in the Krakow milieu, was a~proposal to practice philosophy within the context of the results of contemporary mathematical and natural sciences 
%\label{ref:RNDBJKzmo7hg2}(Heller, 2019; Polak, 2019).
\parencites[][]{heller_how_2019}[][]{polak_philosophy_2019}. %
 As scholars coming from a~Catholic background, Heller and Życiński were formed during their studies in seminary by the spirit of the Thomistic philosophy of nature 
%\label{ref:RNDVQQ0QWGUtQ}(e.g., Heller, 2016, p.107),
\parencite[e.g][p.107]{heller_wierze_2016}, %
 but during their academic careers, they quickly developed a~style of practicing philosophy that was far removed from the Aristotelian and Thomistic trend. There were several reasons for this: According to Heller, Thomism as a~metaphysical system was not capable of creatively addressing key problems on the border between science and philosophy 
%\label{ref:RNDutIEF6MFrZ}(Heller, 1990).
\parencite[][]{janik_nowa_1990}. %
 Moreover, Heller was skeptical of all philosophical systems and critical of the so-called great syntheses in the form of Thomism, Hegelianism, and so on 
%\label{ref:RNDakBk0YlWCR}(see e.g., Heller, 2004, pp.139–146; 2011, pp.92–95).
\parencites[see e.g][pp.139–146]{heller_filozofia_2004}[][pp.92–95]{}. %
 Similar thoughts were echoed by Heller's students and colleagues,\footnote{In the case of Życiński, the problem is somewhat more complex, because he was impressed by some metaphysical systems like Whitehead's philosophical project. It should be noted, however, that Życiński himself never developed any philosophical synthesis that followed the example of the British thinker. In his works, especially from the 1990s, it is difficult to discern any attempt to develop anything like a~philosophical system.} who despite their strictly metaphysical interests, usually rejected the products of philosophical systems as being unsuitable for interdisciplinary research 
%\label{ref:RND20Alu6nKAd}(see Polak and Trombik, 2022).
\parencite[see][]{polak_krakow_2022}.%




Fuliński also shared this critical stance toward philosophy derived from the Aristotelian–Thomistic trend. In this aspect, his thoughts corresponded well with those of Heller and his colleagues. According to Fuliński, Thomistic philosophy was not only outdated, especially in the context of issues bordering science and philosophy, but also harmful in light of the social mission of the Church, which wanted to establish contact with contemporary intellectual culture. As Fuliński wrote, ``I have always had quite mixed feelings towards Thomism (and especially neo-Thomism), suspecting, probably not without reason, that it is today one of the causes of mutual distrust, not to say dislike or even sometimes hostility, between the community of people of science and the Church'' [Original Polish: ``zawsze miałem dość mieszane uczucia wobec tomizmu (a zwłaszcza neotomizmu) podejrzewając, chyba nie bez powodów, że jest on dziś jedną z~przyczyn wzajemnego braku zaufania, by nie powiedzieć: niechęci, czy wręcz nieraz wrogości, pomiędzy społecznością ludzi nauki i~Kościoła''] 
%\label{ref:RND9Al1BkOCEQ}(Fuliński, 1989, p.227).
\parencite[][p.227]{fulinski_maszyna_1989}.%




Fuliński shared the view that the discrepancy between science and Christianity may have its origins in the overly strong connection between the Church's teachings and neo-Thomistic philosophy, a~type of philosophy that is inadequate for addressing problems that have emerged in the context of the modern natural sciences, so it is unattractive for the scientific community. Elsewhere, Fuliński even suggested that the historical rooting of Thomism in Western culture has over time become one of the causes of the gap both between science and religion and, from a~broader perspective, between humanistic culture and scientific culture, thus contributing to the emergence of the so-called ``two cultures'' phenomenon 
%\label{ref:RNDDljNa4ujgb}(Snow, 1959).
\parencite[][]{snow_two_1959}. %
 Fuliński wote: ``It is possible that the roots [of this phenomenon] could be looked for in the Thomistic doctrine. The Thomist assumes that he possesses the Absolute Truth, which gives him the right to treat in advance all those who do not want to recognize this Truth. This mentality was then taken over by both armchair philosophers\footnote{Fuliński used this term to refer to those philosophers who, as he used to say, ``instructed scientists in how science should be interpreted (see Bergson).''} and scientism'' [Original Polish: ``Niewykluczone, że korzeni [tego zjawiska] można by szukać w~doktrynie tomistycznej. Tomista zakłada, że posiadł Prawdę Absolutną, co daje mu prawo do traktowania z~góry wszystkich tych, którzy owej Prawdy nie chcą uznać. Tę mentalność przejęli następnie zarówno filozofowie fotelowi jak i~scjentyści''] 
%\label{ref:RNDRFHs2ULwU7}(Fuliński, 1993, p.32).
\parencite[][p.32]{fulinski_o_1993}. %
 It was obvious to Fuliński that the Thomistic philosophy of nature, even in a~modified form like Louvain Thomism,\footnote{The Louvain type of Thomism was an attempt at the turn of the 19\textsuperscript{th} and 20\textsuperscript{th} centuries to harmonize modern science with Aristotelian–Thomistic philosophy, an attempt that was ultimately unsuccessful, and the Louvain type of Thomism did not gain traction beyond a~narrow circle of Catholic philosophers. } could not be reconciled with contemporary scientific knowledge, so it should be abandoned altogether before looking for a~more adequate system with better methods for analyzing the results of natural sciences.



His approach was not intended to discredit the intellectual heritage of Christianity, however. On the contrary, the development of a~different type of reflection was intended to establish a~new platform of understanding between science and faith.\footnote{Remarks related to this can also be found in, \textit{inter alia}, the transcript of the discussion panel named ``Between knowing and believing'' 
%\label{ref:RNDqeg3vkq5hM}(Fuliński, 1990b),
\parencite[][]{heller_watpliwosci_1990}, %
 where Fuliński, in the context of the question about the relationship between science and faith, referred to the methodological proposals of I. Barbour 
%\label{ref:RNDyH17SjNXze}(Fuliński, 1991b).
\parencite[][]{fulinski_glos_1991}.%
} Together with his colleagues and students, Heller made a~similar assumption when developing the concept of philosophy in science 
%\label{ref:RND2nNT55PeNf}(e.g., Polak and Rodzeń, 2021; 2023; Trombik and Polak, 2022).
\parencites[e.g][]{polak_science-religion_2021}[][]{}[][]{trombik_stworzyc_2022}. %
 It is worth noting here that Fuliński clearly pointed to the historical importance of Christianity in the emergence of modern science 
%\label{ref:RNDJeMq0ODV8b}(Fuliński, 1981),
\parencite[][]{janik_fizyka_1981}, %
 which over time also became the main view of, among others, Życiński, who devoted a~book to this issue 
%\label{ref:RNDsgL6UHZsVO}(Życiński, 2000).
\parencite[][]{zycinski_inspiracje_2000}.%




Both Heller and Życiński were convinced of the need to develop a~philosophy that was in close contact with science and the latest logic and methodologies. The prerequisites for practicing this kind of philosophy include anti-separationism (i.e., a~rejection of the thesis that there is a~radical epistemological rift between the sciences and philosophy) and an openness to the changes and modifications being dictated by the development of the sciences and methodological reflection. A~similar approach can be seen in the works of Fuliński,\footnote{It is worth noting here that in Fuliński's works, one can find numerous references to the philosophical tradition, as well as to contemporary philosophy, especially in the area of the philosophy of physics and the philosophy of science. In addition to the works of Heller and Życiński, Fuliński refers to, among others, the works of K.R. Popper, T. Kuhn, P.K. Feyerabend, W. Quine, W. Heisenberg, and even E. Husserl 
%\label{ref:RNDszgCrC0v4r}(see Fuliński, 1996).
\parencite[see][]{wszolek_o_1996}. %
 This shows that Fuliński attempted to gain a~deeper understanding of the philosophical aspects of natural science rather than limiting his analyses to just the professional perspective of a~theoretical physicist.} which serve as a~good example of the practical application of the assumptions of the ``philosophy in science'' project.



One of the basic goals behind Heller's and Życiński's efforts was an attempt to deepen the dialogue between philosophy and natural sciences. The search for contacts between broadly understood humanities and the mathematical–empirical sciences is also noticeable in Fuliński's work. Taking part in the discussion of the ``two cultures,'' he emphasized how numerous interactions between science and culture exist that are visible, for example, at the level of language:



Such interactions can be seen, for example, in the transition and processing of concepts, in the cycle: philosophy and common parlance, science, and common parlance and general culture. Philosophy or common parlance introduces some concepts. Science takes them over, when it is prepared to do so, and on examining them carefully, processes them in its own way. Eventually, this concept is returned, albeit in a~processed form, into everyday language and common culture. The most obvious example is the concept of the atom [...] An example of a~concept that is currently being refined by detailed science, and at the same time, in a~purified form, is beginning to pass into general culture, is the notion of heredity, which originates in common parlance and the related more technical notion of the gene, innate traits, and so on. Finally, an example of a~concept that is just beginning to enter this processing process is the concept of time [Original Polish: ``Oddziaływania takie można zauważyć na przykład w~przechodzeniu i~przetwarzaniu pojęć, w~cyklu: filozofia i~język potoczny -- nauka szczegółowa -- język potoczny i~kultura ogólna. Filozofia lub język potoczny wprowadza jakieś pojęcia. Przejmuje je nauka szczegółowa -- gdy jest do tego przygotowana -- i~badając dokładnie, przetwarza je po swojemu. W~końcu następuje zwrot tego pojęcia, ale już w~postaci przetworzonej, do języka potocznego i~do kultury powszechnej. Najbardziej oczywistym przykładem jest pojęcie atomu […] Przykładem pojęcia, które aktualnie jest uściślane przez naukę szczegółową i~jednocześnie w~oczyszczonej postaci zaczyna przechodzić do kultury ogólnej, jest pochodzące z~języka potocznego pojęcie dziedziczności i~związane z~nim bardziej techniczne pojęcie genu, cechy wrodzonej i~tak dalej. W~końcu przykładem pojęcia, które właśnie zaczyna wchodzić w~ów proces przetwarzania, jest pojęcie czasu''] 
%\label{ref:RNDNoSWodS43b}(Fuliński, 1981, p.22).
\parencite[][p.22]{janik_fizyka_1981}.%




The interpenetration of the precise language of science with the ambiguous language of culture is a~key, although not the only, area of possible interaction between natural sciences and the humanities. Fuliński also noticed other examples of mutual influences, paying attention, for example, to the importance of various cultural creations in the context of scientific discovery 
%\label{ref:RNDRUHZI7iwr5}(Fuliński, 1981, p.15).
\parencite[][p.15]{janik_fizyka_1981}. %
 The view shared by Fuliński about the unity of the world and therefore the need to integrate the various disciplines that describe the same world (despite them coming from different perspectives) is the main reason for him rejecting the separation paradigm. He also expresses hope ``that the understanding of the unity of the world and the unity of culture will return to our way of thinking in a~purified and processed form in the specific sciences, including the humanities'' [Original Polish: ``rozumienie jedności świata i~jedności kultury powróci do naszego sposobu myślenia w~postaci oczyszczonej i~przetworzonej w~naukach szczegółowych, z~naukami humanistycznymi włącznie.''] 
%\label{ref:RND31dBiqqbfm}(Fuliński, 1981, p.28).
\parencite[][p.28]{janik_fizyka_1981}. %
 The idea of the unity of the world, and consequently the postulated unity of knowledge, would remedy the existing rupture in culture, as manifested by the gap between humanists and representatives of empirical sciences.\footnote{In this context, Fuliński referred to the ``mirror metaphor'' from Professor A. Staruszkiewicz, writing, among other things, ``physics is a~mirror reflecting the world. About a~hundred years ago, it was a~mirror perhaps not the most perfect, a~little cloudy, and the image of the world was not the clearest. But it was one mirror and one image. Today, the image of the world provided by physics is much more accurate and sharper, but the mirror has shattered into many pieces that we cannot fit together. This metaphor can be extended, in particular, to philosophy and physics, and indeed to the entire culture: unfortunately, we still have a~broken mirror. It would be good if we managed not to merge this mirror, but to create one, a~new one'' (Fuliński, 1995, p. 154) [Original Polish: ``fizyka jest zwierciadłem odbijającym świat. Około stu lat temu to było zwierciadło może nie najdoskonalsze, trochę mętne, obraz świata nie był najwyraźniejszy. Ale było to jedno zwierciadło i~jeden obraz. Dzisiaj obraz świata, dostarczany przez fizykę jest znacznie dokładniejszy i~ostrzejszy, ale zwierciadło się rozprysło na wiele kawałków, których do siebie nie umiemy dopasować. Tę metaforę można rozszerzyć, w~szczególności na filozofię i~fizykę, a~właściwie na całą kulturę: niestety mamy ciągle rozbite zwierciadło. Dobrze by było, gdyby udało się nie tyle scalić to zwierciadło, ile stworzyć jedno -- nowe''] (Fuliński, 1995, s.~154).} The concept of ``philosophy in science'' also sought to counteract this discrepancy: Heller and Życiński emphasizing in their works the need to break down the walls between science and culture and justifying it in a~manner similar to Fuliński 
%\label{ref:RNDrPgXU8ydST}(Życiński, 1990; Heller, 1998).
\parencites[][]{zycinski_trzy_1990}[][]{heller_czy_1998}.%




Fuliński believed that at the root of the growing antagonism lies, among other things, a~simplified, colloquial image of science that is deeply rooted in culture. According to Fuliński, various areas of misunderstanding exist between the humanities and science, and one of the key ones is the dispute over evaluating the reductionist method. The issue of reductionism in physics appears in many of Fuliński's works. He already devoted attention to this issue in his opening article for the first seminar at Castel Gandolfo, where he suggested that the reductionist attitude specific to science is sometimes treated by humanists with a~great deal of suspicion, but how does Fuliński himself respond to this type of allegations?



For the Krakow physicist, reductionism is ``an attempt to reduce the world of physics to the basic laws of nature and, if possible, to one basic law of nature'' [Original Polish: ``usiłowanie sprowadzenia świata fizyki do podstawowych praw przyrody, o~ile możności do jednego podstawowego prawa przyrody.''] 
%\label{ref:RNDTvCr7z4QVa}(Fuliński, 1990b, p.187).
\parencite[][p.187]{heller_watpliwosci_1990}. %
 Nevertheless, according to Fuliński, the proposed concept of reductionism is significantly different from the reductionisms of the past that were grounded in mechanistic or scientistic philosophies 
%\label{ref:RNDNBMmkjevPb}(Fuliński, 1993; 2003).
\parencites[][]{fulinski_o_1993}[][]{heller_jednosc_2003-2}. %
 Fuliński is aware that the understanding of reductionism he proposes expresses not just a~specific methodology, but ``it is sometimes actually the adoption of a~certain ontology, the belief that there is a~some unifying principle, some central order of things and phenomena'' [Original Polish: ``jest to nieraz rzeczywiście przyjęcie pewnej ontologii -- wiary w~istnienie jakiejś jednoczącej zasady, jakiegoś centralnego porządku rzeczy i~zjawisk.''] 
%\label{ref:RND3GQv3teBnm}(Fuliński, 1990a, p.36).
\parencite[][p.36]{janik_czesc_1990}. %
 Nevertheless, the reductionism of physics, as Fuliński puts it, does not mean the belief that everything can be reduced to one simple ``world-machine'' model that explains all phenomena. According to Fuliński, reductionism understood like this would be a~real threat to philosophy:



I~see the dangers of today's reflection on the world not in reducing, for example, biology to chemistry or physics, emphasizing the role of chance in evolution, or such like. The pitfalls today lie in the fact that the tendency to think in simple models is strongly established among very wide circles of thinking people: the struggle for existence, the selfish gene, the class struggle, agent activity, and so on. The class of such simplifications also includes viewing the world in terms of purpose, causality, blind fate, or historical necessity. The danger is that belief in simple models leads to belief in simple recipes for understanding the world, taming it, and even worse, repairing all its sins and imperfections. [Original Polish: ``Nie w~sprowadzaniu np. biologii do chemii czy fizyki, podkreślaniu roli przypadku w~ewolucji itp. widzę niebezpieczeństwa grożące dzisiejszej refleksji nad światem. Pułapki leżą dziś raczej w~tym, że silnie ugruntowane wśród bardzo szerokich kręgów myślących ludzi są tendencje do myślenia prostymi modelami: walki o~byt, samolubnego genu, walki klas, rynku, działalności agenturalnej etc. Do klasy takich uproszczeń należy też ujmowanie świata w~kategoriach celowości, przyczynowości, ślepego losu, konieczności historycznej. Niebezpieczeństwo zaś polega na tym, że wiara w~proste modele prowadzi do wiary w~proste recepty na zrozumienie świata, ujarzmienie go i~-- co gorsza -- na naprawę wszelkich jego grzechów i~niedoskonałości.''] 
%\label{ref:RNDdqx2x1IIAP}(Fuliński, 1989, p.230).
\parencite[][p.230]{fulinski_maszyna_1989}.%




In his papers, Fuliński suggested distinguishing between methodological reductionism and the ontological version of reductionism, but he also defined the relationship between them fluently. Drawing attention to the benefits of using reductionist procedures in science, he emphasized that the ontological equivalent of reductionism, as long as it is applied to the scope of the physical world, does not have to necessarily lead to a~monistic, extremely physicalistic metaphysics. According to Fuliński, stating that the properties of increasingly higher levels of the world are reducible to some basic law is not the same as asserting that it is possible to model the entirety of reality according to one pattern and based on one language.



At one point, Fuliński even wrote that ``there is no contradiction between the reductionism of physics, the search for a~unified description of the natural world, and the existence of a~transcending world of freedom, the products of which are not fully determined by the laws of nature, with them containing an element of human creation'' [Original Polish: ``nie ma sprzeczności między redukcjonizmem fizyki, poszukującym zunifikowanego opisu świata przyrody, a~istnieniem transcendującego świata wolności, którego wytwory nie są w~pełni określone przez prawa przyrody, zawierając element ludzkiej kreacji.''] 
%\label{ref:RNDrVjcibsBhO}(Fuliński, 1993, p.47).
\parencite[][p.47]{fulinski_o_1993}. %
 This suggests that Fuliński applied the reductionist theory to the world of physical objects (i.e., the equivalent of Popper's World 1), with him excluding the sphere of the human mind and the results of its activity, such as the issue of self-awareness, the problem of free will\footnote{Particularly interesting in this respect are Fuliński's analyses about the problem of the ``determinism of physics and human free will'' 
%\label{ref:RNDmr9rzvTkgD}(see e.g., Fuliński, 1998; 2005),
\parencites[see e.g][]{janik_fizyka_1998}[][]{wojtowicz_determinizm_2005}, %
 which also demonstrate Fuliński's competence in the area of the traditional problems of philosophical anthropology.}, the issue of values, and so on. This approach to the problem was not so distant from the methodological and ontological views expressed in the OBI community, such as what can be seen, for example, in the works of Życiński related to the concept of emergence 
%\label{ref:RNDa3aIqGQ0Yf}(e.g., Życiński, 2009).
\parencite[e.g][]{zycinski_wszechswiat_2009}.%




Another issue to which Fuliński devoted considerable attention in his philosophical works is the problem of the mathematical nature of the world. The question of ``Is the world mathematical?'' was one of the most important and frequently discussed issues by Heller and Życiński. Many representatives of the OBI formulated an affirmative answer to this question, and their views often moved towards mathematical Platonism (\textcolor{black}{the subject of mathematics research is not a~product of the mind but refers to a~reality that exists independently of cognitive entities}). Fuliński was slightly more cautious in this context 
%\label{ref:RNDP6HPH2C48j}(e.g., Fuliński, 1990c),
\parencite[e.g][]{janik_glos_1990}, %
 with him clearly not taking sides in the philosophical dispute.



Firstly, it was obvious to Fuliński that nature exhibits important features of ordering, so we can model it mathematically, but he also believed that the fact that the world can be described mathematically does not mean that reality is mathematical in the ontological sense 
%\label{ref:RNDvL9xY4wIDj}(Fuliński, 1988b; 1990c; 1993).
\parencites[][]{janik_glos_1988}[][]{janik_glos_1990}[][]{fulinski_o_1993}. %
 Thus, he postulated that the relations between the description of the world (i.e., physical theory) and the world itself should be captured in a~broader context, with this also taking into account other solutions.



When confronted with the question of whether a~scientific theory discovers objectively existing laws or just constructs a~description of the world, Fuliński answered that the problem was apparent and that the two claims should not be considered to be contradictory. A~scientific theory can be a~reflection of reality as well as its reconstruction, structuring, and even a~kind of ``creation.'' A~good illustration of this view is given in his following words:



[…] the statements that theoretical physics discovers objectively existing laws, or that theoretical physics constitutes the description of the world, are probably not contradictory. Like a~work of art, like an artistic creation, theoretical physics is both a~reconstruction (in a~different order) of the world and, to some extent, the creation of this world, except that when we talk about art, we tend to emphasize, to some extent, the moment of creation, but when we talk about physics, we tend to emphasize the moment of mapping'' [Original Polish: ``stwierdzenia, iż fizyka teoretyczna odkrywa istniejące obiektywnie prawa, bądź że fizyka teoretyczna konstytuuje opis świata, zapewne nie są ze sobą sprzeczne. Podobnie jak dzieło sztuki, jak twórczość artystyczna, fizyka teoretyczna to i~rekonstrukcja (w odmiennym porządku) świata i~do pewnego stopnia kreacja tego świata; z~tym, że gdy mówimy o~sztuce, jesteśmy skłonni eksponować do pewnego stopnia moment kreacji; gdy mówimy o~fizyce, jesteśmy skłonni eksponować moment odwzorowania''] 
%\label{ref:RNDd5cjNpx3Yv}(Fuliński, 1988a, p.221).
\parencite[][p.221]{janik_o_1988}.%




Fuliński therefore distanced himself from the question of whether mathematics is a~kind of ontology of the world, as has been assumed, for example, by Życiński 
%\label{ref:RNDSRQmFuMJ9p}(2013).
\parencite*[][]{zycinski_swiat_2013}. %
 Although he did not question this possibility, he demanded greater caution when examining this dispute, pointing to, among other things, the linguistic difficulties that philosophers and scientists encounter here. He pointed to terminological ambiguities that appear in the context of the dispute, as well as to the fact that ``the problem of the primary or secondary nature of language in relation to perception is directly related to the understanding of the mathematical nature of the world and the ontological status of theoretical physics'' (1988a, p. 65; see also Fuliński, 1991, p.. 81) [Original Polish: ``problem pierwotności lub wtórności języka wobec postrzegania jest bezpośrednio związany z~rozumieniem matematyczności świata i~ze statusem ontologicznym fizyki teoretycznej'' 
%\label{ref:RNDnuWvmIzO5D}(Fuliński, 1988b, p.65; see also 1991a, p.81)
\parencites[][p.65]{janik_glos_1988}[see also 1991p.8][]{} %
 and how these make the metaphysical question about the nature of reality require very subtle analyses and caution when formulating an answer.



It is worth emphasizing, however, that Fuliński's analyses in the context of the problem of the mathematical nature of the world were positively received in the OBI community 
%\label{ref:RNDYfzhDI9neu}(e.g., Życiński statement in discussion in: Fuliński, 1988a, pp.217–218).
\parencite[e.g][pp.217–218]{zycinski_teizm_1988}. %
 On analyzing the works of other representatives of the Krakow interdisciplinary community, it can be discerned that they took Fuliński's critical remarks into account. Such critical positions, which also came from other authors, could consequently influence a~more nuanced attitude to the idea of the mathematical universe, and this is already noticeable in the works of the younger generation of philosophers from Heller's milieu, such as Ł. Lamża and M. Hohol.



\section{An Attempt to Summarize}

During his scientific career, Fuliński became well known not just as a~physicist but also as a~scholar who was sensitive to philosophical issues. For many years, he has been involved in the dialogue between science and philosophy and participated in various interdisciplinary projects, with him publishing a~number of works primarily in the area of the philosophy of nature and the methodology of science.



Fuliński's publications clearly bear the mark of ``philosophy in science''. In his texts, the Krakow physicist has addressed issues that fit into the project of philosophy that was outlined by Heller 
%\label{ref:RNDUfGvSsqxoX}(1986, English translation: 2019).
\parencite[][]{}. %
 In his programmatic paper, Heller indicated that the subjects of interest for philosophy in science include (A) the influence of philosophical ideas on the development and evolution of scientific theories; (B) traditional philosophical problems that are entangled in empirical theories; and (C) philosophical reflections on the assumptions of empirical science. The issues discussed by Fuliński correspond to each of the three areas of ``philosophy in science'', e.g.:



(A): methodological analyses of science–culture relations, including issues of interaction; this group could also include, among other things, works on the history of science and philosophy, devoted, for example, to the achievements of the ``philosophical physicist'' Marian Smoluchowski 
%\label{ref:RND20RZHSduO4}(e.g., Fuliński, 2017);
\parencite[e.g][]{fulinski_fluktuujacy_2017};%




(B): problems of time, determinism, the question of chance, and so on 
%\label{ref:RNDM0N3i4NFdN}(e.g. Fuliński, 1993; 2015);
\parencites[e.g.][]{fulinski_o_1993}[][]{fulinski_profesor_2015};%




(C): the question of the mathematicality of the world and the problem of the elementarity and unity of nature (including the issue of reductionism).



It is noteworthy that Fuliński's approach to analyzing philosophical problems also turned out to be close to the style of Heller. The works of the Krakow physicist show that he rejected the radical isolationism of science and philosophy, and he was also very critical of systemic philosophical concepts like Thomism. He placed his reflections within a~scientific context while remaining open to traditional metaphysical problems.\footnote{However, on various occasions, Fuliński himself has expressed a~distanced attitude toward philosophy as such and philosophers in particular. This is well illustrated by a~statement from a~discussion panel during a~symposium organized by the OBI in 1995: ``What do physics and philosophy offer? First, the results of physics and philosophy are sometimes put into practice. The implementation of certain philosophical concepts has brought a~lot of harm, which we experienced first-hand. Everyone knows how much harm is associated with the implementation of some results of physics. I~wouldn't be able to judge which of these effects were worse. What good do physics and philosophy do? Physics certainly gives various good things: the light in this room, the flash just now, and so on. What good things philosophy has brought I~prefer to leave to philosophers to judge. What do physics and philosophy give to each other? First, what does physics give to philosophy? Theoretically, it should give a~lot; at least many physicists believe that physics, especially theoretical physics practiced at a~sufficiently deep level, is actually philosophy. In practice, I'm afraid it doesn't help much, because the typical response of a~philosopher to a~physicist's arguments is at best ‘Yes, but...' or at worst ‘The physicist is being smart again.' What does philosophy directly contribute to physics? Philosophers think there must be a~lot. Physicists know from practice that it is nothing. It is better not to talk about examples of adopting philosophical concepts into science, such as Lysenko's methods. More seriously, philosophy gives some things, but not so much to physics but rather to the physicist, not least because it broadens the imagination. But the whole culture works in the same way as philosophy, like poetry, music, fantasy. To put it maliciously, many physicists have directly benefited more from science fiction than from philosophy'' [Original Polish: ``Co fizyka i~filozofia dają? Po pierwsze, wyniki fizyki i~filozofii bywają wcielane w~życie. Realizacja pewnych koncepcji filozoficznych przyniosła dużo złego, co przeżyliśmy na własnej skórze. Każdy wie, jak wiele złego wiąże się z~realizacją niektórych rezultatów fizyki. Nie umiałbym ocenić, które z~tych skutków były gorsze. Co fizyka i~filozofia dają dobrego? Fizyka na pewno różne dobre rzeczy daje: światło na tej sali, błysk flesza przed chwilą itd. Co dobrego dała filozofia, to wolę pozostawić do oceny filozofom. Co fizyka i~filozofia dają sobie wzajemnie? Najpierw, co daje fizyka filozofii? Teoretycznie powinna dawać dużo; przynajmniej wielu fizyków sądzi, że fizyka -- zwłaszcza teoretyczna, uprawiana na dostatecznie głębokim poziomie -- to jest już właściwie filozofia. W~praktyce obawiam się, że daje niewiele, ponieważ typową odpowiedzią filozofa na wywody fizyka jest w~najlepszym wypadku: «Tak, ale...»; a~w gorszym wypadku: «Znowu fizyk się wymądrza». Co bezpośrednio daje filozofia fizyce? Filozofowie sądzą, że na pewno dużo. Fizycy z~praktyki wiedzą, że -- nic. O~przykładach przejmowania koncepcji filozoficznych do nauki -- takich np., jak metody Łysenki -- lepiej nie mówić. Mówiąc bardziej serio, filozofia pewne rzeczy daje, ale nie tyle fizyce, ile fizykowi, m.in. dlatego, że poszerza wyobraźnię. Ale tak samo jak filozofia działa w~ogóle cała kultura: także poezja, muzyka, fantastyka. Mówiąc złośliwie, wielu fizyków bezpośrednio więcej skorzystało z~literatury fantastyczno-naukowej niż z~filozofii''] 
%\label{ref:RNDZjVasxyB1M}(Fuliński et al., 1995, p.147).
\parencite[][p.147]{fulinski_glos_1995}.%
} This was appreciated by some representatives of the School, such as Życiński, who willingly referred to Fuliński's publications (see footnote 1).



Significantly, the activities of the Krakow physicist fell into, among other things, the early formative period for the concept of ``philosophy in science'' and the milieu of Heller and Życiński 
%\label{ref:RNDTxX1PpPAZm}(Trombik, 2021).
\parencite[][]{trombik_koncepcje_2021}. %
 It is therefore possible to speculate that Fuliński was not just part of the Krakow School of Philosophy in Science current but also a~creative influence within this school, both philosophically and organizationally, having participated in various interdisciplinary undertakings. I~think this thread should be developed and deepened in a~future, larger dissertation that would more comprehensively study the life and work of Fuliński.



Thinking about the research perspectives related to the School's activities, I~believe that it would be worth undertaking detailed research to indicate the possible scope of the impact on Heller and Życiński's milieu from other philosophizing scientists, such as Jerzy Janik, Andrzej Staruszkiewicz, Zygmunt Chyliński, Małgorzata Głódź, Jerzy Rayski, Leszek Sokołowski, Alicja Michalik, or Marek Szydłowski.\footnote{It is worth adding that some of the mentioned representatives of philosophizing scientists were very sympathetic to the idea of ,,philosophy in science'' 
%\label{ref:RNDqb5UQaFfug}(e.g., Głódź, 1999).
\parencite[e.g][]{glodz_zfwn_1999}.%
} Such research would not only enrich our knowledge about the historical development of the School but could also bring closer some interesting and often still-current philosophical views that are part of native interdisciplinary traditions.



\section{References}

Fuliński, A., 1981. Fizyka w~kontekście kultury. In: J. Janik, ed. \textit{Nauka, religia, dzieje: seminarium w~Castel Gandolfo 16-19 sierpnia 1980 roku.} Rome: s.n. pp.13–33.



Fuliński, A., 1988a. [Głos w~dyskusji] Dyskusja po referacie A. Fulińskiego ,,O matematyczności przyrody''. In: J.A. Janik and P. Lenartowicz, eds. \textit{Nauka - religia - dzieje: IV Seminarium interdyscyplinarne w~Castel Gandolfo, 6-8 sierpnia 1986}, Teksty i~Studia / Wydział Filozoficzny Towarzystwa Jezusowego w~Krakowie; nr 19. Kraków: Wydział Filozoficzny Towarzystwa Jezusowego. pp.213–222.



Fuliński, A., 1988b. O~matematyczności świata. In: J.A. Janik and P. Lenartowicz, eds. \textit{Nauka - religia - dzieje: IV Seminarium interdyscyplinarne w~Castel Gandolfo, 6-8 sierpnia 1986}, Teksty i~Studia / Wydział Filozoficzny Towarzystwa Jezusowego w~Krakowie; nr 19. Kraków: Wydział Filozoficzny Towarzystwa Jezusowego. pp.51–72.



Fuliński, A., 1989. Maszyna czy myśl? [recenzja: M. Heller, J. Życiński, Filozofia mechanicyzmu, Kraków 1988]. \textit{Studia Philosophiae Christianae}, [online] 25, pp.226–230. Available at: {\textless}https://bazhum.muzhp.pl/media//files/Studia\_Philosophiae\_Christianae/Studia\_Philosophiae\_Christianae-r1989-t25-n2/Studia\_Philosophiae\_Christianae-r1989-t25-n2-s226-230/Studia\_Philosophiae\_Christianae-r1989-t25-n2-s226-230.pdf{\textgreater} [Accessed 13 November 2023].



Fuliński, A., 1990a. Część i~całość (równoważne sposoby opisu i~redukcjonizm -- w~oczach fizyka). In: J.A. Janik and P. Lenartowicz, eds. \textit{Nauka - religia - dzieje: V~Seminarium Interdyscyplinarne w~Castel Gandolfo, 8-11 sierpnia 1988}, Teksty i~Studia / Wydział Filozoficzny Towarzystwa Jezusowego w~Krakowie, 24. Kraków: Wydział Filozoficzny Towarzystwa Jezusowego. pp.181–191.



Fuliński, A., 1990b. [Głos w~dyskusji] Dyskusja po referatach W. Kołosa, A. Fulińskiego i~J. Janika. In: J.A. Janik and P. Lenartowicz, eds. \textit{Nauka - religia - dzieje: V~Seminarium Interdyscyplinarne w~Castel Gandolfo, 8-11 sierpnia 1988}, Teksty i~Studia / Wydział Filozoficzny Towarzystwa Jezusowego w~Krakowie, 24. K~/ Abp Józef Życiński; Katolicki Uniwersytet Lubelski.raków: Wydział Filozoficzny Towarzystwa Jezusowego. pp.181–191.



Fuliński, A., 1990c. Wątpliwości fizyka. In: M. Heller and J. Życiński, eds. \textit{Matematyczność przyrody}. Kraków: Ośrodek Badań Interdyscyplinarnych przy Wydziale Filozofii Papieskiej Akademii Teologicznej. pp.72–75.



Fuliński, A., 1991a. Co jak istnieje? Z~punktu widzenia fizyka. In: M. Heller, W.) Skoczny and J. Życiński, eds. \textit{Spór o~uniwersalia a~nauka współczesna: sympozjum, Kraków 11-12 maja 1990 / pod red. Michała Hellera, Włodzimierza Skocznego i~Józefa Życińskiego.} Kraków: Ośrodek Badań Interdyscyplinarnych przy Wydziale Filozofii Papieskiej Akademii Teologicznej. pp.81–85.



Fuliński, A., 1991b. [Głos w~dyskusji] Między wiedzieć i~wierzyć. \textit{Znak}, (428 (1)), pp.29–42.



Fuliński, A., 1993. O~chaosie i~przypadku, a~także o~determinizmie, redukcjonizmie i~innych grzechach fizyków czyli o~zmianach w~obrazie świata widzianych okiem jednego z~nich. \textit{Znak}, XLV(456(5)), pp.31–49.



Fuliński, A., 1996. O~czasie i~świadomości. In: S. Wszołek, ed. \textit{Przestrzenie księdza Cogito: księdzu Michałowi Hellerowi w~sześćdziesiątą rocznicę urodzin}. Tarnów: Biblos. pp.58–63.



Fuliński, A., 1998. Fizyka a~wolny wybór. In: J.A. Janik, ed. \textit{Nauka - religia - dzieje: IX Seminarium w~Castel Gandolfo, 5-7 sierpnia 1997}, Varia / Uniwersytet Jagielloński. Kraków: Wydawnictwo Uniwersytetu Jagiellońskiego. pp.45–56.



Fuliński, A., 2003. Jedność czy redukcjonizm, czyli o~zakusach fizyki wobec innych nauk. In: M. Heller and J. Mączka, eds. \textit{Jedność nauki - jedność świata?} Tarnów: Wydawnictwo Diecezji Tarnowskiej Biblos. pp.63–68.



Fuliński, A., 2005. Determinizmy fizyki vs. wolna wola człowieka. \textit{Nauka}, (1), pp.67–74.



Fuliński, A., 2010. Czy istnieje przypadek?: skąd się biorą tzw. zjawiska losowe? \textit{Prace Komisji Filozofii Nauk Przyrodniczych}, 4, pp.117–140.



Fuliński, A., 2015. Słabe łamanie ergodyczności vs. determinizm [Weak ergodicity breaking vs. determinism of physical processes]. \textit{Philosophical Problems in Science (Zagadnienia Filozoficzne w~Nauce)}, [online] (59), pp.83–100. Available at: {\textless}https://zfn.edu.pl/index.php/zfn/article/view/182{\textgreater} [Accessed 11 November 2023].



Fuliński, A., 2017. Fluktuujący świat Mariana Smoluchowskiego. \textit{Philosophical Problems in Science (Zagadnienia Filozoficzne w~Nauce)}, [online] (62), pp.127–138. Available at: {\textless}https://zfn.edu.pl/index.php/zfn/article/view/393{\textgreater} [Accessed 11 November 2023].



Fuliński, A., Heller, M., Kałuszyńska, E., Kijowski, J. and Staruszkiewicz, A., 1995. [Głos w~dyskusji] Racjonalność-Falsyfikowalność-Kosmologia. \textit{Filozofia Nauki}, [online] 3(1–2), pp.143–182. Available at: {\textless}https://www.fn.uw.edu.pl/index.php/fn/article/view/956{\textgreater} [Accessed 13 November 2023].



Fuliński, A. and Maślanka, K.D. eds., 2015. \textit{Profesor Jerzy A. Janik 1927-2012: uczony - myśliciel - mistrz: materiały z~sesji w~dniu 12 kwietnia 2013, poświęconej pamięci Profesora w~pierwszą rocznicę Jego śmierci}. Kraków: Polska Akademia Umiejętności.



Głódź, M., 1999. ZFwN a~OBI: dwom panom służyć. \textit{Zagadnienia Filozoficzne w~Nauce}, (25), pp.16–19.



Heller, M., 1986. Jak możliwa jest ‘filozofia w~nauce'? \textit{Studia Philosophiae Christianae}, 22(1), pp.7–19.



Heller, M., 1990. Nowa fizyka -- perspektywy trwającej rewolucji. In: J.A. Janik and P. Lenartowicz, eds. \textit{Nauka - religia - dzieje: V~Seminarium Interdyscyplinarne w~Castel Gandolfo, 8-11 sierpnia 1988}, Teksty i~Studia / Wydział Filozoficzny Towarzystwa Jezusowego w~Krakowie, 24. Kraków: Wydział Filozoficzny Towarzystwa Jezusowego. pp.66–83.



Heller, M., 1998. \textit{Czy fizyka jest nauką humanistyczną?} Tarnów: Biblos.



Heller, M., 2004. \textit{Filozofia przyrody: zarys historyczny}. Kompendia Filozoficzne. \textit{Filozofia przyrody[202F?]: zarys historyczny}. Kraków: Znak.



Heller, M., 2011. \textit{Philosophy in Science}. [online] Berlin, Heidelberg: Springer Berlin Heidelberg. https://doi.org/10.1007/978-3-642-17705-7.



Heller, M., 2016. \textit{Wierzę, żeby rozumieć: w~osobistej rozmowie o~życiowych wyborach}. Kraków: Wydawnictwo Znak.



Heller, M., 2019. How is philosophy in science possible? \textit{Philosophical Problems in Science (Zagadnienia Filozoficzne w~Nauce)}, [online] (66), pp.231–249. Available at: {\textless}https://zfn.edu.pl/index.php/zfn/article/view/482{\textgreater} [Accessed 6 October 2021].



Heller, M., 2020. Jan Paweł II i~polskie przemiany. \textit{PAUza Akademicka}, [online] (514), p.9. Available at: {\textless}http://www.pauza.krakow.pl/514\_2020.pdf{\textgreater} [Accessed 13 November 2023].



Heller, M. and Mączka, J., 2006. Początki filozofii przyrody w~Ośrodku Badań Interdyscyplinarnych w~Krakowie (The Beginnings of the Center for Interdisciplinary Studies in Cracow). \textit{Roczniki Filozoficzne}, [online] 54(2), pp.49–62. Available at: {\textless}http://www.jstor.org/stable/43409838{\textgreater} [Accessed 14 September 2016].



Heller, M., Mączka, J., Polak, P. and Szczerbińska-Polak, M. eds., 2007. \textit{Krakowska filozofia przyrody w~okresie międzywojennym. Tom I: Początki}. Kraków-Tarnów: OBI-Biblos.



Janik, J., 1981. Nauka -- religia -- dzieje. In: J. Janik, ed. \textit{Nauka, religia, dzieje: seminarium w~Castel Gandolfo 16-19 sierpnia 1980 roku.} Rome: s.n. pp.5–12.



Liana, Z. and Mączka, J., 1999. Z~kroniki OBI. \textit{Philosophical Problems in Science (Zagadnienia Filozoficzne w~Nauce)}, (25), pp.133–152.



Nowina Konopka, M., 2020. Kontakty Jana Pawła II z~fizykami. \textit{Postępy Techniki Jądrowej}, 63(3), pp.28–34.



Polak, P., 2011a. 19th Century Beginnings of the Kraków Philosophy of Nature. In: B. Brożek, J. Mączka and W.P. Grygiel, eds. \textit{Philosophy in Science. Methods and Applications}. Kraków: Copernicus Center Press. pp.325–333.



Polak, P., 2011b. U~źródeł krakowskiej filozofii przyrody. \textit{Studia z~Filozofii Polskiej}, [online] 6, pp.135–153. Available at: {\textless}https://www.researchgate.net/publication/262363588\_U\_zrodel\_krakowskiej\_filozofii\_przyrody{\textgreater}.



Polak, P., 2018. Tradycja krakowskiej filozofii w~nauce: między XIX a~XXI wiekiem (Tradition of Krakow philosophy in science: since 19th to 21st century). In: J. Jagiełło, ed. \textit{40 lat filozofii w~uczelni papieskiej w~Krakowie}. Kraków: Wydawnictwo Naukowe UPJPII. pp.491–514.



Polak, P., 2019. Philosophy in science: A~name with a~long intellectual tradition. \textit{Philosophical Problems in Science (Zagadnienia Filozoficzne w~Nauce)}, [online] (66), pp.251–270. Available at: {\textless}https://zfn.edu.pl/index.php/zfn/article/view/472{\textgreater} [Accessed 6 October 2021].



Polak, P. and Rodzeń, J., 2021. The science-religion relationship in the academic debate in Poland, 1945-1998. \textit{European Journal of Science and Theology}, [online] 17(6), pp.1–17. Available at: {\textless}http://www.ejst.tuiasi.ro/Files/91/1\_Polak\%20\&\%20Rodzen.pdf{\textgreater} [Accessed 20 December 2022].



Polak, P. and Rodzeń, J., 2023. The Theory of Relativity and Theology: The Neo-Thomist Science–Theology Separation vs. Michael Heller's Path to Dialogue. \textit{Theology and Science}, [online] 21(1), pp.157–174. https://doi.org/10.1080/14746700.2022.2155917.



Polak, P. and Trombik, K., 2022. The Kraków School of Philosophy in Science: Profiting from Two Traditions. \textit{Edukacja Filozoficzna}, (2(74)), pp.205–229. https://doi.org/10.14394/edufil.2022.0023.



Snow, C.P., 1959. Two Cultures. \textit{Science}, [online] 130(3373), pp.419–419. Available at: {\textless}https://www.jstor.org/stable/1758035{\textgreater} [Accessed 27 January 2024].



Trombik, K., 2021. \textit{Koncepcje filozofii przyrody w~Papieskiej Akademii Teologicznej w~Krakowie w~latach 1978-1993: studium historyczno-filozoficzne}. Kraków: Wydawnictwo ‘scriptum'.



Trombik, K., 2022. Stworzyć płaszczyznę wolności myśli. Wkład Karola Wojtyły w~powstanie Wydziału Filozoficznego Papieskiej Akademii Teologicznej w~Krakowie. \textit{Ethos Kwartalnik Instytutu Jana Pawła II KUL}, [online] (35(2022) 1(137)), pp.255–276. https://doi.org/10.12887/35-2022-1-137-15.



Trombik, K. and Polak, P., 2022. Teologia nauki -- propozycja nowego otwarcia teologii na nauki. \textit{Człowiek i~Społeczeństwo}, [online] 54, pp.49–64. https://doi.org/10.14746/cis.2022.54.4.



Trombik, K.P., 2019. The origin and development of the Center for Interdisciplinary Studies. A~historical outline by 1993. \textit{Philosophical Problems in Science (Zagadnienia Filozoficzne w~Nauce)}, [online] (66), pp.271–295. Available at: {\textless}https://zfn.edu.pl/index.php/zfn/article/view/474{\textgreater} [Accessed 23 September 2019].



Życiński, J., 1990. \textit{Trzy kultury: nauki przyrodnicze, humanistyka i~myśl chrześcijańska / Józef Życiński.} Poznań: W~Drodze.



Życiński, J., 2000. \textit{Inspiracje chrześcijańskie w~powstaniu nauki nowożytnej}. 1st ed. Lublin: Redakcja Wydawnictw Katolickiego Uniwersytetu Lubelskiego.



Życiński, J., 2009. \textit{Wszechświat emergentny: Bóg w~ewolucji przyrody}. Filozofia Przyrody i~Nauk Przyrodniczych. Lublin: Wydawnictwo KUL.



Życiński, J., 2013. \textit{Świat matematyki i~jej materialnych cieni}. 2nd ed. Kraków: Copernicus Center Press.

\end{document}

