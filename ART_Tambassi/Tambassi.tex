\begin{artengenv}{Timothy Tambassi}
	{For the sake of simplicity: Applying software design parsimony to the content of information system ontologies}
	{For the sake of simplicity\ldots}
	{For the sake of simplicity: Applying software design parsimony to the content of information system ontologies}
	{Ca' Foscari University of Venice}
	{Although many information system ontologies [ISOs] claim to be parsimonious, the notion of parsimony\textcolor{red}{ }seems to influence the debate on ISOs only at the level of vague and uncritical assumption. To challenge this trend, the paper aims to clarify what it means for ISOs to be parsimonious. Specifically, Sect. 2 shows that parsimony in computer science generally concerns software design and, together with elegance, is one of the two aspects of the broader notion of simplicity. Sect. 3 transforms the main claims of parsimony in software design into claims about the content of ISOs, the combination of which is hereafter called ``parsimony of content''---where ``content'' refers only to the content of ISOs. Sects. 4-7 discuss the application of this parsimony to the design of ISOs, and outline different kinds (and combinations) of parsimony of content. Finally, section 8 considers whether parsimony of content could provide some criteria both for selecting and/or classifying the contents of ISOs and for choosing between different and equally consistent ISOs.
	}
	{information system ontologies, ontological aims, parsimony, representation primitives, simplicity.}

\begin{customepigraph}
\textit{There are two ways of constructing a~software design:
one way is to make it so simple that there are obviously no deficiencies,
and the other way is to make it so complicated that there are no obvious deficiencies.
} \\
Tony Hoare (1980)
\end{customepigraph}


\newcounter{saveenumtambassi}

\section{Introduction}

According to Turner 
%\label{ref:RNDV1PAblDjhO}(2018),
\parencite*[][]{turner_computational_2018}, %
 there are two methodological advantages to adopting parsimony in software design:



\begin{itemize}

\item diminishing the amount of work,

\item reducing the risk of error.

\end{itemize}

«This is in line with Quine, who, in the case of theories, argues that parsimony carries with it pragmatic advantages, and that pragmatic considerations themselves provide rational grounds for discriminating between competing theories» 
%\label{ref:RNDkLhu0OCR7q}(Turner, 2018, p.139).
\parencite[][p.139]{turner_computational_2018}.%




Acknowledging such advantages, however, does not imply that the adoption of parsimony is mandatory. Indeed, in speaking of information system ontologies [ISOs], Smith 
%\label{ref:RNDXcl4PsYojm}(2004)
\parencite*[][]{floridi_ontology_2004} %
 and Grenon 
%\label{ref:RNDukQSlXBJb6}(2008)
\parencite*[][]{munn_primer_2008} %
 remark that nothing prevents ISOs from:



\begin{enumerate}[label={[\arabic*]}]

\item endorsing/rejecting different assumptions,

\item including parsimony among those assumptions,

\item considering the possibility of multiple forms of parsimony, and then repeating [1–2].

\end{enumerate}

Despite [1–3], the adoption of parsimony is so common for ISOs that many ISOs implicitly and uncritically assume this notion. To prevent parsimony from influencing the debate on ISOs at the level of an implicit and uncritical assumption, this paper aims to clarify what it means for ISOs to be parsimonious. Sect. 2 shows that parsimony in computer science generally concerns software design and, together with elegance, is one of the two aspects of the broader notion of simplicity. Sect. 3 transforms the main claims of parsimony in software design into two claims about the contents of ISOs, the combination of which is hereafter called ``parsimony of content''---where ``contents'' refers only to the contents of ISOs. Sects. 4–7 discuss the application of this parsimony to the design of ISOs, and outline different kinds (and combinations) of parsimony of content. Finally, Sect. 8 considers whether parsimony of content could provide some criteria both for selecting and/or classifying the contents of ISOs and for choosing between different and equally consistent ISOs.



\section{Parsimony in software design}

One of the main reasons why computer scientists place simplicity at the core of good and/or successful software design\footnote{On software design, see Allen 
%\label{ref:RNDu7LGVjHRhr}(1997);
\parencite*[][]{allen_formal_1997}; %
 Baljon 
%\label{ref:RNDjD9j6SQGDR}(2002);
\parencite*[][]{baljon_history_2002}; %
 Parsons 
%\label{ref:RNDjLVRqmMj5I}(2015).
\parencite*[][]{parsons_philosophy_2015}.%
} is that simplicity contributes to the transparency and reliability of the design.\footnote{On simplicity in software design, see also Wirth 
%\label{ref:RND5FbOktAkkg}(1974);
\parencite*[][]{wirth_design_1974}; %
 Dijkstra 
%\label{ref:RNDcl3gE1ch0O}(1979).
\parencite*[][]{yourdon_humble_1979}.%
} According to Turner 
%\label{ref:RNDQC3jZenQTm}(2018, pp.133–134),
\parencite*[][pp.133–134]{turner_computational_2018}, %
 simplicity does not have a~single meaning in this context; rather, it refers to two distinct and related notions: elegance (or syntactic simplicity) and parsimony (or ontological simplicity).\footnote{See also Baker 
%\label{ref:RNDykvx6jKfKL}(2016),
\parencite*[][]{baker_simplicity_2016}, %
 who analyzes the distinction between elegance and parsimony within the philosophy of science debate. }



Elegance generally concerns the graspability, clarity, transparency, correctness, efficiency, consistency, generality, uniformity, and explanatory power of software.\footnote{On elegance in software design, see Bentley and McIroy 
%\label{ref:RNDjjKxjSNemy}(1993);
\parencite*[][]{bentley_engineering_1993}; %
 Gelernter 
%\label{ref:RND7590bTYUus}(1998);
\parencite*[][]{gelernter_machine_1998}; %
 Oram and Wilson 
%\label{ref:RNDNOqSwpLxL7}(2007);
\parencite*[][]{oram_beautiful_2007}; %
 Hill 
%\label{ref:RNDzvJBzPQkXH}(2018);
\parencite*[][]{de_mol_elegance_2018}; %
 Turner 
%\label{ref:RNDwReEAcJrjf}(2018).
\parencite*[][]{turner_computational_2018}.%
} Parsimony links software design with its specification\footnote{One referee rightly pointed out that there are other ways of relating simplicity and parsimony. The example they give is simplicity in understanding the code (i.e. ``semantic simplicity''), including self-commenting code, which is simple in terms of understanding the code. I~fully agree with them. I~can only note here that this paper is not intended to exhaust the debate on the relationship between simplicity and parsimony. For more details on semantic simplicity, see Gelernter 
%\label{ref:RNDgojrYzKn6o}(1998);
\parencite*[][]{gelernter_machine_1998}; %
 Sober 
%\label{ref:RNDjCK4WyMHqZ}(2002);
\parencite*[][]{sober_what_2002}; %
 Turner 
%\label{ref:RNDowLEJtjqeZ}(2018).
\parencite*[][]{turner_computational_2018}.%
}, and insists that



\setcounter{saveenumtambassi}{\value{enumi}}

\begin{enumerate}[label={[\arabic*]}]

\setcounter{enumi}{\value{saveenumtambassi}}

\item software solutions do not go beyond \textit{what is required}.

\end{enumerate}

While Turner further specifies the meaning of ``what is required'' in [4] by claiming that



\setcounter{saveenumtambassi}{\value{enumi}}

\begin{enumerate}[label={[\arabic*]}]

\setcounter{enumi}{\value{saveenumtambassi}}

\item software should solve the problem it aims to solve, but no more,

\end{enumerate}

Pawson 
%\label{ref:RNDXzNTRi5WyF}(1998)
\parencite*[][]{pawson_minimum_1998} %
 takes one step further. First, he considers



\setcounter{saveenumtambassi}{\value{enumi}}

\begin{enumerate}[label={[\arabic*]}]

\setcounter{enumi}{\value{saveenumtambassi}}

\item parsimony to have been achieved when it is no longer possible to improve software by subtraction.

\end{enumerate}

Then, he adds that

\enlargethispage{1.5\baselineskip}

\setcounter{saveenumtambassi}{\value{enumi}}

\begin{enumerate}[label={[\arabic*]}]

\setcounter{enumi}{\value{saveenumtambassi}}

\item parsimony is the quality that software applications have when their components, details, and junctions have been reduced to the essential.

\end{enumerate}

[7] in turn means that



\setcounter{saveenumtambassi}{\value{enumi}}

\begin{enumerate}[label={[\arabic*]}]

\setcounter{enumi}{\value{saveenumtambassi}}

\item the link between the design and the aims of software (see [4–5]) also concerns the components, details, and junctions of the software.

\end{enumerate}

[4–8] (together) imply that



\setcounter{saveenumtambassi}{\value{enumi}}

\begin{enumerate}[label={[\arabic*]}]

\setcounter{enumi}{\value{saveenumtambassi}}

\item parsimony concerns the [9.1] aims of software and [9.2] its components, details, and junctions.

\end{enumerate}

\section{Parsimony in information system ontologies}

Section 2 has shown that:

\enlargethispage{1.5\baselineskip}

\setcounter{saveenumtambassi}{\value{enumi}}

\begin{enumerate}[label={[\arabic*]}]

\setcounter{enumi}{\value{saveenumtambassi}}

\item simplicity is at the core of good and/or successful software design;

\item simplicity can be divided into elegance and parsimony.

\end{enumerate}

Turner (2018, p. 128) adds that



\setcounter{saveenumtambassi}{\value{enumi}}

\begin{enumerate}[label={[\arabic*]}]

\setcounter{enumi}{\value{saveenumtambassi}}

\item design is everywhere in computer science.

\end{enumerate}

This means that, if [10–12] hold, parsimony also applies to the design of ISOs.



Gruber 
%\label{ref:RND9qXQktxp3y}(2009)
\parencite*[][]{liu_ontology_2009} %
 defines ISOs as follows:



\setcounter{saveenumtambassi}{\value{enumi}}

\begin{enumerate}[label={[\arabic*]}]

\setcounter{enumi}{\value{saveenumtambassi}}

\item ISOs are sets of representational primitives (henceforth, primitives) with which to model a~domain (of knowledge).\footnote{For further (and competing) definitions of ISO, see Neches et al. 
%\label{ref:RNDs79Rw3smTh}(1991);
\parencite*[][]{neches_enabling_1991}; %
 Gruber 
%\label{ref:RNDulDWUalQHD}(1993);
\parencite*[][]{gruber_translation_1993}; %
 Guarino and Giaretta 
%\label{ref:RNDNc93EP1veQ}(1995);
\parencite*[][]{guarino_ontologies_1995}; %
 Bernaras et al. 
%\label{ref:RNDyJd8BX2iSu}(1996);
\parencite*[][]{bernaras_building_1996}; %
 Borst 
%\label{ref:RNDFOBsz9JegY}(1997);
\parencite*[][]{borst_construction_1997}; %
 Swartout et al. 
%\label{ref:RNDZWeg7QpVjT}(1997);
\parencite*[][]{swartout_toward_1997}; %
 Studer et al. 
%\label{ref:RNDDSELqzHVKQ}(1998);
\parencite*[][]{studer_knowledge_1998}; %
 Guarino 
%\label{ref:RNDUOi3fhFvwb}(1998);
\parencite*[][]{guarino_formal_1998}; %
 Uschold and Jasper 
%\label{ref:RNDtVnvyrP1Mw}(1999);
\parencite*[][]{uschold_framework_1999}; %
 Sowa 
%\label{ref:RNDkevWbfYmM7}(2005);
\parencite*[][]{sowa_guided_2005}; %
 Noy and McGuinness 
%\label{ref:RNDCZq17toNhF}(2003);
\parencite*[][]{noy_ontology_2003}; %
 Tambassi and Magro 
%\label{ref:RNDrJpx0AqoZM}(2015).
\parencite*[][]{tambassi_ontologie_2015}. %
 Gruber 
%\label{ref:RNDciyDRYg1al}(2009, p.1964)
\parencite*[][p.1964]{liu_ontology_2009} %
 has also affirmed that ISOs, or ``ontologies'', are artefacts specified by (ontological) languages. Before him, Guarino and Giaretta 
%\label{ref:RNDPAWMiYAUY8}(1995)
\parencite*[][]{guarino_ontologies_1995} %
 have pointed out that ``ontology'' in computer science has (at least) two different meanings: the artefact and the philosophical discipline---which finds direct application in computer science (see, for example, Turner 
%\label{ref:RNDEIxMrfnX5h}(2018);
\parencite*[][]{turner_computational_2018}; %
 Krzanowski and Polak 
%\label{ref:RND4FPWHBdLXS}(2022)
\parencite*[][]{krzanowski_meta-ontology_2022}%
). This explains why ``ontology'' can have the same meaning in both philosophy and computer science. } Primitives are primarily instances, classes, properties, and relations.\footnotemark

\end{enumerate}

Therefore, based on [4–9], applying parsimony (of software design) to [13] means that:

\setcounter{saveenumtambassi}{\value{enumi}}

\begin{enumerate}[label={[\arabic*]}]

\setcounter{enumi}{\value{saveenumtambassi}}

\item ISOs should not go beyond the problem(s) they aim to solve (see [4–5] and [9.1])---that is, beyond the domain(s) (of knowledge) ISOs aim to model;

\item the components, details and junctions of ISOs, that is the primitives of ISOs, should be reduced to the essential (see [6–7] and [9.2]).

\end{enumerate}

\footnotetext{Instances are the lowest-level components, the basic units, of ISOs 
%\label{ref:RNDdr97C4EV5N}(Laurini, 2017).
\parencite[][]{laurini_geographic_2017}. %
 Classes, which may contain sub-classes and/or be sub-classes of other classes, are sets of instances that share common features 
%\label{ref:RNDBg41DF7QTa}(Jaziri and Gargouri, 2010).
\parencite[][]{jaziri_ontology_2010}. %
 Properties describe the various features of a~class and of its instances 
%\label{ref:RNDO5jcCKaTQ1}(Noy and McGuinness, 2003; Jaziri and Gargouri, 2010).
\parencites[][]{noy_ontology_2003}[][]{jaziri_ontology_2010}. %
 Relations represent the way in which both classes and instances interact with each other 
%\label{ref:RNDhcWtvdFqLf}(Laurini, 2017).
\parencite[][]{laurini_geographic_2017}. %
 On primitives, see also Tambassi 
%\label{ref:RNDR0Py0GAbHa}(Tambassi, 2021).
\parencite[][]{tambassi_philosophy_2021}.%
}

Henceforth, by ``parsimony of content'' (where ``content'' refers only to the contents of ISOs) I~will mean the application of [4–9] to [13], namely [14–15]. There are two main reasons for this emphasis on ``content''---rather than on ``parsimony of ISOs'' in the broader sense. The first reason is that, within the debate on ISOs, the notion of parsimony is chiefly associated with the content of primitives.\footnote{See Burgun et al. 
%\label{ref:RNDVwq8XaiFhH}(1999);
\parencite*[][]{burgun_sharing_1999}; %
 Yao et al. 
%\label{ref:RNDLNiZPaJvJH}(2011);
\parencite*[][]{yao_benchmarking_2011}; %
 Motara and Van der Schiff 
%\label{ref:RNDOkvKiog8MQ}(2019);
\parencite*[][]{motara_functional_2019}; %
 Partridge et al. 
%\label{ref:RND3xHRAkVhXC}(2020).
\parencite*[][]{partridge_survey_2020}.%
} Therefore, to speak of ``content'' in ``parsimony of content'' and ``primitives'' in [13] (i.e. according to Gruber's definition of ISO) means to account for this relation. The second reason is that parsimony of content does not (aspire to) exhaust the debate on the parsimony of ISOs. In other words, there may \textit{in principle} be other parsimonies involved in the ISOs debate, as well as other ways of applying [4–9] to ISOs. And this is also in line with [10–12], which do not rule out that parsimony could be ``everywhere'' in ISOs, and thus also apply to something other than the content of ISOs 
%\label{ref:RNDquisra6ZW5}(Turner, 2018, pp.161–167).
\parencite[][pp.161–167]{turner_computational_2018}. %
 Moreover, although it would transitively follow from [8–9] that



\setcounter{saveenumtambassi}{\value{enumi}}

\begin{enumerate}[label={[\arabic*]}]

\setcounter{enumi}{\value{saveenumtambassi}}

\item parsimony of content deals with \textit{both} [14–15],

\end{enumerate}

we should also consider the possibility of



\setcounter{saveenumtambassi}{\value{enumi}}

\begin{enumerate}[label={[\arabic*]}]

\setcounter{enumi}{\value{saveenumtambassi}}

\item following [14–15] separately.

\end{enumerate}

Indeed, if adopting parsimony of content means following both [14–15] (see [16]), nothing prevents us from adopting parsimony of content partially, that is, from adopting either [14] or [15] by itself.



\section{On the rivers of the UK}

To specify what it means to adopt parsimony of content in practice, suppose we build an ISO, \textit{ISO}\textit{\textsubscript{1}}, aimed at [\textit{A}\textit{\textsubscript{1}}] listing and [\textit{A}\textit{\textsubscript{2}}] classifying all the rivers of the UK. Unless \textit{A}\textit{\textsubscript{1}} and \textit{A}\textit{\textsubscript{2}} are further specified, \textit{A}\textit{\textsubscript{1}} is fulfilled if and only if



\setcounter{saveenumtambassi}{\value{enumi}}

\begin{enumerate}[label={[\arabic*]}]

\setcounter{enumi}{\value{saveenumtambassi}}

\item no river in the UK is excluded from \textit{ISO}\textit{\textsubscript{1}},

\end{enumerate}

whereas achieving \textit{A}\textit{\textsubscript{2}} means



\setcounter{saveenumtambassi}{\value{enumi}}

\begin{enumerate}[label={[\arabic*]}]

\setcounter{enumi}{\value{saveenumtambassi}}

\item providing any classification of such rivers.

\end{enumerate}

[18] generally refers to the notion of completeness (of ISOs),\footnote{See Bittner and Smith (2008).} according to which



\setcounter{saveenumtambassi}{\value{enumi}}

\begin{enumerate}[label={[\arabic*]}]

\setcounter{enumi}{\value{saveenumtambassi}}

\item the contents of an ISO should be exhaustive\footnote{See Tambassi (2021b). ``Exhaustive'' in [20] also refers to the debate on categories in philosophy, within which ``exhaustive'' represents one of the three criteria of adequacy (see Cumpa 2019), indicating that whatever there is (or could be) should find its place in one and only one category (see Thomasson 2019). } with respect to the domain that the ISO aims to model.

\end{enumerate}

For \textit{ISO}\textit{\textsubscript{1}}, [20] means that the nearly 1,500 rivers crossing the UK should find their place among the contents of \textit{ISO}\textit{\textsubscript{1}}, which ultimately fall within (one of) the primitives of \textit{ISO}\textit{\textsubscript{1}} (see also [13]), no matter which primitive.



As for [19], \textit{A}\textit{\textsubscript{2}} can in principle be achieved in many ways. For example, \textit{ISO}\textit{\textsubscript{1}} could



\setcounter{saveenumtambassi}{\value{enumi}}

\begin{enumerate}[label={[\arabic*]}]

\setcounter{enumi}{\value{saveenumtambassi}}

\item classify the rivers according to their biotic and/or topographic features;

\item systematize the rivers according to the geographical region(s) they cross;

\item catalogue the rivers according to some (arbitrary) length intervals;

\item consider [21–23] together;

\item provide any arbitrary classification.

\end{enumerate}

The reason why there can be many ways to achieve \textit{A}\textit{\textsubscript{2}} is that \textit{A}\textit{\textsubscript{2}} does not specify any criteria for classifying the UK's rivers. Therefore, to the extent that each of [21–25] classifies the UK's rivers, there is no way to prefer one among [21–25] over the others, at least \mbox{according to \textit{A}\textit{\textsubscript{2}}.}



\section{On the aims of information system ontologies}

According to [14–15], applying parsimony of content to \textit{ISO}\textit{\textsubscript{1}} entails that:



\setcounter{saveenumtambassi}{\value{enumi}}

\begin{enumerate}[label={[\arabic*]}]

\setcounter{enumi}{\value{saveenumtambassi}}

\item \textit{ISO}\textit{\textsubscript{1}} should not go beyond its aims;

\item (and) the primitives of \textit{ISO}\textit{\textsubscript{1}} should be reduced to the essential.

\end{enumerate}

As for [26], \textit{ISO}\textit{\textsubscript{1}} has two aims: \textit{A}\textit{\textsubscript{1}} and \textit{A}\textit{\textsubscript{2}}. In accordance with [26], \textit{ISO}\textit{\textsubscript{1}} is thus expected to



\setcounter{saveenumtambassi}{\value{enumi}}

\begin{enumerate}[label={[\arabic*]}]

\setcounter{enumi}{\value{saveenumtambassi}}

\item list all the UK's rivers (see \textit{A}\textit{\textsubscript{1}}),

\item classify those rivers (see \textit{A}\textit{\textsubscript{2}}),

\item do nothing more than what [25–26] specify.

\end{enumerate}

[28] implies [18], [29] leads to [21–25], and thus assumes that there can be different ways of fulfilling [26], or that \textit{ISO}\textit{\textsubscript{1}} could not go beyond its aims in different ways. [30] limits \textit{ISO}\textit{\textsubscript{1}}'s tasks to [28] (or \textit{A}\textit{\textsubscript{1}}) and to [29] (or \textit{A}\textit{\textsubscript{2}}). This means that, according to [30], \textit{ISO}\textit{\textsubscript{1}} should not, for example,



\setcounter{saveenumtambassi}{\value{enumi}}

\begin{enumerate}[label={[\arabic*]}]

\setcounter{enumi}{\value{saveenumtambassi}}

\item list the UK's lakes,

\item (or) classify Germany's rivers,

\end{enumerate}

because [31–32] would go beyond \textit{A}\textit{\textsubscript{1}} and \textit{A}\textit{\textsubscript{2}}, and hence contradict [26] and [28–29]. All this also implies that



\setcounter{saveenumtambassi}{\value{enumi}}

\begin{enumerate}[label={[\arabic*]}]

\setcounter{enumi}{\value{saveenumtambassi}}

\item (all) \textit{ISO}\textit{\textsubscript{1}}'s contents should be consistent with and functional to its aims,

\end{enumerate}

but also that



\setcounter{saveenumtambassi}{\value{enumi}}

\begin{enumerate}[label={[\arabic*]}]

\setcounter{enumi}{\value{saveenumtambassi}}

\item no content of \textit{ISO}\textit{\textsubscript{1}} should go beyond the aims of \textit{ISO}\textit{\textsubscript{1}}.

\end{enumerate}

However, things can get complicated in cases like the following. Suppose we fulfill [29] by means of [23], that is, by classifying the UK's rivers according to some (arbitrary) length intervals, such as 0–40 miles, 40–80 miles, 80–120 miles, and so on. What about the property ``length of the river''? Does the inclusion of such a~property within \textit{ISO}\textit{\textsubscript{1}}'s contents follow from [33–34]? On the one hand, one could answer ``no'': \textit{A}\textit{\textsubscript{1}} and \textit{A}\textit{\textsubscript{2}} only require [28–29], which do not explicitly refer to the specific length of the rivers. On the other hand, one could also answer ``yes'', insofar as the ``length of the river'' would justify the assignment of each (UK) river to one of the length intervals of [23].



\section{Completeness and parsimony of content}

The principle of completeness (of ISOs) states that the contents of an ISO should be exhaustive for the domain that the ISO aims to model (see [20]). Applying completeness to (\textit{ISO}\textit{\textsubscript{1}}'s) \textit{A}\textit{\textsubscript{1}} implies [14], but does not exclude that:



\setcounter{saveenumtambassi}{\value{enumi}}

\begin{enumerate}[label={[\arabic*]}]

\setcounter{enumi}{\value{saveenumtambassi}}

\item the same river appears twice (or several times) in \textit{ISO}\textit{\textsubscript{1}},

\item \textit{ISO}\textit{\textsubscript{1}} also includes the UK's lakes and/or Germany's rivers.

\end{enumerate}

Conversely, applying parsimony of content to \textit{A}\textit{\textsubscript{1}} implies [18], but excludes [36] because of [33–34]---which are ultimately inferred from [26]. From [36], however, it does not follow that completeness and parsimony of content are mutually contradictory, since:



\setcounter{saveenumtambassi}{\value{enumi}}

\begin{enumerate}[label={[\arabic*]}]

\setcounter{enumi}{\value{saveenumtambassi}}

\item ISOs may consistently follow both completeness and parsimony of content (see [1]).

\end{enumerate}

To justify [37], we could consider the negation of [36] to be only a~possibility for completeness, as well as a~necessity for parsimony of content. The same, we may add, can be said for the negation of [35]. If so,



\setcounter{saveenumtambassi}{\value{enumi}}

\begin{enumerate}[label={[\arabic*]}]

\setcounter{enumi}{\value{saveenumtambassi}}

\item how does the negation of [35] follow from parsimony of content?

\end{enumerate}

To answer [38], let us return to [27], according to which \textit{ISO}\textit{\textsubscript{1}}'s primitives should be reduced to the essential. If [27], an (easy) solution might be to avoid repetitions, so that



\setcounter{saveenumtambassi}{\value{enumi}}

\begin{enumerate}[label={[\arabic*]}]

\setcounter{enumi}{\value{saveenumtambassi}}

\item each content of an ISO should appear only once in the same ISO.

\end{enumerate}

Now, [39] is based on [27], which follows from [15], which in turn is one of the two pillars of parsimony of content (see [14–15]). Moreover, maintaining [39] means affirming the negation of [35], which is a~necessity for parsimony of content and a~possibility for completeness. But if so, [37] can also be justified by [35].



\section{Parsimony of content and (representational) primitives}

While [39] follows from [27], this is not all. Indeed, ``\textit{ISO}\textit{\textsubscript{1}}'s primitives should be reduced to the essential'' seems to be open to different interpretations, such as:



\setcounter{saveenumtambassi}{\value{enumi}}

\begin{enumerate}[label={[\arabic*]}]

\setcounter{enumi}{\value{saveenumtambassi}}

\item reducing the types of the primitives we use (to the essential);

\item reducing the tokens of the primitives we use (to the essential);\footnote{The distinction between [40] and [41] is largely based on Fiddaman and Rodriguez-Pereyra's 
%\label{ref:RND3nV9UetV8V}(2018)
\parencite*[][]{fiddaman_razor_2018} %
distinction between two different forms of ontological economy. According to the authors, the principle of qualitative economy requires us to avoid multiplying types of entities when not necessary, while that of quantitative economy requires us to avoid multiplying token entities when not necessary. For further reading on ontological economy, see Sober 
%\label{ref:RNDG9yseufJnE}(1975),
\parencite*[][]{sober_simplicity_1975}, %
 Lewis 
%\label{ref:RNDS0R1896DZL}(1973),
\parencite*[][]{lewis_counterfactuals_1973}, %
 van Inwagen 
%\label{ref:RNDPFpyRoLDbH}(2001),
\parencite*[][]{van_inwagen_ontology_2001}, %
 Lando 
%\label{ref:RNDJwWAkM47VD}(2010),
\parencite*[][]{lando_ontologia_2010}, %
 and Schaffer 
%\label{ref:RNDme5rBj5rmm}(2015).
\parencite*[][]{schaffer_what_2015}.%
}


\item combining [40] and [41].

\end{enumerate}
\setcounter{saveenumtambassi}{\value{enumi}}

To explain [40–42], let us imagine that \textit{ISO}\textit{\textsubscript{1}} follows [23] and thus classifies all the UK's rivers according to some (arbitrary) length intervals. \textit{ISO}\textit{\textsubscript{1}}\textit{\textcolor[rgb]{0.1254902,0.12941177,0.14117648}{ }}\textit{\textcolor[rgb]{0.1254902,0.12941177,0.14117648}{${\wedge}$}} \textit{[23]} therefore has two aims: (\textit{A}\textit{\textsubscript{3}}) to list all the UK's rivers and (\textit{A}\textit{\textsubscript{4}}) to classify them according to [23].



Now, [40] suggests reducing the types of primitives: using fewer primitive types to model a~domain (see [13] and [20]) is preferable to modelling the same domain using more primitive types. This means that placing [\textit{S}\textit{\textsubscript{1}}] the UK's rivers among the instances of \textit{ISO}\textit{\textsubscript{1}} \textit{${\wedge}$ [23] }and the intervals of length among the classes of \textit{ISO}\textit{\textsubscript{1}} \textit{${\wedge}$ [23]} (respectively) would be preferable to [\textit{S}\textit{\textsubscript{2}}] placing those rivers and length intervals among the instances, classes, and properties of \textit{ISO}\textit{\textsubscript{1}} \textit{${\wedge}$ [23]}. Indeed, \textit{S}\textit{\textsubscript{1}} uses fewer primitive types than \textit{S}\textit{\textsubscript{2}}.



[41] is instead ambiguous. It may refer to


\begin{enumerate}[label={[\arabic*]}]



\item [{[41.1]}] an ISO's overall amount of tokens,

\end{enumerate}

meaning that the tokens of \textit{ISO}\textit{\textsubscript{1}} \textit{${\wedge}$ [23]} should be reduced to the essential. Now, while \textit{A}\textit{\textsubscript{3}} (simply) requires that all of the nearly 1,500 rivers crossing the UK find their place among the contents of \textit{ISO}\textit{\textsubscript{1}} \textit{\textcolor[rgb]{0.1254902,0.12941177,0.14117648}{${\wedge}$}} \textit{[23]} (for example, among the instances of \textit{ISO}\textit{\textsubscript{1}} \textit{\textcolor[rgb]{0.1254902,0.12941177,0.14117648}{${\wedge}$}} \textit{[23]}), \textit{A}\textit{\textsubscript{4}} might be fulfilled in different ways. Supposing, for example, that each length interval corresponds to a~class of \textit{ISO}\textit{\textsubscript{1}} \textit{\textcolor[rgb]{0.1254902,0.12941177,0.14117648}{${\wedge}$}} \textit{[23]}, [41.1] suggests that [\textit{S}\textit{\textsubscript{3}}] classifying the UK's rivers by means of two length intervals (e.g., ``longer than 100 miles'' and ``shorter than 100 miles'') is preferable to [\textit{S}\textit{\textsubscript{4}}] classifying those rivers by means of five length intervals (e.g., ``between 0–30 miles'', ``between 40–80 miles'', ``between 80–120 miles'', and so forth). Why so? Because \textit{S}\textit{\textsubscript{3}} requires (almost) 1,500 instances and 2 classes, 1,502 tokens in total, whereas \textit{S}\textit{\textsubscript{4}} requires (almost) 1,500 instances and 5 classes, 1,505 tokens in total. (This also means that, insofar as \textit{S}\textit{\textsubscript{3}} and \textit{S}\textit{\textsubscript{4}} are both consistent with the aims of \textit{ISO}\textit{\textsubscript{1}} \textit{\textcolor[rgb]{0.1254902,0.12941177,0.14117648}{${\wedge}$}} \textit{[23]}, it is irrelevant to [41.1] whether \textit{S}\textit{\textsubscript{4}} is more detailed than \textit{S}\textit{\textsubscript{3}}). But [41.1] also represents a~way of balancing the overall tokens of \textit{ISO}\textit{\textsubscript{1}} \textit{${\wedge}$ [23]} within the various primitives. For example, if achieving \textit{A}\textit{\textsubscript{3}} and \textit{A}\textit{\textsubscript{4}} required that \textit{S}\textit{\textsubscript{3}} also include 10 properties and \textit{S}\textit{\textsubscript{4}} includes 2 properties, then \textit{S}\textit{\textsubscript{4}} would be preferable to \textit{S}\textit{\textsubscript{3}}. In other words, the reduction of tokens within one primitive should not be at the expense of a~proliferation of tokens within the whole ISO.



[41.1], however, is not the only way to interpret [41], which might also refer to





\begin{enumerate}[label={[\arabic*]}]



\item [{[41.2]}] the tokens of each primitive.

\end{enumerate}

In turn, [41.2] could have two interpretations: [41.2.1] and [41.2.2]. [41.2.1] indicates that modelling \textit{ISO}\textit{\textsubscript{1}} \textit{${\wedge}$ [23]} by means of, for example, [\textit{S}\textit{\textsubscript{5}}] 10 classes, 2 relations and 1,500 instances is preferable to modelling \textit{ISO}\textit{\textsubscript{1}} \textit{${\wedge}$ [23]} by means of [\textit{S}\textit{\textsubscript{6}}] 2 classes, 3 relations, 2 properties and 1,500 instances. For although there is no difference between \textit{S}\textit{\textsubscript{5}} and \textit{S}\textit{\textsubscript{6}} in terms of the tokens of instances, \textit{and} \textit{S}\textit{\textsubscript{6}} is preferable to \textit{S}\textit{\textsubscript{5}} in terms of the tokens of classes, \textit{S}\textit{\textsubscript{5}} is preferable to \textit{S}\textit{\textsubscript{6}} in terms of the tokens of relations and properties. This means that, according to [41.2.1], we should prefer \textit{S}\textit{\textsubscript{5}} over \textit{S}\textit{\textsubscript{6}}, insofar as \textit{S}\textit{\textsubscript{5}} is preferable with regard to both relations and properties, and \textit{S}\textit{\textsubscript{6}} is preferable only with regard to classes. [41.2.2] instead focuses on the tokens of each primitive, the reduction of which is independent from one primitive to another, and does not directly concern [41.1] or [41.2.1]. In other words, [41.2.2] offers the chance to apply [41] (and more generally [15] or [27]) to one and only one primitive. Consequently, we could have a~[41] based on tokens of classes when the tokens of classes are reduced to the essential, a~[41] based on the tokens of relations when the tokens of relations are reduced to the essential, and so on for each primitive. All this does not imply that those applications of [41] to one and only one primitive cannot be combined to improve [15], [27] and/or [40–41], nor that the list of primitives will never change, and with it the varieties of applications of [41] to which the primitives refer.



However, there are also ambiguities surrounding [42]. Firstly, it is unclear whether



%\setcounter{saveenumtambassi}{\value{enumi}}

\begin{enumerate}[label={[\arabic*]}]

\setcounter{enumi}{\value{saveenumtambassi}}

\item the combination refers to [40] and [41.1], or [40] and [41.2.1], or [40] and [41.2.2], or [40], [41.1], and [41.2.1], and so on.

\end{enumerate}

Secondly,



\setcounter{saveenumtambassi}{\value{enumi}}

\begin{enumerate}[label={[\arabic*]}]

\setcounter{enumi}{\value{saveenumtambassi}}

\item once [43] is clarified, we should also define the order of priority of the combination.

\end{enumerate}

To clarify [44], let us suppose that the combination refers to [40] and [41.1]. Giving priority to [40] means that reducing the types of primitives is more important than reducing the total number of tokens: that is, both primitive types and tokens should be reduced to the essential but the reduction of the tokens comes after that of the types of primitives. Giving priority to [41.1] means the opposite.



\section{Parsimony (of content) as a~set of criteria}

According to [14], ISOs should not go beyond their aims, whatever these may be. As regards the contents of an ISO, [14] means that they should all be consistent with the ISO's aims (see [33–34]). According to [15], for any ISO, we should reduce the types of primitives (see [40]), the total number of tokens (see [41.1]), or the tokens of each primitive (see [41.2.1] and [41.2.2]) to the essential. Alternatively (see [42]), we could adopt [40] and one or more of [41.1], [41.2.1], and [41.2.2] by defining their priority. According to [16], we should adopt both [14] and [15], or better [14] and at least one of [40], [41.1], [41.2.1], [41.2.2], or [42].



On this basis, let us focus on \textit{ISO}\textit{\textsubscript{1}}'s \textit{A}\textit{\textsubscript{2}}, according to which \textit{ISO}\textit{\textsubscript{1}} should provide a~classification of the UK's rivers. Now, insofar as \textit{A}\textit{\textsubscript{2}} does not specify any criteria to classify the UK's rivers and [21–25] are (all) consistent with \textit{A}\textit{\textsubscript{2}}, there is no reason why we should not regard



\setcounter{saveenumtambassi}{\value{enumi}}

\begin{enumerate}[label={[\arabic*]}]

\setcounter{enumi}{\value{saveenumtambassi}}

\item {[21–25]}  as \textit{equally} \textit{consistent} with \textit{A}\textit{\textsubscript{2}}.

\end{enumerate}

But, if [45], how are we to choose among [21–25]? The fact that the criteria, if any, are not deducible from \textit{A}\textit{\textsubscript{2}} does not imply or guarantee that [14–16] provide any criteria. In other words,



\setcounter{saveenumtambassi}{\value{enumi}}

\begin{enumerate}[label={[\arabic*]}]

\setcounter{enumi}{\value{saveenumtambassi}}

\item choosing among [21–25] may both [46.1] (at least partially) depend \textit{and} [46.2] not depend on (some of) [14–16].

\end{enumerate}

In turn, [46] does not imply or guarantee that



\setcounter{saveenumtambassi}{\value{enumi}}

\begin{enumerate}[label={[\arabic*]}]

\setcounter{enumi}{\value{saveenumtambassi}}

\item once we choose among [21–25], [14–16] provide criteria for selecting and/or classifying the contents of ISOs.

\end{enumerate}

All this means that parsimony of content (in general) can provide:



\setcounter{saveenumtambassi}{\value{enumi}}

\begin{enumerate}[label={[\arabic*]}]

\setcounter{enumi}{\value{saveenumtambassi}}

\item some criteria for choosing among different and equally consistent classifications/ISOs;

\item some criteria for selecting and/or classifying the content of ISOs;

\item both [48] and [49];

\item neither [48] nor [49].

\end{enumerate}

\section{Concluding remarks }

Since some ISOs adopt parsimony as an implicit and uncritical assumption, and/or without explaining what parsimony specifically consists of (or refers to), these pages sought to clarify the point. In this regard, I~introduced the notion of parsimony of content, showing that



\setcounter{saveenumtambassi}{\value{enumi}}

\begin{enumerate}[label={[\arabic*]}]

\setcounter{enumi}{\value{saveenumtambassi}}

\item this parsimony concerns two main claims, [14–15], as well as their connection, [16], from which [33–34], [37], [39–40], [41.1], [41.2.1], [41.2.2], [43–44] and [48–51] follow.

\end{enumerate}

[52] broadly suggests that the adoption of parsimony of content has to do with



\setcounter{saveenumtambassi}{\value{enumi}}

\begin{enumerate}[label={[\arabic*]}]

\setcounter{enumi}{\value{saveenumtambassi}}

\item the interpretation and combination of claims about parsimony of content,

\item specifying whether parsimony of content provides some criteria for choosing among different classifications/ISOs and/or for selecting and/or classifying the contents of ISOs.\footnote{Unlike some computer scientists 
%\label{ref:RNDYmGYe3RjMq}(Floyd, 1967),
\parencite[][]{floyd_assigning_1967}, %
 I~have not considered the possibility of combining parsimony of content with modularization: indeed, breaking down complex ISOs into (in)dependent modules would simply defer the question of adopting parsimony of content to both complex ISOs and their (in)dependent modules.}



\end{enumerate}

All this means that



\setcounter{saveenumtambassi}{\value{enumi}}

\begin{enumerate}[label={[\arabic*]}]

\setcounter{enumi}{\value{saveenumtambassi}}

\item the notion (and application) of parsimony of content is multifaceted;

\item an informed adoption of parsimony of content requires [53–54].

\end{enumerate}

It does not follow from [55–56] that parsimony of content exhausts the debate on the parsimony of ISOs, nor that ISOs are bound to adopt parsimony of content. In other words, [55–56] are consistent with [1–3], thus ensuring the plurality of the methodological approaches shaping the debate on ISOs.



\paragraph{Funding.} This paper has received funding from the European Research Council under the European Union Horizon Europe Research and Innovation Programme (GA no. 101041596 ERC---PolyphonicPhilosophy).



\paragraph{Disclamer.} Funded by the European Union. Views and opinions expressed are however those of the author(s) only and do not necessarily reflect those of the European Union. Neither the European Union nor the granting authority can be held responsible for them.

\end{artengenv}
