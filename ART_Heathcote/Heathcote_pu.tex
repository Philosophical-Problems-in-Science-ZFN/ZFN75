\begin{artengenv}{Adrian Heathcote}
	{Realism, irrationality,\\and spinor spaces}
	{Realism, irrationality, and spinor spaces}
	{Realism, irrationality, and spinor spaces}
	{
%	University of Sydney}
	}
	{Mathematics, as Eugene Wigner noted, is unreasonably effective in physics. The argument of this paper is that the disproportionate attention that philosophers have paid to discrete structures such as the natural numbers, for which a~nominalist construction may be possible, has deprived us of the best argument for platonism, which lies in continuous structures---in fields and their derived algebras, such as Clifford algebras. The argument that Wigner was making is best made with respect to such structures---in a~loose sense, with respect to geometry rather than arithmetic. The purpose of the present paper is to make this connection between mathematical realism and geometrical entities. It thus constitutes an argument against formalism, for which mathematics is merely a~game with humanly set rules; and nominalism, in which whatever mathematics is used is eliminable in the final analysis, by often insufficiently specified means. The hope is that light may be cast on the stubborn mysteries of the nature of quantum mechanics and its mathematical formulation, with particular reference to spinor representations---as they have been developed by Andrej Trautman. Thus, according to our argument, QM may appear more natural, as we have better reasons to take spinor structures as irreducibly real, a~view consonant with the work of Trautman and Penrose in particular.
	}
	{indispensibility, nominalism, spinors, complex numbers, incommensurability.}




\lettrine[loversize=0.13,lines=2,lraise=-0.03,nindent=0em,findent=0.2pt]%
{M}{}any who have more than a~passing interest in mathematical physics have been impressed by the intimate connection that exists between quite advanced mathematics and the elucidation of our best physical theories, and being so impressed have taken this as an argument for a~form of mathematical platonism. Yet, in the wider philosophical community, and certainly in the culture at large, nominalism \textit{seems} (perhaps only to a~jaundiced eye) to dominate. Thus we have a~rather stark opposition between philosophy and science in which the two sides appear to be largely talking past one another, and little that is said advances the debate in a~successful manner,

The present paper is an attempt to get beyond this impasse by offering a~way of recasting the issues, so that 1) a~central part of the nominalist intuition can be seen to have some plausibility; and 2) that nevertheless the platonist can be seen to be correct in that mathematical physics does in fact offer an argument for the reality of mathematical entities. Indeed, my suggestion will be that there is a~straight line between the motivation for platonism among the ancient Greeks and platonism today. Thus the main claim of the present work is that there is a~mechanism for the expansion of our mathematical ontology that is directly tied to our progress in mathematical physics, a~connection that is unlikely to be accidental. In brief: the taking of roots is often \textit{ontologically ampliative}. 

We may begin by noting that perhaps the most important way that the discussion has gone astray is through the historical focus on the arithmetic of the natural numbers, a~focus that was present in Kant as well as Frege, and that flowed naturally through the reductive programmes of the 20\textsuperscript{th} Century. The natural numbers were seen to have first place in the \textit{ordo cognoscendi}: they were our original mathematics---account for these and all else will somehow surely fall into place. In due course philosophical discussion became bound to the twin poles of arithmetic and set theory---the latter having first place in the \textit{ordo essendi}. Though nominalists and realists disagreed on what should be in our ontology, they were at least disposed to agree on what mathematics we should be considering. 

The implicit thought here seems to be that whatever we can say about the natural numbers we will be able to say about any other mathematical structure. However I~want to suggest that this is false: that the natural numbers are a~special case that lend themselves to a~very special nominalist explanation, an explanation that does not extend to other mathematical entities in which we might be interested.



\section{A~nominalism for arithmetic}

Let us begin by giving the Peano axioms in their second-order form. We modify them in a~way that is now customary by taking the first number as 0. Since 0 is the additive unit it means that much more of what would ordinarily be considered elementary arithmetic is derivable. However it also means that we would have to be careful in the statement of divisibility. Peano's own statement would lead to problems unless modified, for it would allow division by zero.\footnote{An extra condition stating that in all cases of $m/n, n~\neq 0$ would be sufficient. This original axiomatisation is weaker than that of Hilbert and Bernays in their \textit{Grundlagen}.}

Where Peano speaks, in the first axiom (and then throughout) of the $n$ being a~member of a~set $N$, I~will be explicit that this set is to be the set of natural numbers $\mathbb{N}$. \\

\noindent\textsc{Axioms for Peano Arithmetic}

\begin{enumerate}[label=P\Roman* :]

  \item 0 is a~natural number;
  
  \item For every natural number $n, n~+ 1$ is a~natural number
  
  \item For every natural number $n, n~+ 1 \neq$ 0;
  
  \item For all natural numbers n~and m, $n + 1 = m~+ 1$ if and only if $n = m$;
  
  \item If $\phi$ is a~property of numbers such that: 0 is $\phi$, and for every natural number $n$, if $n$ is $\phi$, then $n + 1$ is $\phi$, then all natural numbers $n$ are $\phi$;
  
  \item $n + 0 = n$;
  
  \item $n + (m + 1) = (n + m) + 1$;
  
  \item $n . 0 = 0$;
  
  \item $n . (m + 1) =  (n . m) + n$;
  
  \item $n . (m + p) = (n . m) + (n . p)$.\\
\end{enumerate}

These axioms, as is well known, are derived from Dedekind's \textit{Was Sind und was sollen die Zahlen?} \parencite*{dedekind_was_1888}, and Dedekind had there shown that his axiom-set is categorical. His method, as outlined in his letter to Hans Keferstein in 1890, is not to appeal to known features of the natural numbers---this, he says, would result in a~vicious circularity---but to give axioms that ought to determine any infinite, well-ordered set \parencite{van_heijenoort_frege_1967}.

But now we come to the crucial point. Not only are these axioms such that they characterise the natural numbers, \emph{they also characterise the numerals that \emph{name} the natural numbers}. For the numerals are also a~well-ordered infinite set and begin with a~first numeral `0'. To achieve this isomorphism we must understand that numerals are not identical with inscriptions of numerals: there are numerals that no one \textit{will} ever, or \textit{could} ever, write down. But no matter, these numerals exist and there are many that cannot be written down that can be characterised by a~definite description---thus the name ``Graham's Number'' is an abbreviation of a~definite description where the numeral itself could not be written down without a~secondary abbreviated notation. 

Of course, there will be some nominalists for whom an infinite set of numerals is already going too far in the direction of platonism: it must be understood that the way out of the problem that I~am offering here will not be a~way that is open to them. But a~rigid Inscriptionism is, I~believe, a~most difficult position to extract explanatory content from, and so we must await someone who is prepared to try to make it work. At any rate I~say no more about such a~view here.

Allowing ourselves an infinite set of numerals we can check the Peano axioms to see what they mean when applied to numerals. As already noted neither Peano nor Dedekind mention numbers, for their purpose in providing an axiomatisation is to characterise numbers without circular descriptions. So, adapting Peano, we have simply:

\begin{enumerate}[label=P*\Roman* :]

  \item 0 $\in N$;
  
  \item If $n \in N$ then $n + 1 \in N$;
  
  \item If $n \in N$ then $n + 1 \neq 0$;
  
  \item For $n$ and $m \in N$, $n + 1 = m~+ 1$ if and only if $n = m$;
  
  \item If $\phi$ is a~property of the members of N~such that: `0' is $\phi$, and for every $n \in N$, if $n$ is $\phi$, then $n + 1$ is $\phi$, then all $n \in N$ are $\phi$;
  
  \item etc.
  \end{enumerate}
Since addition is simply an operation that takes a~member of $N$ to another member of $N$ it is also well-defined on numerals: it is simply counting forward. Likewise for multiplication. Thus the remaining Peano axioms will also have a~clear meaning.

Now the philosophical point should be clear: since there is an isomorphism between the two models of the Peano axioms, and since we use the numerals to speak of numbers, there is always a~danger that we will confuse the two---and, the nominalist may say, we \emph{have} confused them, and confused them throughout history. Thus we are, whether we are nominalists or realists, simply creating confusion if we say that `numbers can be written down'. I~can write down a~numeral but I~cannot write down a~number. By analogy, to make the point clear, I~cannot write down Mary but I~can write down Mary's name, `Mary'. So when we speak of writing down numbers we are already confusing a~name with the referent of the name. Thus in Peano's axiomatization what is written down and axiomatised are numerals.\footnote{See Button and Walsh \parencite*{button_philosophy_2018} for a~ discussion of the rôle of language in axiomatisations, including second-order axiomatisations of arithmetic. In their section 1.13 there is again signs of confusion between numbers and numerals. Properly, however, in such second-order axiomatisations we are quantifying over properties of numbers themselves, but we then also sacrifice Dedekind's desideratum of non-circularity.}

Now if we take a~Medieval conception of nominalism, we may hold that there are nothing \textit{but} numerals, that these do not refer to numbers, as a~name refers to a~thing, but that they are \textit{all there is} to what we think of as number. Thus numerals are a~\textit{flatus vocis}, in Roscelin's phrase, an empty wind, and mathematics is simply a~game with rules for the manipulation of these numerals. In the 19\textsuperscript{th} Century there is evidence that this was the view of Helmholtz and Kronecker, though undoubtedly many others followed in the 20\textsuperscript{th} Century, notably the Formalists.\footnote{No direct evidence of Roscelin's position survives, only the replies of his opponents, such as John of Salisbury. Thus see Joseph Owens \parencite*{owens_faith_1982}.} 

Some credence is given to this position if we ask ourselves, sympathetic to this nominalism, what the law of the commutativity for multiplication means: if we multiply together two numbers \thing{a} and \thing{b} then the order of the multiplication does not matter. But, says my imaginary nominalist, surely the \textit{order of an operation} suggests something that \textit{we} do, some way of manipulating objects, in a~particular sequence, and the only objects available for us to manipulate are \textit{numerals}. Likewise with associativity: the order in which an operation is performed suggests an action with consequences. After all, \textit{to add} and \textit{to multiply} are verbs and require objects on which the action is to be performed.\footnote{One can find something of this view expressed in Whitehead's \textit{Universal Algebra}, where he speaks in the introduction of $a + b$ and $b + a$ `directing different thoughts'. I~do not say that this performative interpretation of arithmetical operations is correct, merely that if we have it then it seems most apt to apply it to numerals.}

Now I~will say that I~think we have here the beginning of an interesting discussion about nominalism that could be developed further, and one that would be helpful in clearing our minds of long standing confusion. In particular it may help us understand what we mean when we make a~distinction between the \textit{potential infinite} and the \textit{actual infinite}, for there is a~clear sense in which there are a~potential infinity of numerals that we may write down. By contrast I~am not sure that sense can be made of saying that \textit{numbers themselves} are potentially infinite: either they are finite or they are infinite, and there is nothing in between these two cardinalities. Nor, if it is numbers themselves that are being thought of as potentially infinite, is it all clear what would be \textit{releasing} or \textit{realising} this potential. For \textit{whom} is this potential realised? \textit{When} is it being realised? Can these numbers \textit{return} to being \textit{un}realised? Confusion between name and referent is rife in this area, and of long standing.

But I~cut this discussion short to say that, ultimately, I~do not believe that it can be correct for anything more than the natural numbers (and in the light of an argument to come in \S 5, not even there). It depends on our having numerals which can stand in proxy for natural numbers and thinks of numerical operations as manipulations of those numerals. But, as Hilbert realised, this cannot be extended to the real or complex numbers---a point I~come to in the next two sections.

However, I~think that something like the above reasoning was present to the Pythagoreans and Plato: as long as we had to think \textit{only} of the natural numbers we were able to be lulled into a~state of Nominalism about numbers. But when irrational magnitudes were discovered there was no longer a~way to avoid realism. The argument for this, with some historical evidence, is given in the next section.




\section{Plato and incommensurability}

Mathematics began as an abstract discipline, I~suggest---as opposed to a~pragmatic aid to accounting---with the Pythagorean discovery that the square root of two cannot be either a~whole number or a~ratio of whole numbers. There are now many proofs of this, but here is a~beautiful, little-known one by Theodor Estermann \parencite*{estermann_irrationality_1975}.
 (It isn't known what proof the Pythagoreans actually used, though there has been much speculation. Nor can it be certain that the Pythagoreans were the first to construct such a~proof.) 

If \sqrtwo were a~fraction then there would be a~set of natural numbers~\thing{S,} whose members when multiplied by said fraction would yield a~natural number. And if there is such a~set then by well-ordering there is a~least member of that set: call it \thing{k}. So \ksqrtwo{k} is a~natural number and by definition the smallest such number. But on the hypothesis that \sqrtwo is a~fraction we can find a~number \thing{m} that is smaller than \thing{k} for which \ksqrtwo{m} is also a~whole number. Thus consider \thing{m} = \thing{k}(\sqrtwo $- 1$) = \ksqrtwo{k} $-$ \thing{k}. We now have $$m\sqrt{2} = (k\sqrt{2} - k)\sqrt{2} = 2k - k\sqrt{2}.$$ This shows that \thing{m} is a~member of \thing{S} since \thing{2k} $-$ \ksqrtwo{k} is obviously a~whole number. But this \thing{m} is also less than \thing{k}  (the number 1 was chosen specifically so that we would have$$0 < \sqrt{2} - 1 < 1$$ and thus \thing{m} = \thing{k}(\sqrtwo $-$ 1) is less than \thing{k}). So we have found an \thing{m} $<$ \thing{k,} with \thing{m} $\in$ \thing{S}, contrary to the hypothesis that \thing{k} was the least member in \thing{S}. Repeating the proof will produce an infinitely descending set of natural numbers, which is impossible.

The beauty of this proof, besides its great simplicity, is that it relies only on the properties of natural numbers and ratios of same. As Man-Keung Siu has pointed out there is an interpretation of this proof in the geometry of triangles, but the proof itself is free of any geometric assumptions.\footnote{Man-Keung Siu \parencite*{siu_estermann_1998}. See also P. Shiu \parencite*{shiu_more_1999}.} The proof can also be generalised to the square root of any number that is not a~perfect square, as Estermann noted, while requiring no heavy theorems like the Fundamental Theorem of Arithmetic.

The Pythagoreans of the 6\textsuperscript{th} Century \textsc{bc} probably did not have available \textit{this} proof (if they had, the generalisations to other non-square numbers would have been evident to them) but no matter---they had some other that proved the same fact: \sqrtwo cannot be either a~whole number or a~ratio of whole numbers. And it is a~simple application of the Pythagorean Theorem that the diagonal of a~unit square has a~length that is \sqrtwo \emph{and so, such a~length \emph{must} exist}. It was left to the mathematician Theodorus to extend the proof up to 17 and Theaetetus to generalise the discovery to the square roots of all numbers that are not perfect squares (and again, we cannot be sure what proof was used). By the time of Euclid this discovery was well-developed as the theory of \textit{incommensurable} magnitudes, and developed in books V, IX and X~of the \textit{Elements}. In Book X~Euclid extends the theory of irrationals to all that have the form $$\sqrt{\sqrt{a} \pm \sqrt{b}}.$$ A~lost book of Apollonius is meant to have gone further and considered those that were unordered---possibly including $\pi$.

The mathematical significance of this discovery has been thoroughly researched, by Knorr \parencite*{knorr_evolution_1975} and Fowler \parencite*{fowler_mathematics_1999}. But what about the metaphysical significance? In metaphysical terms, what can \sqrtwo be, and what can it not be?

The best way to approach this question is to ask what \sqrtwo could \textit{not} be. The first thing is that, given the above proof and others like it, we cannot automatically think that a~nominalising strategy that might look promising for the natural numbers or the rationals will work for  \sqrtwo. Thus it might be thought that we could regard number as an abstraction \textit{for our purposes} from aggregates of individuals, as in, \textit{five sheep}, \textit{three goats}---that this is a~\textit{social} fact, like their worth in a~marketplace. I~don't say that such a~nominalising strategy has any real plausibility merely that it will not work for \sqrtwo, for no aggregate of individuals has that number. 

Secondly, it might be thought that some geometrical magnitudes---lengths, areas and volumes---might \textit{be} this number \sqrtwo. But this cannot be right either. The hypotenuse of a~right-angled isosceles (RAI) triangle is not intrinsically any number at all, rational or irrational. Thus if we start with an RAI triangle with catheti of unit length then the hypotenuse will have the length \sqrtwo. But if we had chosen instead to make the hypotenuse of unit length then the catheti of the triangle will each be $\frac{1}{\sqrt{2}}$, which is irrational. The same can be said, \textit{mutatis mutandis}, for areas and volumes. Whereas it might be plausible to think of \textit{things} as having natural units---one goat, one sheep, one neutron, \etc---this cannot be carried across to geometrical magnitudes. And if there are no natural units for geometrical magnitudes then no other such magnitude is intrinsically irrational either.\footnote{The Planck length might be thought to be a~candidate for such a~fundamental unit but it is not clear whether at this level the continuity of the space is destroyed as well.} It is for this reason that, by Euclid's time, the phenomenon revealed by the Pythagorean proof was sometimes referred to as \textit{incommensurability}. This is a~pair-wise relation. The catheti and the hypotenuse of a~RAI triangle cannot \textit{both} be whole numbers or ratios of whole numbers: one must fail, but it is an arbitrary choice which one is made to fail. The consequence is that \sqrtwo cannot be identified with geometrical magnitudes in an absolute sense.

The third form of Nominalism is the one that I~regard as initially the most plausible, and the one that was outlined in the first section, above,. The trouble is that this view will not work either for \sqrtwo. This is because there is \textit{no} \textit{numerical} expression---I must emphasise `\textit{numerical}' to forestall the irrelevant objection that `\sqrtwo' is itself such an expression---for this or any other irrational number. In fact this seems to be how the Pythagoreans themselves understood their discovery: that they had discovered numbers that were \textit{unsayable}. Evidence for this can be found in Plato's statement in \textit{The Republic}: such numbers (or magnitudes) were \textit{arrheton} (unspeakable or unsayable).\footnote{Additional evidence is provided by the title of a~lost work of Democritus, \textit{Of Unsayable Straight Lines and Solids}, noted by Diogenes Laertius. This is the earliest known written work on the Pythagorean discovery, since the Pythagoreans themselves, famously, committed nothing to writing.}


In fact as late as Euclid, Heath reminds us that the term that is normally translated as `rational' was \emph{rheta}, meaning \textit{sayable}, and the obvious root of \textit{arrheton}. By contrast the word in Euclid that we translate as `irrational' was \emph{aloga} which can have as many meanings as that very loaded word \textit{logos}---but will certainly include \textit{beyond words}.\footnote{Euclid in Heath translation \parencite{euclid_thirteen_1956}.}

In saying that irrational numbers are unsayable we do not of course mean (and nor did the Greeks mean) that there is \textit{no} form of words which will describe such numbers, for the expression `the square root of two' is obviously such an expression. The point is that there is no finite expression \textit{in numerals} that will do so. As Leibniz put it, in his \textit{Dialogue on Human Freedom and the Origin of Evil}, of 1695 \parencite{leibniz_dialogue_1989}, such magnitudes as \sqrtwo are not expressible in \textit{numbres exact}, and even God could not find such an expression. If we allow \textit{infinite} forms of expression then we can think of these numbers as limits, for example by the approximation method known as \textit{anthyphyrasis}, which was known in Plato's time. And this in itself leads to a~continued fraction representation of these numbers, as discovered by Pietro Cataldi, Brouncker, Wallis, and Euler. But all of these means of expression are essentially infinite: there is no finite expression in numerals, or \textit{numbres exact}: it is in this sense that they are unsayable. Every schoolchild learns at least one manifestation of this profound fact: the decimal representation of \sqrtwo would be an infinite, non-recurring string of numerals. Cutting it off after any finite length will give a~rational number that is \textit{not} equal to \sqrtwo. So \sqrtwo is something beyond what we can express in numerals. The habitual confusion between numerals and numbers that has given nominalism its longevity is simply not available in this case.

To say that \sqrtwo is unsayable in numerals must also be to wonder whether it is a~number at all. This is the important ontological issue to which we have become numb, but which was still very much a~live issue in the 19\textsuperscript{th} Century. It is a~familiar point that `number' for the Greeks meant \textit{natural} numbers, though they also understood ratios of these natural numbers. So it is possible that Plato could have said, cautiously, that there was \textit{something} that was \sqrtwo but remained agnostic as to whether it was number in a~new sense of the term, or whether it was some \textit{other} kind of entity whose square was a~number! And yet there was at least one good argument for thinking of these unsayable entities as numbers in a~new sense: the square root of 4 is a~number, namely 2; so the square root of 2 surely ought to be something of the same kind, despite being `unsayable'. They marked their caution by distinguishing between \textit{geometria}, as the study that encompasses these entities, and \textit{arithmos}. There is some evidence in the later dialogues that Plato was prepared to take the step of expanding the concept of number to including these new entities, at  \textit{Epinomis} 990d, for example.\footnote{That Plato came at some time in his adulthood to be imbued with Pythagorean concerns is standard, and many date this transition to the post-\textit{Republic} period. But precise dating is more difficult. Philodemus dates it as early as Plato's 27th year. He then says: I~[Philodemus] wrote it up. ‘It had been recognised, however’, he says, ‘that, during that time, the mathematical sciences were also greatly advanced, because Plato was supervising (them) and posing problems that the mathematicians investigated with zeal. In this way, accordingly, this was the first time that issues related to the theory of ratios reached [the peak of their development], and the same holds for the problems related [to definition], since Eudoxus and his followers introduced changes to the old-fashioned approach [of Hippocrates]. Geometry [too] made great progress. For there were produced both the method of analysis and the examination of the limits (of a~problem) and geometry in general was much [advanced]; furthermore, in [optics] and mechanics [\ldots]' Philodemus \textit{History of the Philosophers} in Kalligas \emph{et al} \parencite*{kalligas_platos_2020}.}

If there are entities without numerical names then those entities cannot be collapsed into such names---and the proof that there are an uncountable number of real numbers means that not every real number \textit{can} receive a~name of any kind. Thus even if we allow ourselves to make use of countable \textit{ad hoc} names---as we do with `$\pi$' or \name{e}---or disguised definite descriptions---as we do with `\sqrtwo'---we still have a~significant problem. For with the numerals expressing the natural numbers there come algorithms for the common arithmetical operations. But there is no such natural extension of these algorithms for these \textit{ad hoc} names. How would Plato (or any mathematician before the 19\textsuperscript{th} Century) go about adding \sqrtwo and $\pi$? Can we be sure that \sqrtwo $\times$ \sqrthree = $\sqrt{6}$? In fact it was Dedekind who noted, as late as 1858, that it had never been proven but only assumed that for real numbers
$$\sqrt{a} . \sqrt{b} = \sqrt{ab}.$$
(And the issue was not trivial, as this equation \textit{fails} for complex numbers \parencite[see][]{waterhouse_square_2012}. So a~formalist or fictionalist conception of Nominalism---in which mathematics is \textit{just} the manipulation of symbols according to set rules---has to confront the fact that here we have entities for which there can be no systematic naming procedure. Moreover this must have been evident even in Plato's day, for there is a~complete absence of discussion of adding or multiplying arbitrary incommensurables.

\section{Taking roots}


It seems that we have in this reconstruction a~quite solid argument for a~form of mathematical realism.
\begin{enumerate}[label=\emph{\alph*})]
  \item There exist mathematical entities for which there is no plausible nominalist construal.
  
  \item These entities figure in the measurement of space and time intervals  and curvature, but also in particular physical problems, including those that require the use of calculus. Moreover their properties explain certain things that are impossible: namely the Delian cube problem, squaring the circle, etc.
 \end{enumerate}
In a~sense we have here an indispensability argument. But this `indispensability' is quite targeted in this case, for it is not simply an indispensibility to modern science, but has a~more general cast: an indispensibility to nature herself. For if the nature of irrational numbers is able to explain the \textit{impossibility} of carrying out particular acts, how does a~nominalist or a~fictionalist strategy have anything that can equally explain that impossibility? After all, neither are invoking the existence of any items not already available to the realist. They are arguing for less, and so have fewer resources. As far as I'm aware there is no answer to this in the existent literature. The only nominalist strategy of which I'm aware that might have something to say here is that of Hartry Field in his \parencite*{field_science_1980}. Field helps himself to a~particular space-time manifold model to argue that real numbers are unnecessary, but his argument is restricted just to explaining positive metrical facts, not all facts. I~think his argument fails in general (I take it up in section 5) and if it fails there is nothing to replace it.

And yet though this gives us a~realism of the real numbers---it does not in itself provide us with a~reason to be realist about other mathematical entities.

But the way we go beyond this beginning point is exactly the same as the way mathematics itself evolved beyond this beginning point. Euclid is the germ from which mathematics grew, by demonstration from axioms which are self-evident. For 1700 years mathematics consisted of furthering the work of Euclid by enlarging on the subjects of geometry, arithmetic and analysis. Abstraction led to algebra, whether in whole number solutions, as in that of Diophantus, or generally in real numbers. But whether mathematics was furthered by solving equations or giving proofs, the method by which mathematical knowledge was gained was hardly a~mystical intuition. Mathematical truths are known by proof and calculation.

The stability of mathematical ontology up to the 15\textsuperscript{th} Century and the revival of Platonism and the re-establishment of the Academy in Florence under Marsilio Ficino and Cosimo de Medici, laid the ground for the next expansion: the discovery of the complex numbers.

The tale has been told often enough of the discovery of the method of solving cubic equations by Tartaglia and its theft and publishing by Cardano in 1545. The interpretation of the root of $-1$ as a~geometric mean of $1$ and $-1$ obtained by solving$$\frac{1}{x} = \frac{x}{-1}$$ and the interpretation of this, geometrically, as a~mean proportional perpen\-dicular to the ordinary number line gives us `two-dimensional numbers', removing linearity as an essential condition of what it is to be a~number. Again, the mathematical aspect of the discovery of complex numbers has been well-described elsewhere, but what of the philosophical significance?

The striking thing about the way complex numbers arise in the solution to the cubic is that they seem to force themselves upon us. We are looking for real solutions to a~cubic equation, which itself has only real coefficients, and yet complex numbers arise naturally on the way to the real number solutions. Thus consider this example, from Bombelli's \textit{L' Algebra}: \textit{x}$^{3}$ = 15\textit{ x} + 4.  The three roots of this equation are 4, $-2-\sqrt{3}$, $-2 + \sqrt{3}$. They can be found by solving this equation from Scipione Dal Ferro, with \thing{b} = 15, and \thing{c} = 4: $$x = \sqrt[3]{\frac{c}{2} + \sqrt{\frac{c^{2}}{4} - \frac{b^{3}}{27}}} + \sqrt[3]{\frac{c}{2} - \sqrt{\frac{c^{2}}{4} - \frac{b^{3}}{27}}}.$$ This will give us, on substitution: $$x = \sqrt[3]{2 + \sqrt{-121}} + \sqrt[3]{2 - \sqrt{-121}}$$ or $$x = \sqrt[3]{2 + 11i} + \sqrt[3]{2 - 11i} $$ where each cube root has three solutions. One of these, 2 + \textit{i} along with its conjugate 2 $-$ \textit{i}, Bombelli must have found, since it yields the root 4, which he gives as a~solution to the equation. (Bombelli would have been inclined to discard the negative roots.) 

The philosophical puzzle that Bombelli faced was this: the roots of the equation are acceptable numbers, or at the very least, one of them is; but the method by which we reach them involves taking the cube roots of numbers that appear unreal or ``sophistical''. And the cube roots themselves are also unreal or sophistical. But it is only by adding together these unreal numbers (in conjugate pairs) that we reach the roots, that we \textit{must} take seriously. For Bombelli the puzzle must have verged upon paradox: for he did not regard negative numbers as proper---by contrast he had no problem with irrational numbers---and yet he was taking the square root of negative numbers, and then taking the cube root of the complex radicals that resulted---and then adding them pairwise.\footnote{His notation for $\sqrt{-1}$ was \thing{R} (0 $\cdot$ \thing{m} $\cdot$ 1) which translates directly to $\sqrt{0 - 1}$ with `m' standing for `minus'--- thus neatly avoiding making the negative sign an adjectival modifier. Note also that there are nine pairs that could be summed, and it requires some clarity to realise that only three of those pairs, the conjugates, are relevant for finding the roots.} He declared this discovery as the discovery of a~new kind of cubic radical and said that he had a~geometrical proof of it. He says:
\begin{quotation}\noindent
This kind of root has in its calculation different operations than the others and has a~different name\ldots [It] will seem to most people more sophistic than real. This was the opinion I~held too until I~found its geometrical proof (translated in Federica La Nave and Barry Mazur's \parencite*{la_nave_reading_2002}).
\end{quotation}
This geometric proof of Dal Ferro's equation appears late in Bombelli's work and resembles the geometric proofs of the existence of irrationals: in a~sense complex numbers stand to irrationals as Dal Ferro's equation stands to Pythagoras's Theorem---they both emerge as surprising solutions given well-recognised inputs.\footnote{The geometric proof is broken down a~little in La Nave and Mazur \parencite*[17ff]{la_nave_reading_2002}.  See also Barry Mazur's \parencite*{mazur_imagining_2004}, which tells the story of Bombelli's imaginative leap.} However it was not for another 100 years, when Wallis and then De Moivre showed that \isqrtwo{}{-1} could be not just be proven to exist but also given a~representation in the Euclidean plane, the mis-named \textit{Argand} plane, that its acceptance was assured. But they---\emph{i.e.} complex numbers---come to us as a~natural extension of our previous ontological commitments---they were not `posited' for the purposes of doing physics, or whatever, they were instead a~discovery that emerged naturally from pursuing ordinary mathematics. And it is \textit{this} that gives one the confidence that they exist.\footnote{Thus I~am here resisting the idea that the indispensibility of mathematics be given a~pragmatic cast, as though it were a~tool of an engineer with an Aristotelian bent \parencite[\qv,][]{newstead_indispensability_2012}.}

\textit{En passant} this helps to solve another puzzle. It has sometimes been said that the discovery that our physical space is not Euclidean but instead has a~Riemannian curvature shows that Euclidean geometry is ``wrong''. This, I~think, is a~mis-saying. The geometrical representation of the complex numbers shows that the axioms for two-dimensional Euclidean geometry are instantiated after all. They are just not instantiated in the way one might have thought. And once we have an instantiation for Euclidean space then we get linear algebra and operators all as part of the machinery for the description of that space. The rich connections between Euclidean geometry and the real and complex numbers have been thoroughly explored, and need no further comment. Again, this is an issue we come back to.\footnote{See for example Liang Shin-Hahn \parencite*{hahn_complex_1994}; also Kaplansky \parencite*{kaplansky_linear_2003} (a reedition of 1969).}

Our realism, or platonism, has taken us as far as complex numbers and linear geometry with no reliance on the usefulness of mathematics to physics---and Bombelli died 60 years before the appearance of even Galileo's \textit{Dialogue Concerning the Two Chief World Systems}. Most curiously, the expansion of the mathematical ontology---or, to put it more accurately, the realisation that there was more ontology implicit in the initial commitment to whole numbers than had been realised---in both Pythagoras and Cardano-Tartaglia-Bombelli---involved taking roots. Once again: taking roots has been ontologically ampliative. In fact had the ancients been prepared from the outset to countenance negative numbers then the process of taking roots might have led directly to the complex numbers two millennia earlier.

Complex numbers are used routinely in quantum mechanics---but do we have any evidence that their use is unavoidable? Until recently the answer would have seemed to be `no', for it always looked possible to translate standard quantum mechanics on the complex field (CQM) into a~more cumbersome real number form (which we will abbreviate to RQM). This is hardly any form of nominalism, but it has been a~standard suggestion made against being realists about complex numbers. This situation may have changed recently by a~paper that argues that there are situations in CQM that cannot be explained in RQM \parencite{renou_quantum_2021}. The gist of the argument is that if we take three individuals, Alice, Bob and Charlie, and have two entangled photons shared between Alice and Bob, and another two shared between Bob and Charlie: when Bob measures the two particles he has received the entanglement is transferred to one between Alice and Charlie, even though they have not received particles from a~common source. The claim of Renou et al. \parencite{renou_quantum_2021} is that this transfer of entanglement can't be explained in RQM, though it can be explained in CQM. They calculate an entanglement coefficient, based on the Clauser-Horne-Shimony-Holt inequality, of 6\sqrtwo, which is higher than the maximum attainable by RQM. There is also an experimental protocol that could test this difference. If the test were to come out as the authors believe then complex numbers would not after all be eliminable in favour of real numbers.

If this is so, what we have is a~mathematical discovery that is essential for physics being made well before that physics came into existence. It would be hard in this circumstance not to come to the conclusion that mathematical discoveries are of something real that are laying the groundwork for us to make such physical discoveries.

A~very similar case is provided by the \textit{quaternions}. Hamilton's construction of these was designed to be by analogy with the complex numbers: he wished to find a~four-dimensional analogue of them to represent spatial rotation. But it was not forced by the solution of any existing equations or problems in mainstream physics or mathematics. So we once had no reason to believe that they exist---only that they could possibly exist. Nevertheless, subsequently, we may feel quite differently: W. K. Clifford's use of them in what we now call Clifford algebra, and the role that they play in the theory of spinors, may convince us that Hamilton's instincts were right, against the critics of the day. This is the issue we take up in the next section.\footnote{For the fraught history of quaternions see Simon Altmann's \parencite*{altmann_hamilton_1989}.}



\section{Spinors}

We can find an even more significant discovery that affords a~better example of mathematics preceding the physics for which it is indispensible.

In his \parencite*{cartan_les_1913} Élie Cartan discovered an entirely new representation of the orthogonal Lie Algebra SO(3) which could not be obtained from vector representations. This was, again, a~discovery in pure mathematics---following on from previous discoveries in transformation groups: there were entities which transformed in a~wholly unexpected way. Quite separately, however, Wolfgang Pauli began to employ these entities in quantum mechanics in 1927 as a~way of describing electron spin (followed, independently, by Dirac for the relativistic electron in 1928) and the mathematical entities were then named after their physical manifestations: \textit{spinors}. R. Brauer and H. Weyl described the mathematical theory of these entities in a~paper in 1935, without knowledge it seems of Clifford algebra, and then Cartan followed with a~fuller monograph in 1937---making full reference to Grassmann's exterior algebra and Clifford's usage of it. In Weyl's \textit{Classical Groups} \parencite*{weyl_classical_1939}, the fuller picture is given also.\footnote{Brauer and Weyl \parencite*[425--449]{brauer_spinors_1935}. For the English translation of Cartan's monograph: \parencite{cartan_theory_1966}. B.L. van der Waerden was the important link between Ehrenfest's physics group and the mathematical community in the early 1930's: it was the latter who simplified and made accessible the mathematics. See Veblen \parencite*{veblen_geometry_1933} and \parencite*{veblen_spinors_1934}, also Payne \parencite*{payne_elementary_1952}.}
Thus we have from Weyl \parencite*{weyl_classical_1939} the derivation of the spin representation. `Instead of the projective we have thus obtained an ordinary though double-valued representation $ \pm S~(o)$ of degree $2^{\nu}$, called the \textit{spin representation}.' Significantly, he goes on:  %Hijazi inequality.

\begin{quotation}
The normalization requires the possibility of \textit{extracting square roots}. The constructions in Euclidean geometry with ruler and compass are algebraically equivalent to the four species and the extraction of square roots. A~field in which every quadratic equation  $x^{2} - \rho = 0$ is solvable may therefore be called a~\textit{Euclidean field}. Our result is then that \textit{in every Euclidean field we can construct the spin representation}; the Euclidean nature of the field is essential. The orthogonal transformations are the automorphisms of Euclidean vector space. Only with the spinors do we strike that level in the theory of its representations on which Euclid himself, flourishing ruler and compass, so deftly moves in the realm of geometric figures. In some way Euclid's geometry must be deeply connected with the existence of the spin representation \parencite[273]{weyl_classical_1939}.
\end{quotation}
What might Weyl have meant by this enigmatic final remark? We find it echoed by Michael Atiyah. `No one fully understands spinors. Their algebra is formally understood but their general significance is mysterious. In some sense they describe the `square root' of geometry and, just as understanding the square root of $-1$ took centuries, the same might be true of spinors.' (quoted in Farmelo \parencite*{farmelo_strangest_2009}).\footnote{In a~direct reference, Atiyah, in his 2013 conference lecture ``What is a~Spinor?'' quoted Weyl's line verbatim.} What is the `square root of geometry'?

Isotropic vectors are those whose `length'---as given by the square of the modulus---is zero. So let \textbf{x} = $(x_{1}, x_{2}, x_{3})$ be an isotropic vector in a~three-dimensional space. In fact we will specify that the space is $\mathbb{C}^{3}$ to make the connection with the physics more apparent---thus each of the components is a~complex number. The isotropic vectors form a~two-dimensional surface in $\mathbb{C}^{3}$, and for each we will have $$x_{1}^{2} + x_{2}^{2} + x_{3}^{2} = 0.$$ Each such isotropic vector has associated with it two numbers $\xi_{0}$ and $\xi_{1}$ given as solutions to the following three equations: \[  \begin{array}{cc}
x_{1} = & \xi_{0}^{2} - \xi_{1}^{2},   \\[2.0mm]
x_{2} =  & i(\xi_{0}^{2} + \xi_{1}^{2}), \\[2.0mm]
x_{3} =  & -2 \xi_{0} \xi_{1}, \\
 \end{array} \] 
where these are of the form $$\xi_{0} = \pm \sqrt{\frac{x_{1} - ix_{2}}{2}}\quad \mbox{and}\quad  \xi_{1} = \pm \sqrt{\frac{- x_{1} - ix_{2}}{2}}.$$ These two numbers parameterize the two-dimensional surface of isotropic vectors. The vector  \[ \left( \begin{array}{c}
\xi_{0} \\
\xi_{1} \\ \end{array} \right)\] 
is a~spinor. But as with Bombelli's solution to Dal Ferro's formula there are two choices, depending on the sign, as the solutions come in yoked pairs (again the cross terms are discarded). So we also have
\[ \left( \begin{array}{c}
- \xi_{0} \\
- \xi_{1} \\ \end{array} \right)\] as a~second solution, analogous to the partnering of \isqrtwo{}{-1} and \mbox{\isqrtwo{$-$}{-1}.}

Though Atiyah spoke of spinors as being `square roots' of (isotropic) vectors, Cartan himself refers to them as ``polarisations''---``en quelque sorte un vecteur isotrope \textit{orienté} ou \textit{polarisé}'', where a~rotation of this vector through 2$\pi$ changes this polarisation of the isotropic vector \parencite[42]{cartan_theory_1966}. They are of course now ubiquitous in physics since fermion states are spinors. These are not unknown in relativity theory either---the light cone is represented by isotropic vectors and has associated with it spinors (with real components) which are time-like. This was the point of view emphasised by Cartan in his 1937 lectures, with particular emphasis on Minkowskian geometry. Since Brauer and Weyl in \parencite*{brauer_spinors_1935} had given an algebraic view, Cartan wanted to emphasise their relation to space-time geometry. Thus he presented 
\begin{quotation}\noindent
[\ldots] a~purely geometrical definition of these mathematical entities: because of this geometrical origin, the matrices used by physicists in Quantum Mechanics appear of their own accord, and we can grasp the profound origin of the property, possessed by Clifford algebras, of representing rotations in space having any number of dimensions. (Cartan \parencite*{cartan_theory_1966} from his Introduction.)
\end{quotation}
But, with respect to his conception of spinors, he also pointed to the impossibility of using the usual coordinate transformation techniques in Riemannian geometry (a remark that was sometimes mistakenly construed as an impossibility proof of introducing spinors into general relativity).

Spinors are closely related to the Atiyah-Singer Index theorem and K-Theory, the Seiberg-Witten theory and Alain Connes' non-commutative geometry. Roger Penrose has made them the centrepiece of his proposed unification scheme for relativity and quantum mechanics in Twistor theory \parencite{penrose_road_2004_ah}. Their fundamental character would be hard to overestimate---and yet they emerged, firstly, from pure mathematics, only to (independently) come, some 13 years later, to represent a~property that had no macroscopic visualisation: a~hitherto unsuspected property of matter that arose first from the abstract study of Lie groups---from the Lie group SO(3) and its double cover SU(2). This is surely one of the most dramatic and least heralded examples of the uncovering of mathematical structures in nature. And here the mathematics seems very close to being directly physically detectable in the form of spin eigenvalues. And due to the character of the double cover SU(2) spinors have the remarkable property that if we pick an isotropic vector and rotate it through 2$\pi$ it returns to its original position but the spinor is only rotated through $\pi$ and its sign is reversed. It takes a~rotation of 4$\pi$ to bring it back to its original state. It is argued in Christian (2014) that this is also measurable.\footnote{See Penrose and Rindler \parencites*{penrose_spinors_1987}{penrose_spinors_1988}. Also Claude Chevalley \parencite*{chevalley_algebraic_1997}, particularly the afterword by J.-P. Bourguignon; also Lounesto \parencite*{lounesto_clifford_2001}.} Moreover it is remarkable that the spin values of the fermions and bosons arise directly from the dimension of the irreducible representations of the Lie algebra $\mathfrak{sl} (3)$, which is the Lie algebra of the groups SO(3) and SU(2)---the former giving the spin values for bosons and the latter for fermions.

The non-classical nature of a~spinor's double-rotational invariance is surprising and constitutes a~challenge to the idea that particles can be seen as physical objects in the classical manner. Despite this, and acknowledging that it represents only a~partial solution to the geometrical problem, Penrose has ingeniously utilised the properties of the Riemann Sphere $\mathbb{P}(\Hilbtwo)$ to give a~graphical representation of the pure states of spin. It is when one moves to higher fermionic spin states that this picture---the \textit{Majorana picture}---becomes highly non-classical and defies ready visualisation. Penrose pointed out that as we aggregate matter to form higher spin values that  there is no convergence to the classical picture, rather the opposite. 
\begin{quotation}\noindent
[\ldots] we see that a~randomly chosen quantum system with a~\textit{large} angular momentum (large \textit{j} value) has a~state defined by a~Majorana description consisting of 2\textit{j} points more-or-less randomly peppered about the sphere $S^{2}$. This bears no resemblance to the classical angular momentum state of a~system of large angular momentum, despite the common impression that a~quantum system with large values for its quantum numbers should approximate a~classical system! [\ldots] The answer is that almost all `large' quantum states do not resemble classical ones \parencites[566]{penrose_road_2004_ah}[also see][]{penrose_spinors_1987}{penrose_spinors_1988}.
\end{quotation}

But despite defying ready geometric visualisation, spinors are required in quantum theory. Since the work of Cartan, Weyl, and then Chevalley in the 1950's it has become clear that the natural home for a~discussion of spinors is Clifford Algebra. And within the Clifford Algebra in which the simplest expression of quantum mechanical spin is representable, the 8-dimensional algebra usually denoted $\mathit{Cl}_{3}$, the real numbers and the complex numbers are naturally represented as sub-algebras. Thus, spinors represent a~\textit{culmination of algebraic structure} within the structures applicable in physics, that includes the real and complex numbers, and also the quaternions. And it is the unit quaternions that are the spinors as defined by Pauli. Thus Clifford Algebras encapsulate and relate together these seemingly different mathematical structures---all of which are intimately related to our most successful physical theories and in the case of the real and  complex numbers, spinors, and quaternions, actually preceded them.

We can close the circle on the progression that we have been noting here: from right angled triangles to the Pythagorean understanding of irrationality and the real numbers, to complex numbers, to spinors, by mentioning a~remarkable fact: Pythagorean triples can be understood as generating spinors defined on the null vectors of $\mathbb{Z}^{3}$. This is due to the mapping induced by the Euclidean parameters $(p, q)$, with $p > q$, to the Pythagorean triples $(x, y, z)$ by
$$ x~= p^{2} - q^{2}, \quad y~= 2pq,\quad z~= p^{2} + q^{2}.$$
At least one of the numbers $(x, y)$ must be even. The \textit{primitive} Pythagorean triples are those that are mutually prime. A~\textit{standard} Pythagorean triple is one which is either primitive with $z$ positive and $y$ even, or $(\frac{x}{2}, \frac{y}{2}, \frac{z}{2})$ is primitive and $\frac{y}{2}$ is odd. Thus the triple $(3, 4, 5)$ is standard, whereas $(4, 3, 5)$ is not. Then, it is provable that for every standard Pythagorean triple there is a~pair of Euclidean parameters that are relatively prime which generate the Pythagorean triple. This is then a~one-to-one correspondence (bijection) between the directions in $\mathbb{Z}^{2}$ and the null directions in $\mathbb{Z}^{3}$.\footnote{See Trautman \parencite*{trautman_pythagorean_1998} Proposition 1. These ideas are developed in greater detail in Kocik \parencite*{kocik_clifford_2007}, though without acknowledging Trautman's prior work. Kocik links this with quasi-quaternions and the Apollonian Gasket.}

Euclid's discovery of the parameterisation of Pythagorean triples may be viewed then as the first recorded use of a~spinor space.

This in turn is related to complex numbers: $c = p~+ qi$, since the norm is equal to $cc*$, the complex number multiplied by its conjugate, which is $$p^{2} + q^{2}.$$ And the square of the complex number is 
\[ (p^{2} - q^{2}) + 2pqi.\]
Thus the squares of certain integer complex numbers generate Pythagorean triples. Or, to put it another way, Pythagorean triples have square roots that are integer complex numbers. A~comparison with the immediately preceding discussion of isotropic vectors shows that Euclid's three equations for Pythagorean triples are analogous to the equations that define a~spinor in Cartan's formulation. Pythagorean triples \textit{are} spinors in $\mathbb{Z}^{2}$! As Kocik \parencite*{kocik_clifford_2007} puts it: `Euclid's discovery of the parameterisation of Pythagorean triples may be viewed then as the first recorded use of a~spinor space.'

This appears to vindicate Weyl's mysterious remark.\footnote{Of course we might also add, for further evidence, that the square root of the classical Laplacian is the Dirac operator of relativistic quantum mechanics---and this takes us to the Lorentz invariant spinors of Cartan. There is thus a~sense, not entirely figurative, in which quantum mechanics is the square root of classical mechanics, as suggested by Penrose.} But it also emphasises that there is a~connection between the metric on the space and the definition of spinors---so that the latter actually \textit{requires} the former. This dependence is further discussed in Bär \textit{et al.} \parencite*{bar_generalized_2005} and Bourguinon \textit{et al} \parencite*{bourguignon_spinorial_2015}.

Let us return briefly to Penrose's idea of the centrality of the Riemann sphere. As noted, he pointed out that a~spin-\textonehalf\ particle can have the possible directions in which its spin can be measured mapped to the Riemann sphere. But he then said:
\begin{quotation}
Although quantum amplitudes seem to be very abstract things, having this strange `square root' relation to a~probability, they actually have close associations with space-time geometry \parencite[230]{arnold_mathematical_2000}.
\end{quotation}
To make this connection he noted that being situated at a~point in space the light cone at that point can also be represented by a~Riemann sphere. This sphere represents all of the light like rays that pass through the observer's point in space. This Riemann sphere is then conformally deformed if we pass to another observer passing through that same point with a~different velocity, Thus the non-reflective Lorentz transformations can be represented by complex conformal transformations of the Riemann sphere. It would be interesting to consider that these different usages of the Riemann sphere could be unified by Cartan's geometrical picture of spinors as square roots of null vectors.\footnote{Penrose \parencite*{penrose_road_2004_ah} does not reference Cartan in this context.}


\section{Realism defended}


The enlargement of mathematical ontology from Pythagoras through to Cartan and Weyl is properly the uncovering of structure already present, and uncovered through the process of doing ordinary mathematics---solving equations, constructing proofs, analysing existing mathematical structures. And through this process mathematicians have given us an understanding of real numbers and analysis, including differential geometry; complex numbers and their associated structures in geometry and algebra; and spinors and their structures. In these three cases the mathematical structures preceded, sometimes by centuries, their application in physics. 

We can thus see the danger in an over-reliance on the indispensability thesis. There is a~strongly pragmatist construal of this thesis that would have it that the \textit{only} reason we should believe in mathematical entities is their usefulness in physical explanation---with the implication that if they had \textit{not} found an application in physical explanation we would not have reason to believe in them. This does an injustice to the very thing that makes mathematical epistemology unique: \textit{proof}. A~far more compelling fact about the use of mathematics in physics is that the mathematical discoveries made by entirely different methods often precede the discovery that they can be found also in the natural world. It is this that should keep the Nominalist awake at night. But we should accept a~more modest role for indispensability: that physics is capable of providing a~layer of additional confirmation that mathematical structures and entities exist, and moreover that this existence should not be regarded as an \textit{abstract} matter, for they are part of the fabric of the Universe.

Thus let us consider the most well-developed nominalist view: that proposed by Hartry Field in his \parencite*{field_science_1980}. The central idea is to take congruences on a~Newtonian space as giving one all the `numbers' that we need. And yet I~think it misses the mark. As suggested earlier, if the nominalist is permitted to help himself to space-time as a~flat 4-dimensional differentiable manifold \textit{with a~metric structure} then he has thereby helped himself to the real numbers already, both in the metric and also in the differentiable structure. For an \textit{n}-dimensional differentiable manifold is locally isomorphic to $\mathbb{R}^{n}$.\footnote{This style of criticism of Field was signalled early on by Michael D. Resnik in his review of Field's book: \parencites{resnik_hartry_1983}[also][]{resnik_how_1985}[see also][]{steiner_applicability_1998}.} In fact a~4-dimensional, not necessarily flat, differentiable manifold proves to be unlucky for Field and nominalists generally as it is the only dimension for which there can exist an infinite number of \textit{distinct} quasi-conformal structures---and thus there can be more than one way to determine the local mappings to $\mathbb{R}^{4}$ that are conformally inequivalent \parencite{donaldson_quasiconformal_1989}. These are not simply many different metrics definable on a~differentiable manifold---which would be a~trivial point and would not distinguish 4-manifolds. Rather the quasi-conformal structures are distinct in that \textit{no} finite amount of stretching or shrinking of the metric will deform one quasi-conformal manifold into another, despite being topologically identical. (Guided as we are by 2- and 3-dimensional topology this seems impossible to visualise.) The problem for Field is that these infinite possibilities are precisely the kind of \textit{abstracta} that his nominalism cannot countenance.

If the space-time is Newtonian (as it is for Field) then the metric is globally singular though well-defined on the time-like fibrations---this alone creates complications since then his congruence relations are only defined on the fibrations. If it is Minkowskian then it is globally well-defined everywhere.\footnote{This leads David Malament in his review \parencite*{malament_science_1982} to shift Field's case to a~Klein-Gordon scalar field theory.} Field is of necessity a~substantivalist about space-time geometry but I~cannot see how comparatives will allow him to give a~Nominalist construal of the light cone structure, since \textit{all} points that are light-like separated \emph{have 0 distance from one another by the Lorentz-signature metric, even when they are collinear}!  This by itself refutes the idea that congruence can be a~nominalist substitute for the role that the metric structure plays! Since Minkowski space-time is a~more realistic space-time structure than Field's preferred euclidean space this seems definitive.

But I'd like to sketch an ancilliary argument of a~different kind, which suggests that Field's strategy does not do away with numbers in the way he suggests, even on Euclidean space. Suppose that there were two four-dimensional manifolds, one with its metrical structure determined by a~mapping to $\mathbb{R}^{4}$ and the other with the a~mapping to the quaternions $\mathbb{H}$ or even to $\mathbb{C} \times \mathbb{C}$. According to Field's nominalism these spaces are acceptable because they can be construed substantivally, though the metrical structures are not taken to be substantive, because they involve numbers. So his plan is to eliminate these metrical structures using his reformulation thesis in favour of segment-congruences. But this presents him with a~dilemma: either the results of this elimination gives us the same `space-time', or they are different. If they are the same then the reformulation has eliminated crucial information---because multiplication (and therefore segment length) acts differently in these cases---but if they are different then the metrical structures are still present in an implicit form: we have simply stopped using useful numerical words! We could run this same argument with a~comparison of $\mathbb{R}^{4}$ with Minkowski space-time, or with a~different signature metric entirely, such as (+ + - -), or, most significantly, with a~Kähler 4-manifold in which there is more than one metric-like structure. 

If real numbers are smuggled in in the form of geometric structure then the nominalist, though helping himself to a~lot, has still not got enough even for the simplest cases of quantum mechanics. If we consider the Hilbert space as a~space of the possible states of a~system then it is clear that even in the simplest case of $\mathbb{C}^{2}$---for a~spin-half spinor space---it is not reducible to anything that Field is prepared to countenance---for despite being topologically identical to $\mathbb{R}^{4}$---which Field needs for the purpose of his space-time structure---it is precisely \textit{unlike} $\mathbb{R}^{4}$ in its metrical features. And for the Hilbert space of the spin-1 particle there is simply nothing available at all. The problems are then only compounded from this point on. Once we begin to consider quantum field theory we must consider spaces of operators that are defined on each of the space-time points. So let us consider the noncommuting operators of the electro-magnetic field: then the algebra will be an infinite-dimensional noncommutative algebra. Dispensibilist Instrumentalism has no hope in this case, nor has it ever been attempted.\footnote{Of course, we can accept, with Field, that there is no canonical natural isomorphic mapping of an \textit{n}-manifold to $\mathbb{R}^{4}$. But that is less than Field needs, for we can allow that the metric is defined only up to a~scale factor without abandoning the idea that the metric is a~part of the space. The metric simply becomes an equivalence class of numerical assignments, equivalent up to a~scalar factor. But by only allowing congruence classes of intervals Field ends up with less than this---and here we come back to one of the main themes of this paper---for he cannot capture the  facts about incommensurable magnitudes that so impressed the Greeks. Thus consider again the 1:1:\sqrtwo triangle. Congruence classes will allow Field to say that the two catheti are the same length, but \textit{not} that no integer that is assigned to that class will allow an integer to be assigned to the hypotenuse, or vice versa. Incommensurability is a~pairwise, \textit{metrical} relation, and it is entirely \textit{intrinsic} to the space. So it holds for every coordinatization in the equivalence class that defines the metric.}

The \textit{In}dispensibilist Instrumentalist might accept all of this as evidence of the indispensability of said mathematics but insist that we can think of the mathematics as merely ``indexing'' the physical facts. The term `indexing' comes from Melia \parencite*{melia_weaseling_2000} and is meant to cover the use of real numbers for distances as well as other cases of measured magnitudes. However it is not at all clear what else it is meant to cover and without a~very clear recipe for applying the term the charge of question begging will be hard to avoid (see Daly and Langford \parencite*{daly_mathematical_2009} for a~defence of this way of understanding Indispensibilist Instrumentalism).\footnote{I~say nothing in this paper about structuralism, as I've discussed it elsewhere---see Heathcote \parencite*{heathcote_exhaustion_2014}. In its anti-realist form structuralism is unable to address the objections made here.}

Thus it is hard to see how we can account for \textit{dimensionless} physical constants---such as Summerfeld's \textit{fine structure constant} $\alpha = 0.0072973525693\ldots$, first introduced in 1917. The constant has the (or one) meaning:
\[ \alpha\; =\; \frac{1}{4 \pi e_{0}} \frac{e^{2}}{\hbar c}.  \]
Here $e_{0}$ is the electric constant and $e^{2}$ is the square of the elementary charge of an electron. The value of the constant has been measured very accurately---and the accuracy is always improving, but it does not seem to be related to any known mathematical constants, and note that it isn't clear, and may never be clear, whether it is rational or irrational.\footnote{The fine structure constant is often given in the reciprocal form $\alpha^{-1} = 137.035999206$\ldots} And yet, it has been argued that if $\alpha$ were different by even a~small amount then the Universe would not exist: matter as we know it would not exist. However it is not its precise value that is our concern here, but simply the fact that it is a~\textit{dimensionless number.} For the nominalist view is that numbers don't exist, and thus that $\alpha$ does not exist either. But if that is the case then, never mind its exact value, \textit{no} such value exists---and so matter can't exist. No form of nominalism of which I'm aware has made an attempt to deal with this problem of dimensionless constants such as $\alpha$---and no strategy suggests itself. That is the realist argument in its starkest form, and indeed may summarise the point of this paper: \textit{either numbers exist or nothing exists.}\footnote{As Wolfgang Pauli is alleged to have said: `When I~die my first question to the Devil will be: What is the meaning of the fine structure constant?' Of course there are other dimensionless physical constants besides $\alpha$ that could make the same point.}


But, as hinted at earlier, I~believe we can find a~simpler case, with ancient and venerable Platonic credentials, that seems rather clearly to not be a~case of mere indexing. And it is one that is equally as hard for any form of nominalism that is currently espoused.\footnote{Jody Azzouni has recently resurrected, in his \parencite*{azzouni_deflating_2006} and \parencite*{azzouni_talking_2010}, a~form of pure fictionalism about mathematics---mixed with a~form of social constructionism---that seems particularly vulnerable to this challenge, as it makes no attempt to deal with the mathematics that occurs in physics and is content to discuss the counting and computation of natural numbers\parencite[see][]{batterman_explanatory_2010}.}

The argument is as follows. Premise 1: Whether an action \textit{can be} performed, or a~task completed, has a~determinate truth value: one either can or can't. Premise 2: Whatever facts the ability to perform the act or complete the task depends upon must likewise  be determinate.  But consider the task set by the Delphic oracle to the Delians: they were required to double the size of their altar---which was cubic shaped. And let us suppose, as Plato apparently did, when the Delians approached him on the matter, that this doubling of the cube must be done only within constructive geometry, that is with straightedge and compass, anything else being merely approximate.\footnote{Thus note Plutarch's comment on this: `And therefore Plato himself dislikes Eudoxus, Archytas, and Menaechmus for endeavouring to bring down the doubling of the cube to mechanical operations; for by this means all that was good in geometry would be lost and corrupted, it falling back again to sensible things, and not rising upward and considering immaterial and immortal images [\ldots] .' Platonic Questions, Quest. 2, \textit{Moralia}.} The doubling of the cube requires finding $\sqrt[3]{2}$ which is irrational (the proof is an easy generalisation of some of the proofs of the irrationality of \sqrtwo). But it is also a~non-constructible number---as proven by Wantzel in 1837. And this means that there is no way to perform the action required by the oracle. So it is false that $\sqrt[3]{2}$ is constructible and so false that the Delians task can be performed. The same argument can be run using \textit{squaring the circle} as the example, where the impossibility depends upon the transcendental character of $\pi$, which implies that it too is non-constructible.\footnote{As proven by Lindemann in 1882. The impossibility of squaring the circle was probably known by the time Plato was writing: Aristophanes ridicules circle-squarers in \textit{The Birds}.} The point is that mathematics is not just, as a~discipline, indispensable to science, it is that mathematical facts \textit{constrain and determine physical facts, and cannot easily be distinguished from them}. Thus, as another example, it is the topology of the 2-sphere that determines that there must be some point on Earth where the wind does not blow. It is impossible to partition explanation into the physical versus the mathematical in a~way that leaves Nominalism with any clear content. Once we have let in what is needed for physical explanation then mathematics has been let in as well. This is particularly the case with the structures chosen here: the division algebras and the spinor structures. Mathematics and physics seem to have converged.


\section{The royal road to ontology}

It is time to take stock. 

In the process of taking roots we have jumped from a~discrete structure to continuous structures, in other words to geometry. In the first instance this led us to the real numbers, via incommensurable magnitudes and irrational numbers. Then in a~second step we were led to the complex numbers and their richer geometry. And then, through complex numbers, Clifford algebras, and quaternions, we arrived at spinors. I've argued that there is no plausible nominalist strategy that can account for these structures: Field's nominalist strategy won't work and---even if its problems could be set aside (as I~believe they cannot)---we would confront the problem of the dimensionless constants. This latter problem defeats even a~putative structuralist solution. Nor is a~narrow indispensibilist explanation plausible. My suggestion in this paper has been that the major steps of this progress would warrant a~realistic attitude to these entities even if one could lay aside the application of this mathematics to physics.

So the process of taking roots turns out to be ontologically ampliative, and resists nominalistic reduction. Should we find this surprising? One might suggest, aphoristically, that platonism manifests itself in its most irresistible form as geometry. In support we may quote Shing-Tung Yau, the inventor of Calabi-Yau manifolds, on the importance of geometry: 
\begin{quotation}
Since the time of the Greek mathematicians, geometry has always been at the centre of science. Scientists cannot resist explaining natural phenomena in terms of the language of geometry. Indeed, it is reasonable to consider geometric objects as part of nature. Practically all elegant theorems in geometry have found applications in classical or modern physics \parencite[253]{arnold_review_2000}.
\end{quotation}
This of course is not to seek to take anything away from algebra, or to suggest that arguments for realism do not extend to algebras. How could they not when there is such a~close relationship between algebra and geometry? If geometry may be likened to the face, then algebra is the mind behind the face. As Kähler said (in a~philosophical essay): ``[\ldots] one must interpret the development of algebra as the revelation of the realm of ideas postulated by \textit{Plato}'' \parencite{kahler_il_2003}.

Thus this argument for mathematical realism gives precedence to the reals over the integers, and to the complex numbers over the reals. This is not to say that nominalism can easily deal with the integers---I believe that even here it must fail. But in mathematical terms the integers are now just one example of a~commutative ring, one among an infinite number of others---and quantum mechanics has directed our attention to the \textit{non}-commutative rings as possibly equally or more fundamental. The primacy of the three associative division algebras in mathematical explanation — the reals, the complex numbers and the quaternions — is what I~mean by saying that these `almost geometrical structures' are the primary basis for mathematical realism, a~meaning that is in accord with Plato's own emphasis on the importance of geometry. These division algebras and their associated higher structures, such as Clifford Algebras, or spin representations, are structures about which we \textit{must} be realist.\footnote{In this context, the importance of group representation theory in quantum physics is worth emphasising. For here we take an often complicated non-linear algebraic object like a~group and we consider it under the aspect of a~geometric object by considering a~homomorphism to a~vector space. That this is especially fruitful has been argued often, as far back as Weyl \parencite*{weyl_theory_1931} or Wigner \parencite*{wigner_unitary_1939}. For additional comments on this see Heathcote \parencite*{heathcote_multiplicity_2021}.} It is here that the evidence is most irresistible. Indeed, if we turn the matter around, we could say this: the \textit{only} plausible explanation for physics continually using the seemingly abstract mathematical structures uncovered by mathematics is that our universe contains those mathematical facts as generalised, non-local, parts of itself. In short: as `geometry'. My historical conjecture is that this was itself Plato's original insight, inscribed on the entrance to the Academy: \textit{Let No-one Unskilled in Geometry Enter Here}.


\paragraph{Acknowledgement:} I~wish to express my warm thanks to the readers for the journal who offered useful suggestions.

\paragraph{Declaration:} The author declares that there are no conflicts of interest, no funding issues, and no ethics issues involved with this paper.




\end{artengenv}


