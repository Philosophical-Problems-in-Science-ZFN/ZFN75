\begin{editorialeng}{Michał Heller}
	{At the interface of theory and experience}
	{At the interface of theory and experience}
	{At the interface of theory and experience}
	{Copernicus Center for Interdisciplinary Studies}



\lettrine[loversize=0.13,lines=2,lraise=-0.03,nindent=0em,findent=0.2pt]%
{T}{}he founding motto of philosophy in science is ``tracking down big philosophical problems in contemporary science.'' Knowing the basic history of philosophy and the history of science, we more or less know what ``big philosophical topics'' mean. The most representative topics of this kind include: time, space, causality, matter, life, consciousness, thinking... The tables of contents of philosophy textbooks could be copied to continue this list. These topics are big not only when they remain at a~high level of generality, but also when they get down to special cases and particular sub-problems. Sometimes it is only then that they fully reveal their big format.

But where in science should we pursue these topics? As usual, when struggling with a~difficult question, it is worth limiting ourselves to an easier case. Such a~``methodologically easier'' case is, of course, physics; this is where we will focus our attention in this short essay.

But where exactly in physics should we look for these philosophical topics? To be sure, in the core of modern physics, that is, at the interface of theory and experience. The final instance for physical theories is experience, but experience without theory would be reduced to crude sensory perceptions, which have little to do with science and are completely powerless against more advanced physical theories. Not only should we look for traces of great philosophical problems in the interface between the theories of physics and experiment, but this interface itself creates a~great philosophical problem which could only be vaguely intuited in the old problems of philosophical epistemology.

For obvious reasons, the problem of the relationship between the mathematical formalism of theory and empirical data is also one of the main, if not simply the main, problem in contemporary philosophy of science. Moreover, this problem is becoming more and more urgent. Some theories of modern physics seem to reach domains in which experiment is impossible, either for financial reasons (theories of extremely high energies) or for even more fundamental reasons (theories of multiverses). Is physics without the possibility to confront its hypotheses with experimental data still physics? The question of the relationship between formalism and experience becomes the question of the identity of physics as a~science.

Undoubtedly, the identity of modern physics was determined by its empirical character. Rapid progress in physics occurred precisely when experience became the main criterion for the acceptability of its theories. The turning point in the emergence of modern science was the departure from the belief, cultivated throughout antiquity and the Middle Ages, that the universe can be reconstructed basing on rigorous deduction from ``first principles'' and the understanding that such a~deduction must---as Whitehead elegantly put it---face ``irreducible and stubborn facts'', and if the facts stubbornly persist despite the results of the deduction, then the whole deduction, together with its conclusions, must be abandoned.

As physical theories became more and more sophisticated, the understanding of their empirical character (that is supposed to constitute the identity of physics) became less and less obvious. In fact, the entire history and philosophy of science of the last two centuries has revolved around this concept.

Empiricism achieved its maximum in the views of logical empiricism, which postulated the reduction of the entire theoretical ``superstructure'' of modern physics to direct empirical data. Although logical empiricism did not survive into the 21\textsuperscript{st} century, it left a~strong mark on contemporary philosophy of science. One of the clearest features of this heritage are empiricist tendencies. Of course, there is no return to the idea of direct reports of experimental results (the so-called elementary propositions), to which all physical theories should be reduced. No one denies that mathematical formalism is an important element of physical theories, but in many so-called case studies, i.e. in methodological analyzes of specific theories or models of contemporary physics, we find attempts to distinguish as clearly as possible those elements of formalism that can be directly associated with measurement procedures. What is evident in these attempts is the idea that a~given physical theory will be more empirical the more precisely it can be done.

This is not how it works in the scientific practice of physicists. The practice of physics is much more monolithic. When you enter a~modern physics laboratory, you take a~closer look at all this complicated equipment (if it is possible at all, because it may have dimensions far beyond what you can see) and look at the diagrams in which the results of the experiment are encoded, you can really have the impression that you are touching a~nerve of reality. But you only need to look a~little more carefully into what is actually happening here to understand that it is impossible to draw even a~relatively sharp line separating what is theoretical from what is empirical.

It would seem that at least what is theoretical can be clearly distinguished from what is empirical. After all, ``theoretical'' is simply the mathematical formalism of a~theory. But that is not entirely true. Because the mathematical formalism of the theory can virtually contain the results of future measurements. This is eloquently evidenced by the history of the field equations of general relativity, which ``knew'' about future empirical discoveries (microwave background radiation, gravitational radiation and many others) much earlier than they could be made.

It is often said, somewhat metaphorically, that theoretical and empirical elements in physical theories are nonlinearly coupled with each other. This is an apt metaphor. Just as the solution of a~nonlinear differential equation cannot be decomposed into the sum of two solutions to that equation, a~physical theory cannot be decomposed into the sum of a~theoretical component and an empirical component.

According to aesthetic criteria, that go back to the shadows of logical empiricism, this would be an argument on behalf of the thesis that the theories of modern physics do not meet the criterion of being an empirical science. I~think that it is just the opposite: physics is an empirical science precisely because the empiricism runs so deep into its theoretical body that it cannot be separated from it.

This coupling of mathematical formalism and empirical results, the element of rationalism and the element of empiricism, constitutes a~Big Philosophical Problem. We have here not only a~case for \textit{philosophy in science}, but also a~beautiful example of what physics contributes to Big Philosophical Problems.

%\begin{flushright}
%Michał Heller
%\end{flushright}

\end{editorialeng}
